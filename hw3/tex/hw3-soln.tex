
% Default to the notebook output style

    


% Inherit from the specified cell style.




    
\documentclass[11pt]{article}

    
    
    \usepackage[T1]{fontenc}
    % Nicer default font (+ math font) than Computer Modern for most use cases
    \usepackage{mathpazo}

    % Basic figure setup, for now with no caption control since it's done
    % automatically by Pandoc (which extracts ![](path) syntax from Markdown).
    \usepackage{graphicx}
    % We will generate all images so they have a width \maxwidth. This means
    % that they will get their normal width if they fit onto the page, but
    % are scaled down if they would overflow the margins.
    \makeatletter
    \def\maxwidth{\ifdim\Gin@nat@width>\linewidth\linewidth
    \else\Gin@nat@width\fi}
    \makeatother
    \let\Oldincludegraphics\includegraphics
    % Set max figure width to be 80% of text width, for now hardcoded.
    \renewcommand{\includegraphics}[1]{\Oldincludegraphics[width=.8\maxwidth]{#1}}
    % Ensure that by default, figures have no caption (until we provide a
    % proper Figure object with a Caption API and a way to capture that
    % in the conversion process - todo).
    \usepackage{caption}
    \DeclareCaptionLabelFormat{nolabel}{}
    \captionsetup{labelformat=nolabel}

    \usepackage{adjustbox} % Used to constrain images to a maximum size 
    \usepackage{xcolor} % Allow colors to be defined
    \usepackage{enumerate} % Needed for markdown enumerations to work
    \usepackage{geometry} % Used to adjust the document margins
    \usepackage{amsmath} % Equations
    \usepackage{amssymb} % Equations
    \usepackage{textcomp} % defines textquotesingle
    % Hack from http://tex.stackexchange.com/a/47451/13684:
    \AtBeginDocument{%
        \def\PYZsq{\textquotesingle}% Upright quotes in Pygmentized code
    }
    \usepackage{upquote} % Upright quotes for verbatim code
    \usepackage{eurosym} % defines \euro
    \usepackage[mathletters]{ucs} % Extended unicode (utf-8) support
    \usepackage[utf8x]{inputenc} % Allow utf-8 characters in the tex document
    \usepackage{fancyvrb} % verbatim replacement that allows latex
    \usepackage{grffile} % extends the file name processing of package graphics 
                         % to support a larger range 
    % The hyperref package gives us a pdf with properly built
    % internal navigation ('pdf bookmarks' for the table of contents,
    % internal cross-reference links, web links for URLs, etc.)
    \usepackage{hyperref}
    \usepackage{longtable} % longtable support required by pandoc >1.10
    \usepackage{booktabs}  % table support for pandoc > 1.12.2
    \usepackage[inline]{enumitem} % IRkernel/repr support (it uses the enumerate* environment)
    \usepackage[normalem]{ulem} % ulem is needed to support strikethroughs (\sout)
                                % normalem makes italics be italics, not underlines
    

    
    
    % Colors for the hyperref package
    \definecolor{urlcolor}{rgb}{0,.145,.698}
    \definecolor{linkcolor}{rgb}{.71,0.21,0.01}
    \definecolor{citecolor}{rgb}{.12,.54,.11}

    % ANSI colors
    \definecolor{ansi-black}{HTML}{3E424D}
    \definecolor{ansi-black-intense}{HTML}{282C36}
    \definecolor{ansi-red}{HTML}{E75C58}
    \definecolor{ansi-red-intense}{HTML}{B22B31}
    \definecolor{ansi-green}{HTML}{00A250}
    \definecolor{ansi-green-intense}{HTML}{007427}
    \definecolor{ansi-yellow}{HTML}{DDB62B}
    \definecolor{ansi-yellow-intense}{HTML}{B27D12}
    \definecolor{ansi-blue}{HTML}{208FFB}
    \definecolor{ansi-blue-intense}{HTML}{0065CA}
    \definecolor{ansi-magenta}{HTML}{D160C4}
    \definecolor{ansi-magenta-intense}{HTML}{A03196}
    \definecolor{ansi-cyan}{HTML}{60C6C8}
    \definecolor{ansi-cyan-intense}{HTML}{258F8F}
    \definecolor{ansi-white}{HTML}{C5C1B4}
    \definecolor{ansi-white-intense}{HTML}{A1A6B2}

    % commands and environments needed by pandoc snippets
    % extracted from the output of `pandoc -s`
    \providecommand{\tightlist}{%
      \setlength{\itemsep}{0pt}\setlength{\parskip}{0pt}}
    \DefineVerbatimEnvironment{Highlighting}{Verbatim}{commandchars=\\\{\}}
    % Add ',fontsize=\small' for more characters per line
    \newenvironment{Shaded}{}{}
    \newcommand{\KeywordTok}[1]{\textcolor[rgb]{0.00,0.44,0.13}{\textbf{{#1}}}}
    \newcommand{\DataTypeTok}[1]{\textcolor[rgb]{0.56,0.13,0.00}{{#1}}}
    \newcommand{\DecValTok}[1]{\textcolor[rgb]{0.25,0.63,0.44}{{#1}}}
    \newcommand{\BaseNTok}[1]{\textcolor[rgb]{0.25,0.63,0.44}{{#1}}}
    \newcommand{\FloatTok}[1]{\textcolor[rgb]{0.25,0.63,0.44}{{#1}}}
    \newcommand{\CharTok}[1]{\textcolor[rgb]{0.25,0.44,0.63}{{#1}}}
    \newcommand{\StringTok}[1]{\textcolor[rgb]{0.25,0.44,0.63}{{#1}}}
    \newcommand{\CommentTok}[1]{\textcolor[rgb]{0.38,0.63,0.69}{\textit{{#1}}}}
    \newcommand{\OtherTok}[1]{\textcolor[rgb]{0.00,0.44,0.13}{{#1}}}
    \newcommand{\AlertTok}[1]{\textcolor[rgb]{1.00,0.00,0.00}{\textbf{{#1}}}}
    \newcommand{\FunctionTok}[1]{\textcolor[rgb]{0.02,0.16,0.49}{{#1}}}
    \newcommand{\RegionMarkerTok}[1]{{#1}}
    \newcommand{\ErrorTok}[1]{\textcolor[rgb]{1.00,0.00,0.00}{\textbf{{#1}}}}
    \newcommand{\NormalTok}[1]{{#1}}
    
    % Additional commands for more recent versions of Pandoc
    \newcommand{\ConstantTok}[1]{\textcolor[rgb]{0.53,0.00,0.00}{{#1}}}
    \newcommand{\SpecialCharTok}[1]{\textcolor[rgb]{0.25,0.44,0.63}{{#1}}}
    \newcommand{\VerbatimStringTok}[1]{\textcolor[rgb]{0.25,0.44,0.63}{{#1}}}
    \newcommand{\SpecialStringTok}[1]{\textcolor[rgb]{0.73,0.40,0.53}{{#1}}}
    \newcommand{\ImportTok}[1]{{#1}}
    \newcommand{\DocumentationTok}[1]{\textcolor[rgb]{0.73,0.13,0.13}{\textit{{#1}}}}
    \newcommand{\AnnotationTok}[1]{\textcolor[rgb]{0.38,0.63,0.69}{\textbf{\textit{{#1}}}}}
    \newcommand{\CommentVarTok}[1]{\textcolor[rgb]{0.38,0.63,0.69}{\textbf{\textit{{#1}}}}}
    \newcommand{\VariableTok}[1]{\textcolor[rgb]{0.10,0.09,0.49}{{#1}}}
    \newcommand{\ControlFlowTok}[1]{\textcolor[rgb]{0.00,0.44,0.13}{\textbf{{#1}}}}
    \newcommand{\OperatorTok}[1]{\textcolor[rgb]{0.40,0.40,0.40}{{#1}}}
    \newcommand{\BuiltInTok}[1]{{#1}}
    \newcommand{\ExtensionTok}[1]{{#1}}
    \newcommand{\PreprocessorTok}[1]{\textcolor[rgb]{0.74,0.48,0.00}{{#1}}}
    \newcommand{\AttributeTok}[1]{\textcolor[rgb]{0.49,0.56,0.16}{{#1}}}
    \newcommand{\InformationTok}[1]{\textcolor[rgb]{0.38,0.63,0.69}{\textbf{\textit{{#1}}}}}
    \newcommand{\WarningTok}[1]{\textcolor[rgb]{0.38,0.63,0.69}{\textbf{\textit{{#1}}}}}
    
    
    % Define a nice break command that doesn't care if a line doesn't already
    % exist.
    \def\br{\hspace*{\fill} \\* }
    % Math Jax compatability definitions
    \def\gt{>}
    \def\lt{<}
    % Document parameters
    \title{hw3-soln}
    
    
    

    % Pygments definitions
    
\makeatletter
\def\PY@reset{\let\PY@it=\relax \let\PY@bf=\relax%
    \let\PY@ul=\relax \let\PY@tc=\relax%
    \let\PY@bc=\relax \let\PY@ff=\relax}
\def\PY@tok#1{\csname PY@tok@#1\endcsname}
\def\PY@toks#1+{\ifx\relax#1\empty\else%
    \PY@tok{#1}\expandafter\PY@toks\fi}
\def\PY@do#1{\PY@bc{\PY@tc{\PY@ul{%
    \PY@it{\PY@bf{\PY@ff{#1}}}}}}}
\def\PY#1#2{\PY@reset\PY@toks#1+\relax+\PY@do{#2}}

\expandafter\def\csname PY@tok@w\endcsname{\def\PY@tc##1{\textcolor[rgb]{0.73,0.73,0.73}{##1}}}
\expandafter\def\csname PY@tok@c\endcsname{\let\PY@it=\textit\def\PY@tc##1{\textcolor[rgb]{0.25,0.50,0.50}{##1}}}
\expandafter\def\csname PY@tok@cp\endcsname{\def\PY@tc##1{\textcolor[rgb]{0.74,0.48,0.00}{##1}}}
\expandafter\def\csname PY@tok@k\endcsname{\let\PY@bf=\textbf\def\PY@tc##1{\textcolor[rgb]{0.00,0.50,0.00}{##1}}}
\expandafter\def\csname PY@tok@kp\endcsname{\def\PY@tc##1{\textcolor[rgb]{0.00,0.50,0.00}{##1}}}
\expandafter\def\csname PY@tok@kt\endcsname{\def\PY@tc##1{\textcolor[rgb]{0.69,0.00,0.25}{##1}}}
\expandafter\def\csname PY@tok@o\endcsname{\def\PY@tc##1{\textcolor[rgb]{0.40,0.40,0.40}{##1}}}
\expandafter\def\csname PY@tok@ow\endcsname{\let\PY@bf=\textbf\def\PY@tc##1{\textcolor[rgb]{0.67,0.13,1.00}{##1}}}
\expandafter\def\csname PY@tok@nb\endcsname{\def\PY@tc##1{\textcolor[rgb]{0.00,0.50,0.00}{##1}}}
\expandafter\def\csname PY@tok@nf\endcsname{\def\PY@tc##1{\textcolor[rgb]{0.00,0.00,1.00}{##1}}}
\expandafter\def\csname PY@tok@nc\endcsname{\let\PY@bf=\textbf\def\PY@tc##1{\textcolor[rgb]{0.00,0.00,1.00}{##1}}}
\expandafter\def\csname PY@tok@nn\endcsname{\let\PY@bf=\textbf\def\PY@tc##1{\textcolor[rgb]{0.00,0.00,1.00}{##1}}}
\expandafter\def\csname PY@tok@ne\endcsname{\let\PY@bf=\textbf\def\PY@tc##1{\textcolor[rgb]{0.82,0.25,0.23}{##1}}}
\expandafter\def\csname PY@tok@nv\endcsname{\def\PY@tc##1{\textcolor[rgb]{0.10,0.09,0.49}{##1}}}
\expandafter\def\csname PY@tok@no\endcsname{\def\PY@tc##1{\textcolor[rgb]{0.53,0.00,0.00}{##1}}}
\expandafter\def\csname PY@tok@nl\endcsname{\def\PY@tc##1{\textcolor[rgb]{0.63,0.63,0.00}{##1}}}
\expandafter\def\csname PY@tok@ni\endcsname{\let\PY@bf=\textbf\def\PY@tc##1{\textcolor[rgb]{0.60,0.60,0.60}{##1}}}
\expandafter\def\csname PY@tok@na\endcsname{\def\PY@tc##1{\textcolor[rgb]{0.49,0.56,0.16}{##1}}}
\expandafter\def\csname PY@tok@nt\endcsname{\let\PY@bf=\textbf\def\PY@tc##1{\textcolor[rgb]{0.00,0.50,0.00}{##1}}}
\expandafter\def\csname PY@tok@nd\endcsname{\def\PY@tc##1{\textcolor[rgb]{0.67,0.13,1.00}{##1}}}
\expandafter\def\csname PY@tok@s\endcsname{\def\PY@tc##1{\textcolor[rgb]{0.73,0.13,0.13}{##1}}}
\expandafter\def\csname PY@tok@sd\endcsname{\let\PY@it=\textit\def\PY@tc##1{\textcolor[rgb]{0.73,0.13,0.13}{##1}}}
\expandafter\def\csname PY@tok@si\endcsname{\let\PY@bf=\textbf\def\PY@tc##1{\textcolor[rgb]{0.73,0.40,0.53}{##1}}}
\expandafter\def\csname PY@tok@se\endcsname{\let\PY@bf=\textbf\def\PY@tc##1{\textcolor[rgb]{0.73,0.40,0.13}{##1}}}
\expandafter\def\csname PY@tok@sr\endcsname{\def\PY@tc##1{\textcolor[rgb]{0.73,0.40,0.53}{##1}}}
\expandafter\def\csname PY@tok@ss\endcsname{\def\PY@tc##1{\textcolor[rgb]{0.10,0.09,0.49}{##1}}}
\expandafter\def\csname PY@tok@sx\endcsname{\def\PY@tc##1{\textcolor[rgb]{0.00,0.50,0.00}{##1}}}
\expandafter\def\csname PY@tok@m\endcsname{\def\PY@tc##1{\textcolor[rgb]{0.40,0.40,0.40}{##1}}}
\expandafter\def\csname PY@tok@gh\endcsname{\let\PY@bf=\textbf\def\PY@tc##1{\textcolor[rgb]{0.00,0.00,0.50}{##1}}}
\expandafter\def\csname PY@tok@gu\endcsname{\let\PY@bf=\textbf\def\PY@tc##1{\textcolor[rgb]{0.50,0.00,0.50}{##1}}}
\expandafter\def\csname PY@tok@gd\endcsname{\def\PY@tc##1{\textcolor[rgb]{0.63,0.00,0.00}{##1}}}
\expandafter\def\csname PY@tok@gi\endcsname{\def\PY@tc##1{\textcolor[rgb]{0.00,0.63,0.00}{##1}}}
\expandafter\def\csname PY@tok@gr\endcsname{\def\PY@tc##1{\textcolor[rgb]{1.00,0.00,0.00}{##1}}}
\expandafter\def\csname PY@tok@ge\endcsname{\let\PY@it=\textit}
\expandafter\def\csname PY@tok@gs\endcsname{\let\PY@bf=\textbf}
\expandafter\def\csname PY@tok@gp\endcsname{\let\PY@bf=\textbf\def\PY@tc##1{\textcolor[rgb]{0.00,0.00,0.50}{##1}}}
\expandafter\def\csname PY@tok@go\endcsname{\def\PY@tc##1{\textcolor[rgb]{0.53,0.53,0.53}{##1}}}
\expandafter\def\csname PY@tok@gt\endcsname{\def\PY@tc##1{\textcolor[rgb]{0.00,0.27,0.87}{##1}}}
\expandafter\def\csname PY@tok@err\endcsname{\def\PY@bc##1{\setlength{\fboxsep}{0pt}\fcolorbox[rgb]{1.00,0.00,0.00}{1,1,1}{\strut ##1}}}
\expandafter\def\csname PY@tok@kc\endcsname{\let\PY@bf=\textbf\def\PY@tc##1{\textcolor[rgb]{0.00,0.50,0.00}{##1}}}
\expandafter\def\csname PY@tok@kd\endcsname{\let\PY@bf=\textbf\def\PY@tc##1{\textcolor[rgb]{0.00,0.50,0.00}{##1}}}
\expandafter\def\csname PY@tok@kn\endcsname{\let\PY@bf=\textbf\def\PY@tc##1{\textcolor[rgb]{0.00,0.50,0.00}{##1}}}
\expandafter\def\csname PY@tok@kr\endcsname{\let\PY@bf=\textbf\def\PY@tc##1{\textcolor[rgb]{0.00,0.50,0.00}{##1}}}
\expandafter\def\csname PY@tok@bp\endcsname{\def\PY@tc##1{\textcolor[rgb]{0.00,0.50,0.00}{##1}}}
\expandafter\def\csname PY@tok@fm\endcsname{\def\PY@tc##1{\textcolor[rgb]{0.00,0.00,1.00}{##1}}}
\expandafter\def\csname PY@tok@vc\endcsname{\def\PY@tc##1{\textcolor[rgb]{0.10,0.09,0.49}{##1}}}
\expandafter\def\csname PY@tok@vg\endcsname{\def\PY@tc##1{\textcolor[rgb]{0.10,0.09,0.49}{##1}}}
\expandafter\def\csname PY@tok@vi\endcsname{\def\PY@tc##1{\textcolor[rgb]{0.10,0.09,0.49}{##1}}}
\expandafter\def\csname PY@tok@vm\endcsname{\def\PY@tc##1{\textcolor[rgb]{0.10,0.09,0.49}{##1}}}
\expandafter\def\csname PY@tok@sa\endcsname{\def\PY@tc##1{\textcolor[rgb]{0.73,0.13,0.13}{##1}}}
\expandafter\def\csname PY@tok@sb\endcsname{\def\PY@tc##1{\textcolor[rgb]{0.73,0.13,0.13}{##1}}}
\expandafter\def\csname PY@tok@sc\endcsname{\def\PY@tc##1{\textcolor[rgb]{0.73,0.13,0.13}{##1}}}
\expandafter\def\csname PY@tok@dl\endcsname{\def\PY@tc##1{\textcolor[rgb]{0.73,0.13,0.13}{##1}}}
\expandafter\def\csname PY@tok@s2\endcsname{\def\PY@tc##1{\textcolor[rgb]{0.73,0.13,0.13}{##1}}}
\expandafter\def\csname PY@tok@sh\endcsname{\def\PY@tc##1{\textcolor[rgb]{0.73,0.13,0.13}{##1}}}
\expandafter\def\csname PY@tok@s1\endcsname{\def\PY@tc##1{\textcolor[rgb]{0.73,0.13,0.13}{##1}}}
\expandafter\def\csname PY@tok@mb\endcsname{\def\PY@tc##1{\textcolor[rgb]{0.40,0.40,0.40}{##1}}}
\expandafter\def\csname PY@tok@mf\endcsname{\def\PY@tc##1{\textcolor[rgb]{0.40,0.40,0.40}{##1}}}
\expandafter\def\csname PY@tok@mh\endcsname{\def\PY@tc##1{\textcolor[rgb]{0.40,0.40,0.40}{##1}}}
\expandafter\def\csname PY@tok@mi\endcsname{\def\PY@tc##1{\textcolor[rgb]{0.40,0.40,0.40}{##1}}}
\expandafter\def\csname PY@tok@il\endcsname{\def\PY@tc##1{\textcolor[rgb]{0.40,0.40,0.40}{##1}}}
\expandafter\def\csname PY@tok@mo\endcsname{\def\PY@tc##1{\textcolor[rgb]{0.40,0.40,0.40}{##1}}}
\expandafter\def\csname PY@tok@ch\endcsname{\let\PY@it=\textit\def\PY@tc##1{\textcolor[rgb]{0.25,0.50,0.50}{##1}}}
\expandafter\def\csname PY@tok@cm\endcsname{\let\PY@it=\textit\def\PY@tc##1{\textcolor[rgb]{0.25,0.50,0.50}{##1}}}
\expandafter\def\csname PY@tok@cpf\endcsname{\let\PY@it=\textit\def\PY@tc##1{\textcolor[rgb]{0.25,0.50,0.50}{##1}}}
\expandafter\def\csname PY@tok@c1\endcsname{\let\PY@it=\textit\def\PY@tc##1{\textcolor[rgb]{0.25,0.50,0.50}{##1}}}
\expandafter\def\csname PY@tok@cs\endcsname{\let\PY@it=\textit\def\PY@tc##1{\textcolor[rgb]{0.25,0.50,0.50}{##1}}}

\def\PYZbs{\char`\\}
\def\PYZus{\char`\_}
\def\PYZob{\char`\{}
\def\PYZcb{\char`\}}
\def\PYZca{\char`\^}
\def\PYZam{\char`\&}
\def\PYZlt{\char`\<}
\def\PYZgt{\char`\>}
\def\PYZsh{\char`\#}
\def\PYZpc{\char`\%}
\def\PYZdl{\char`\$}
\def\PYZhy{\char`\-}
\def\PYZsq{\char`\'}
\def\PYZdq{\char`\"}
\def\PYZti{\char`\~}
% for compatibility with earlier versions
\def\PYZat{@}
\def\PYZlb{[}
\def\PYZrb{]}
\makeatother


    % Exact colors from NB
    \definecolor{incolor}{rgb}{0.0, 0.0, 0.5}
    \definecolor{outcolor}{rgb}{0.545, 0.0, 0.0}



    
    % Prevent overflowing lines due to hard-to-break entities
    \sloppy 
    % Setup hyperref package
    \hypersetup{
      breaklinks=true,  % so long urls are correctly broken across lines
      colorlinks=true,
      urlcolor=urlcolor,
      linkcolor=linkcolor,
      citecolor=citecolor,
      }
    % Slightly bigger margins than the latex defaults
    
    \geometry{verbose,tmargin=1in,bmargin=1in,lmargin=1in,rmargin=1in}
    
    

    \begin{document}
    
    
    \noindent
\large\textbf{Homework Assignment 3} \hfill \textbf{Anirudh Ganesh} \\
\normalsize Computer Vision for HCI \hfill CSE5524 (Au `18) \\
Prof. Jim Davis \hfill Score: \_\_\_/18 \\
TA: Sayan Mandal \hfill Due Date: 09/11/18
    
    

    
    \hypertarget{imports}{%
\subsubsection{Imports}\label{imports}}

    \begin{Verbatim}[commandchars=\\\{\}]
{\color{incolor}In [{\color{incolor}1}]:} \PY{k+kn}{from} \PY{n+nn}{skimage}\PY{n+nn}{.}\PY{n+nn}{io} \PY{k}{import} \PY{n}{imread}
        \PY{k+kn}{from} \PY{n+nn}{skimage}\PY{n+nn}{.}\PY{n+nn}{filters} \PY{k}{import} \PY{n}{gaussian}
        \PY{k+kn}{import} \PY{n+nn}{numpy} \PY{k}{as} \PY{n+nn}{np}
        \PY{k+kn}{from} \PY{n+nn}{matplotlib} \PY{k}{import} \PY{n}{pyplot} \PY{k}{as} \PY{n}{plt}
        \PY{k+kn}{from} \PY{n+nn}{skimage} \PY{k}{import} \PY{n}{img\PYZus{}as\PYZus{}float}
        \PY{k+kn}{import} \PY{n+nn}{math}
\end{Verbatim}


    \begin{Verbatim}[commandchars=\\\{\}]
{\color{incolor}In [{\color{incolor}2}]:} \PY{k+kn}{from} \PY{n+nn}{skimage}\PY{n+nn}{.}\PY{n+nn}{color} \PY{k}{import} \PY{n}{rgb2gray}
        
        \PY{n}{mittens} \PY{o}{=} \PY{n}{imread}\PY{p}{(}\PY{l+s+s1}{\PYZsq{}}\PY{l+s+s1}{./data/murder\PYZus{}mittens.png}\PY{l+s+s1}{\PYZsq{}}\PY{p}{)}
        
        \PY{n}{gray\PYZus{}mitts} \PY{o}{=} \PY{n}{rgb2gray}\PY{p}{(}\PY{n}{mittens}\PY{p}{)}
\end{Verbatim}


    \hypertarget{generate-4-layer-gaussian-pyramid}{%
\section{Generate 4 Layer Gaussian
Pyramid}\label{generate-4-layer-gaussian-pyramid}}

    \hypertarget{helper-functions}{%
\subsection{Helper Functions}\label{helper-functions}}

Some more details on the helper functions, \texttt{smooth} and
\texttt{resize}.

\texttt{smooth} applies a simple Gaussian filter over the given image.
This is used to compute \$ G(m,n) = w(m)w(n) \$ where
\(w(m) = [ 0.25 - 0.5a, 0.25, a, 0.25, 0.25 - 0.5a ]\). In-order to
apply this `a' notice that we had implemented the gaussian filter as:

(snippet)

\begin{verbatim}
    lw = int(truncate * sd + 0.5)
    weights = [0.0] * (2 * lw + 1)
    weights[lw] = 1.0
    sum = 1.0
    sd = sd * sd
    # calculate the kernel:
    for ii in range(1, lw + 1):
        tmp = math.exp(-0.5 * float(ii * ii) / sd)
        weights[lw + ii] = tmp
        weights[lw - ii] = tmp
        sum += 2.0 * tmp
    for ii in range(2 * lw + 1):
        weights[ii] /= sum
\end{verbatim}

Thus, by using a \(\sigma\) of 0.5, we get
\texttt{int(4\ *\ 0.5\ +\ 0.5)\ =\ 2} which gets us
\texttt{weights\ =\ {[}0.0{]}\ *\ (2\ *\ 2\ +\ 1)} which then gets
filled as \texttt{weights\ =\ {[}0.05,\ 0.25,\ 0.4,\ 0.25,\ 0.05{]}}
which is our intended \(w(m)\). This is applied seperately row and
column-wise using \texttt{correlated1d}.

\texttt{resize} applies a simple resizing operation that helps us sample
half the bits in case scaling factor is less than 1, else it applies a
Affine Transform to double the size and fill the alternating rows with
0. This is why we use \texttt{tform.params}.

\begin{verbatim}
    tform.params[0, 1] = 0 # This tells us to fill alternated column with 0
    tform.params[1, 0] = 0 # This tells us to fill alternated row with 0
\end{verbatim}

Thus a \texttt{(2,2)} image when stretched to \texttt{(4,4)} will look
like:

\begin{verbatim}
            | a | 0 | b | 0 |
            ----------------
            | 0 | 0 | 0 | 0 |
 |a|b|      ----------------
 -----  --> | a | 0 | b | 0 |
 |a|b|      ----------------
            | 0 | 0 | 0 | 0 |
\end{verbatim}

    \begin{Verbatim}[commandchars=\\\{\}]
{\color{incolor}In [{\color{incolor}3}]:} \PY{k+kn}{from} \PY{n+nn}{scipy} \PY{k}{import} \PY{n}{ndimage} \PY{k}{as} \PY{n}{ndi}
        \PY{k+kn}{from} \PY{n+nn}{skimage}\PY{n+nn}{.}\PY{n+nn}{transform}\PY{n+nn}{.}\PY{n+nn}{\PYZus{}geometric} \PY{k}{import} \PY{n}{AffineTransform}
        \PY{k+kn}{from} \PY{n+nn}{skimage}\PY{n+nn}{.}\PY{n+nn}{transform}\PY{n+nn}{.}\PY{n+nn}{\PYZus{}warps} \PY{k}{import} \PY{n}{warp}
        
        \PY{k}{def} \PY{n+nf}{smooth}\PY{p}{(}\PY{n}{image}\PY{p}{,} \PY{n}{sigma}\PY{p}{)}\PY{p}{:}
            \PY{n}{smoothed} \PY{o}{=} \PY{n}{np}\PY{o}{.}\PY{n}{empty}\PY{p}{(}\PY{n}{image}\PY{o}{.}\PY{n}{shape}\PY{p}{,} \PY{n}{dtype}\PY{o}{=}\PY{n}{np}\PY{o}{.}\PY{n}{double}\PY{p}{)}
            \PY{n}{ndi}\PY{o}{.}\PY{n}{gaussian\PYZus{}filter}\PY{p}{(}\PY{n}{image}\PY{p}{,} \PY{n}{sigma}\PY{p}{,} \PY{n}{output}\PY{o}{=}\PY{n}{smoothed}\PY{p}{)}
            \PY{k}{return} \PY{n}{smoothed}
        
        
        \PY{k}{def} \PY{n+nf}{resize}\PY{p}{(}\PY{n}{image}\PY{p}{,} \PY{n}{output\PYZus{}shape}\PY{p}{)}\PY{p}{:}
            \PY{n}{output\PYZus{}shape} \PY{o}{=} \PY{n+nb}{tuple}\PY{p}{(}\PY{n}{output\PYZus{}shape}\PY{p}{)}
            \PY{n}{input\PYZus{}shape} \PY{o}{=} \PY{n}{image}\PY{o}{.}\PY{n}{shape}
        
            \PY{n}{factors} \PY{o}{=} \PY{p}{(}\PY{n}{np}\PY{o}{.}\PY{n}{asarray}\PY{p}{(}\PY{n}{input\PYZus{}shape}\PY{p}{,} \PY{n}{dtype}\PY{o}{=}\PY{n+nb}{float}\PY{p}{)} \PY{o}{/}
                       \PY{n}{np}\PY{o}{.}\PY{n}{asarray}\PY{p}{(}\PY{n}{output\PYZus{}shape}\PY{p}{,} \PY{n}{dtype}\PY{o}{=}\PY{n+nb}{float}\PY{p}{)}\PY{p}{)}
        
            \PY{n}{rows} \PY{o}{=} \PY{n}{output\PYZus{}shape}\PY{p}{[}\PY{l+m+mi}{0}\PY{p}{]}
            \PY{n}{cols} \PY{o}{=} \PY{n}{output\PYZus{}shape}\PY{p}{[}\PY{l+m+mi}{1}\PY{p}{]}
            \PY{n}{input\PYZus{}rows} \PY{o}{=} \PY{n}{input\PYZus{}shape}\PY{p}{[}\PY{l+m+mi}{0}\PY{p}{]}
            \PY{n}{input\PYZus{}cols} \PY{o}{=} \PY{n}{input\PYZus{}shape}\PY{p}{[}\PY{l+m+mi}{1}\PY{p}{]}
            \PY{k}{if} \PY{n}{rows} \PY{o}{==} \PY{l+m+mi}{1} \PY{o+ow}{and} \PY{n}{cols} \PY{o}{==} \PY{l+m+mi}{1}\PY{p}{:}
                \PY{n}{tform} \PY{o}{=} \PY{n}{AffineTransform}\PY{p}{(}\PY{n}{translation}\PY{o}{=}\PY{p}{(}\PY{n}{input\PYZus{}cols} \PY{o}{/} \PY{l+m+mf}{2.0} \PY{o}{\PYZhy{}} \PY{l+m+mf}{0.5}\PY{p}{,}
                                                     \PY{n}{input\PYZus{}rows} \PY{o}{/} \PY{l+m+mf}{2.0} \PY{o}{\PYZhy{}} \PY{l+m+mf}{0.5}\PY{p}{)}\PY{p}{)}
            \PY{k}{else}\PY{p}{:}
                
                
                \PY{c+c1}{\PYZsh{} 3 control points necessary to estimate exact AffineTransform}
                \PY{n}{src\PYZus{}corners} \PY{o}{=} \PY{n}{np}\PY{o}{.}\PY{n}{array}\PY{p}{(}\PY{p}{[}\PY{p}{[}\PY{l+m+mi}{1}\PY{p}{,} \PY{l+m+mi}{1}\PY{p}{]}\PY{p}{,} \PY{p}{[}\PY{l+m+mi}{1}\PY{p}{,} \PY{n}{rows}\PY{p}{]}\PY{p}{,} \PY{p}{[}\PY{n}{cols}\PY{p}{,} \PY{n}{rows}\PY{p}{]}\PY{p}{]}\PY{p}{)} \PY{o}{\PYZhy{}} \PY{l+m+mi}{1}
                \PY{n}{dst\PYZus{}corners} \PY{o}{=} \PY{n}{np}\PY{o}{.}\PY{n}{zeros}\PY{p}{(}\PY{n}{src\PYZus{}corners}\PY{o}{.}\PY{n}{shape}\PY{p}{,} \PY{n}{dtype}\PY{o}{=}\PY{n}{np}\PY{o}{.}\PY{n}{double}\PY{p}{)}
                
                
                \PY{c+c1}{\PYZsh{} take into account that 0th pixel is at position (0.5, 0.5)}
                \PY{n}{dst\PYZus{}corners}\PY{p}{[}\PY{p}{:}\PY{p}{,} \PY{l+m+mi}{0}\PY{p}{]} \PY{o}{=} \PY{n}{factors}\PY{p}{[}\PY{l+m+mi}{1}\PY{p}{]} \PY{o}{*} \PY{p}{(}\PY{n}{src\PYZus{}corners}\PY{p}{[}\PY{p}{:}\PY{p}{,} \PY{l+m+mi}{0}\PY{p}{]} \PY{o}{+} \PY{l+m+mf}{0.5}\PY{p}{)} \PY{o}{\PYZhy{}} \PY{l+m+mf}{0.5}
                \PY{n}{dst\PYZus{}corners}\PY{p}{[}\PY{p}{:}\PY{p}{,} \PY{l+m+mi}{1}\PY{p}{]} \PY{o}{=} \PY{n}{factors}\PY{p}{[}\PY{l+m+mi}{0}\PY{p}{]} \PY{o}{*} \PY{p}{(}\PY{n}{src\PYZus{}corners}\PY{p}{[}\PY{p}{:}\PY{p}{,} \PY{l+m+mi}{1}\PY{p}{]} \PY{o}{+} \PY{l+m+mf}{0.5}\PY{p}{)} \PY{o}{\PYZhy{}} \PY{l+m+mf}{0.5}
        
                \PY{n}{tform} \PY{o}{=} \PY{n}{AffineTransform}\PY{p}{(}\PY{p}{)}
                \PY{n}{tform}\PY{o}{.}\PY{n}{estimate}\PY{p}{(}\PY{n}{src\PYZus{}corners}\PY{p}{,} \PY{n}{dst\PYZus{}corners}\PY{p}{)}
        
            \PY{c+c1}{\PYZsh{} Parameters for accurate resizing}
            \PY{n}{tform}\PY{o}{.}\PY{n}{params}\PY{p}{[}\PY{l+m+mi}{2}\PY{p}{]} \PY{o}{=} \PY{p}{(}\PY{l+m+mi}{0}\PY{p}{,} \PY{l+m+mi}{0}\PY{p}{,} \PY{l+m+mi}{1}\PY{p}{)}
            \PY{n}{tform}\PY{o}{.}\PY{n}{params}\PY{p}{[}\PY{l+m+mi}{0}\PY{p}{,} \PY{l+m+mi}{1}\PY{p}{]} \PY{o}{=} \PY{l+m+mi}{0}
            \PY{n}{tform}\PY{o}{.}\PY{n}{params}\PY{p}{[}\PY{l+m+mi}{1}\PY{p}{,} \PY{l+m+mi}{0}\PY{p}{]} \PY{o}{=} \PY{l+m+mi}{0}
        
            \PY{n}{out} \PY{o}{=} \PY{n}{warp}\PY{p}{(}\PY{n}{image}\PY{p}{,} \PY{n}{tform}\PY{p}{,} \PY{n}{output\PYZus{}shape}\PY{o}{=}\PY{n}{output\PYZus{}shape}\PY{p}{)}
        
            \PY{k}{return} \PY{n}{out}
\end{Verbatim}


    \hypertarget{pyramid-down-function}{%
\subsection{Pyramid Down function}\label{pyramid-down-function}}

This generates a downscaled image of the given image, which is also
smoothed out.

    \begin{Verbatim}[commandchars=\\\{\}]
{\color{incolor}In [{\color{incolor}4}]:} \PY{k}{def} \PY{n+nf}{pyramid\PYZus{}down}\PY{p}{(}\PY{n}{image}\PY{p}{,} \PY{n}{downscale}\PY{o}{=}\PY{l+m+mi}{2}\PY{p}{)}\PY{p}{:}
            \PY{n}{image} \PY{o}{=} \PY{n}{img\PYZus{}as\PYZus{}float}\PY{p}{(}\PY{n}{image}\PY{p}{)}
        
            \PY{n}{out\PYZus{}shape} \PY{o}{=} \PY{n+nb}{tuple}\PY{p}{(}\PY{p}{[}\PY{n}{math}\PY{o}{.}\PY{n}{ceil}\PY{p}{(}\PY{n}{d} \PY{o}{/} \PY{n+nb}{float}\PY{p}{(}\PY{n}{downscale}\PY{p}{)}\PY{p}{)} \PY{k}{for} \PY{n}{d} \PY{o+ow}{in} \PY{n}{image}\PY{o}{.}\PY{n}{shape}\PY{p}{]}\PY{p}{)}
        
            \PY{n}{sigma} \PY{o}{=} \PY{l+m+mf}{0.5}
        
            \PY{n}{smoothed} \PY{o}{=} \PY{n}{smooth}\PY{p}{(}\PY{n}{image}\PY{p}{,} \PY{n}{sigma}\PY{p}{)}
            \PY{n}{out} \PY{o}{=} \PY{n}{resize}\PY{p}{(}\PY{n}{smoothed}\PY{p}{,} \PY{n}{out\PYZus{}shape}\PY{p}{)}
        
            \PY{k}{return} \PY{n}{out}
\end{Verbatim}


    \hypertarget{gaussian-pyramid}{%
\subsection{Gaussian Pyramid}\label{gaussian-pyramid}}

Generates the Gaussian Pyramid of a given image

    \begin{Verbatim}[commandchars=\\\{\}]
{\color{incolor}In [{\color{incolor}5}]:} \PY{k}{def} \PY{n+nf}{gaussian\PYZus{}pyramid}\PY{p}{(}\PY{n}{image}\PY{p}{,} \PY{n}{max\PYZus{}layer}\PY{o}{=}\PY{o}{\PYZhy{}}\PY{l+m+mi}{1}\PY{p}{,} \PY{n}{downscale}\PY{o}{=}\PY{l+m+mi}{2}\PY{p}{)}\PY{p}{:}
            \PY{n}{image} \PY{o}{=} \PY{n}{img\PYZus{}as\PYZus{}float}\PY{p}{(}\PY{n}{image}\PY{p}{)}
        
            \PY{n}{layer} \PY{o}{=} \PY{l+m+mi}{0}
            \PY{n}{current\PYZus{}shape} \PY{o}{=} \PY{n}{image}\PY{o}{.}\PY{n}{shape}
        
            \PY{n}{prev\PYZus{}layer\PYZus{}image} \PY{o}{=} \PY{n}{image}
            \PY{k}{yield} \PY{n}{image}
        
            \PY{c+c1}{\PYZsh{} build downsampled images until max\PYZus{}layer is reached or downscale process}
            \PY{c+c1}{\PYZsh{} does not change image size}
            \PY{k}{while} \PY{n}{layer} \PY{o}{!=} \PY{n}{max\PYZus{}layer}\PY{p}{:}
                \PY{n}{layer} \PY{o}{+}\PY{o}{=} \PY{l+m+mi}{1}
        
                \PY{n}{layer\PYZus{}image} \PY{o}{=} \PY{n}{pyramid\PYZus{}down}\PY{p}{(}\PY{n}{prev\PYZus{}layer\PYZus{}image}\PY{p}{,} \PY{n}{downscale}\PY{p}{)}
        
                \PY{n}{prev\PYZus{}shape} \PY{o}{=} \PY{n}{np}\PY{o}{.}\PY{n}{asarray}\PY{p}{(}\PY{n}{current\PYZus{}shape}\PY{p}{)}
                \PY{n}{prev\PYZus{}layer\PYZus{}image} \PY{o}{=} \PY{n}{layer\PYZus{}image}
                \PY{n}{current\PYZus{}shape} \PY{o}{=} \PY{n}{np}\PY{o}{.}\PY{n}{asarray}\PY{p}{(}\PY{n}{layer\PYZus{}image}\PY{o}{.}\PY{n}{shape}\PY{p}{)}
        
                \PY{c+c1}{\PYZsh{} no change to previous pyramid layer}
                \PY{k}{if} \PY{n}{np}\PY{o}{.}\PY{n}{all}\PY{p}{(}\PY{n}{current\PYZus{}shape} \PY{o}{==} \PY{n}{prev\PYZus{}shape}\PY{p}{)}\PY{p}{:}
                    \PY{k}{break}
        
                \PY{k}{yield} \PY{n}{layer\PYZus{}image}
\end{Verbatim}


    \hypertarget{laplacian-pyramid}{%
\subsection{Laplacian Pyramid}\label{laplacian-pyramid}}

Generates the Laplacian Pyramid of the given image

    \begin{Verbatim}[commandchars=\\\{\}]
{\color{incolor}In [{\color{incolor}6}]:} \PY{k}{def} \PY{n+nf}{laplacian\PYZus{}pyramid}\PY{p}{(}\PY{n}{image}\PY{p}{,} \PY{n}{max\PYZus{}layer}\PY{o}{=}\PY{o}{\PYZhy{}}\PY{l+m+mi}{1}\PY{p}{,} \PY{n}{downscale}\PY{o}{=}\PY{l+m+mi}{2}\PY{p}{)}\PY{p}{:}
            \PY{c+c1}{\PYZsh{} cast to float for consistent data type in pyramid}
            \PY{n}{image} \PY{o}{=} \PY{n}{img\PYZus{}as\PYZus{}float}\PY{p}{(}\PY{n}{image}\PY{p}{)}
        
            \PY{n}{sigma} \PY{o}{=} \PY{l+m+mf}{0.5}
        
            \PY{n}{current\PYZus{}shape} \PY{o}{=} \PY{n}{image}\PY{o}{.}\PY{n}{shape}
        
            \PY{n}{smoothed\PYZus{}image} \PY{o}{=} \PY{n}{smooth}\PY{p}{(}\PY{n}{image}\PY{p}{,} \PY{n}{sigma}\PY{p}{)}
            
            \PY{c+c1}{\PYZsh{} Generate the first layer}
            \PY{k}{yield} \PY{n}{image} \PY{o}{\PYZhy{}} \PY{n}{smoothed\PYZus{}image}
        
            \PY{c+c1}{\PYZsh{} build downsampled images until max\PYZus{}layer is reached or downscale process}
            \PY{c+c1}{\PYZsh{} does not change image size}
            \PY{k}{if} \PY{n}{max\PYZus{}layer} \PY{o}{==} \PY{o}{\PYZhy{}}\PY{l+m+mi}{1}\PY{p}{:}
                \PY{n}{max\PYZus{}layer} \PY{o}{=} \PY{n+nb}{int}\PY{p}{(}\PY{n}{np}\PY{o}{.}\PY{n}{ceil}\PY{p}{(}\PY{n}{math}\PY{o}{.}\PY{n}{log}\PY{p}{(}\PY{n}{np}\PY{o}{.}\PY{n}{max}\PY{p}{(}\PY{n}{current\PYZus{}shape}\PY{p}{)}\PY{p}{,} \PY{n}{downscale}\PY{p}{)}\PY{p}{)}\PY{p}{)}
        
            \PY{k}{for} \PY{n}{layer} \PY{o+ow}{in} \PY{n+nb}{range}\PY{p}{(}\PY{n}{max\PYZus{}layer}\PY{p}{)}\PY{p}{:}
        
                \PY{n}{out\PYZus{}shape} \PY{o}{=} \PY{n+nb}{tuple}\PY{p}{(}
                    \PY{p}{[}\PY{n}{math}\PY{o}{.}\PY{n}{ceil}\PY{p}{(}\PY{n}{d} \PY{o}{/} \PY{n+nb}{float}\PY{p}{(}\PY{n}{downscale}\PY{p}{)}\PY{p}{)} \PY{k}{for} \PY{n}{d} \PY{o+ow}{in} \PY{n}{current\PYZus{}shape}\PY{p}{]}\PY{p}{)}
        
                \PY{n}{resized\PYZus{}image} \PY{o}{=} \PY{n}{resize}\PY{p}{(}\PY{n}{smoothed\PYZus{}image}\PY{p}{,} \PY{n}{out\PYZus{}shape}\PY{p}{)}
                \PY{n}{smoothed\PYZus{}image} \PY{o}{=} \PY{n}{smooth}\PY{p}{(}\PY{n}{resized\PYZus{}image}\PY{p}{,} \PY{n}{sigma}\PY{p}{)}
                \PY{n}{current\PYZus{}shape} \PY{o}{=} \PY{n}{np}\PY{o}{.}\PY{n}{asarray}\PY{p}{(}\PY{n}{resized\PYZus{}image}\PY{o}{.}\PY{n}{shape}\PY{p}{)}
        
                \PY{k}{yield} \PY{n}{resized\PYZus{}image} \PY{o}{\PYZhy{}} \PY{n}{smoothed\PYZus{}image}
\end{Verbatim}


    \begin{Verbatim}[commandchars=\\\{\}]
{\color{incolor}In [{\color{incolor}7}]:} \PY{n}{pyramid} \PY{o}{=} \PY{n+nb}{tuple}\PY{p}{(}\PY{n}{gaussian\PYZus{}pyramid}\PY{p}{(}\PY{n}{gray\PYZus{}mitts}\PY{p}{,} \PY{n}{max\PYZus{}layer}\PY{o}{=}\PY{l+m+mi}{3}\PY{p}{)}\PY{p}{)}
        
        \PY{n}{fig}\PY{p}{,} \PY{n}{axarr} \PY{o}{=} \PY{n}{plt}\PY{o}{.}\PY{n}{subplots}\PY{p}{(}\PY{l+m+mi}{2}\PY{p}{,} \PY{l+m+mi}{2}\PY{p}{,} \PY{n}{dpi}\PY{o}{=}\PY{l+m+mi}{200}\PY{p}{)}
        
        \PY{k}{for} \PY{n}{i} \PY{o+ow}{in} \PY{n+nb}{range}\PY{p}{(}\PY{l+m+mi}{4}\PY{p}{)}\PY{p}{:}
            \PY{n}{axarr}\PY{p}{[}\PY{n}{i}\PY{o}{/}\PY{o}{/}\PY{l+m+mi}{2}\PY{p}{,} \PY{n}{i}\PY{o}{\PYZpc{}}\PY{k}{2}].axis(\PYZsq{}off\PYZsq{})
            \PY{n}{axarr}\PY{p}{[}\PY{n}{i}\PY{o}{/}\PY{o}{/}\PY{l+m+mi}{2}\PY{p}{,} \PY{n}{i}\PY{o}{\PYZpc{}}\PY{k}{2}].imshow(pyramid[i], cmap=\PYZsq{}gray\PYZsq{})
\end{Verbatim}


    \begin{center}
    \adjustimage{max size={0.9\linewidth}{0.9\paperheight}}{output_12_0.png}
    \end{center}
    { \hspace*{\fill} \\}
    
    \begin{Verbatim}[commandchars=\\\{\}]
{\color{incolor}In [{\color{incolor}8}]:} \PY{n}{pyramid2} \PY{o}{=} \PY{n+nb}{tuple}\PY{p}{(}\PY{n}{laplacian\PYZus{}pyramid}\PY{p}{(}\PY{n}{gray\PYZus{}mitts}\PY{p}{,} \PY{n}{max\PYZus{}layer}\PY{o}{=}\PY{l+m+mi}{3}\PY{p}{)}\PY{p}{)}
        
        \PY{n}{fig}\PY{p}{,} \PY{n}{axarr} \PY{o}{=} \PY{n}{plt}\PY{o}{.}\PY{n}{subplots}\PY{p}{(}\PY{l+m+mi}{2}\PY{p}{,} \PY{l+m+mi}{2}\PY{p}{,} \PY{n}{dpi}\PY{o}{=}\PY{l+m+mi}{200}\PY{p}{)}
        
        \PY{k}{for} \PY{n}{i} \PY{o+ow}{in} \PY{n+nb}{range}\PY{p}{(}\PY{l+m+mi}{4}\PY{p}{)}\PY{p}{:}
            \PY{n}{axarr}\PY{p}{[}\PY{n}{i}\PY{o}{/}\PY{o}{/}\PY{l+m+mi}{2}\PY{p}{,} \PY{n}{i}\PY{o}{\PYZpc{}}\PY{k}{2}].axis(\PYZsq{}off\PYZsq{})
            \PY{n}{axarr}\PY{p}{[}\PY{n}{i}\PY{o}{/}\PY{o}{/}\PY{l+m+mi}{2}\PY{p}{,} \PY{n}{i}\PY{o}{\PYZpc{}}\PY{k}{2}].imshow(pyramid2[i], cmap=\PYZsq{}gray\PYZsq{})
\end{Verbatim}


    \begin{center}
    \adjustimage{max size={0.9\linewidth}{0.9\paperheight}}{output_13_0.png}
    \end{center}
    { \hspace*{\fill} \\}
    
    \hypertarget{background-subtraction-single-image}{%
\section{Background Subtraction (single
image)}\label{background-subtraction-single-image}}

    \begin{Verbatim}[commandchars=\\\{\}]
{\color{incolor}In [{\color{incolor}9}]:} \PY{k+kn}{from} \PY{n+nn}{skimage} \PY{k}{import} \PY{n}{img\PYZus{}as\PYZus{}float}
        \PY{n}{background} \PY{o}{=} \PY{n}{imread}\PY{p}{(}\PY{l+s+s1}{\PYZsq{}}\PY{l+s+s1}{./data/bg000.bmp}\PY{l+s+s1}{\PYZsq{}}\PY{p}{)}
        \PY{n}{woman} \PY{o}{=} \PY{n}{imread}\PY{p}{(}\PY{l+s+s1}{\PYZsq{}}\PY{l+s+s1}{./data/walk.bmp}\PY{l+s+s1}{\PYZsq{}}\PY{p}{)}
\end{Verbatim}


    \begin{Verbatim}[commandchars=\\\{\}]
{\color{incolor}In [{\color{incolor}10}]:} \PY{n}{background} \PY{o}{=} \PY{n}{img\PYZus{}as\PYZus{}float}\PY{p}{(}\PY{n}{background}\PY{p}{)}
         \PY{n}{woman} \PY{o}{=} \PY{n}{img\PYZus{}as\PYZus{}float}\PY{p}{(}\PY{n}{woman}\PY{p}{)}
         
         \PY{n}{diff} \PY{o}{=} \PY{n}{np}\PY{o}{.}\PY{n}{abs}\PY{p}{(}\PY{n}{np}\PY{o}{.}\PY{n}{subtract}\PY{p}{(}\PY{n}{woman}\PY{p}{,} \PY{n}{background}\PY{p}{)}\PY{p}{)}
         
         \PY{n}{threshLevels} \PY{o}{=} \PY{p}{[}\PY{l+m+mf}{0.15}\PY{p}{,} \PY{l+m+mf}{0.25}\PY{p}{,} \PY{l+m+mf}{0.3}\PY{p}{,} \PY{l+m+mf}{0.50}\PY{p}{,} \PY{l+m+mf}{0.65}\PY{p}{,} \PY{l+m+mf}{0.75}\PY{p}{]}
         \PY{n}{f}\PY{p}{,} \PY{n}{axarr} \PY{o}{=} \PY{n}{plt}\PY{o}{.}\PY{n}{subplots}\PY{p}{(}\PY{l+m+mi}{2}\PY{p}{,} \PY{l+m+mi}{3}\PY{p}{,} \PY{n}{sharex}\PY{o}{=}\PY{l+s+s1}{\PYZsq{}}\PY{l+s+s1}{col}\PY{l+s+s1}{\PYZsq{}}\PY{p}{,} \PY{n}{sharey}\PY{o}{=}\PY{l+s+s1}{\PYZsq{}}\PY{l+s+s1}{row}\PY{l+s+s1}{\PYZsq{}}\PY{p}{,} \PY{n}{dpi}\PY{o}{=}\PY{l+m+mi}{200}\PY{p}{)}
         \PY{k}{for} \PY{n}{thresh} \PY{o+ow}{in} \PY{n}{threshLevels}\PY{p}{:}
             \PY{n}{tImg} \PY{o}{=} \PY{n}{diff} \PY{o}{\PYZgt{}} \PY{n}{thresh}
             \PY{n}{idx} \PY{o}{=} \PY{n}{threshLevels}\PY{o}{.}\PY{n}{index}\PY{p}{(}\PY{n}{thresh}\PY{p}{)}
             \PY{c+c1}{\PYZsh{}print((int(idx/3), idx\PYZpc{}3))}
             \PY{n}{axarr}\PY{p}{[}\PY{n}{idx} \PY{o}{/}\PY{o}{/} \PY{l+m+mi}{3}\PY{p}{,} \PY{n}{idx}\PY{o}{\PYZpc{}}\PY{k}{3}].axis(\PYZsq{}off\PYZsq{})
             \PY{n}{axarr}\PY{p}{[}\PY{n}{idx} \PY{o}{/}\PY{o}{/} \PY{l+m+mi}{3}\PY{p}{,} \PY{n}{idx}\PY{o}{\PYZpc{}}\PY{k}{3}].set\PYZus{}title(f\PYZsq{}Value = \PYZob{}thresh\PYZcb{}\PYZsq{})
             \PY{n}{axarr}\PY{p}{[}\PY{n}{idx} \PY{o}{/}\PY{o}{/} \PY{l+m+mi}{3}\PY{p}{,} \PY{n}{idx}\PY{o}{\PYZpc{}}\PY{k}{3}].imshow(tImg, cmap = \PYZsq{}gray\PYZsq{}, aspect=\PYZsq{}auto\PYZsq{})
\end{Verbatim}


    \begin{center}
    \adjustimage{max size={0.9\linewidth}{0.9\paperheight}}{output_16_0.png}
    \end{center}
    { \hspace*{\fill} \\}
    
    From the above diagram, we can clearly observe that for threshold values
\texttt{0.3\ and\ 0.25} our naive subtraction does a pretty good job of
identifying the woman from the background. However, notice that higher
values like \texttt{0.75} and \texttt{0.15} do a bad job. Not only does
the lower threshold introduce specs of activation all over the image,
but also creates the overall body a lot ``grainy'', and higher threshold
on the other hand are only able to capture small bits and pieces. If we
were to post process these results with morphological operator, it may
cause the person to dissappear completely, or the bounding box generated
would be pretty bad.

    \hypertarget{background-subtraction-multiple-images}{%
\section{Background Subtraction (multiple
images)}\label{background-subtraction-multiple-images}}

    \begin{Verbatim}[commandchars=\\\{\}]
{\color{incolor}In [{\color{incolor}11}]:} \PY{n}{background\PYZus{}cube} \PY{o}{=} \PY{p}{[}\PY{p}{]}
         
         \PY{k}{for} \PY{n}{i} \PY{o+ow}{in} \PY{n+nb}{range}\PY{p}{(}\PY{l+m+mi}{30}\PY{p}{)}\PY{p}{:}
             \PY{n}{tmp} \PY{o}{=} \PY{n}{imread}\PY{p}{(}\PY{l+s+s1}{\PYZsq{}}\PY{l+s+s1}{./data/bg}\PY{l+s+si}{\PYZob{}:03\PYZcb{}}\PY{l+s+s1}{.bmp}\PY{l+s+s1}{\PYZsq{}}\PY{o}{.}\PY{n}{format}\PY{p}{(}\PY{n}{i}\PY{p}{)}\PY{p}{)}
             \PY{n}{tmp} \PY{o}{=} \PY{n}{img\PYZus{}as\PYZus{}float}\PY{p}{(}\PY{n}{tmp}\PY{p}{)}
             \PY{n}{background\PYZus{}cube}\PY{o}{.}\PY{n}{append}\PY{p}{(}\PY{n}{tmp}\PY{p}{)}
\end{Verbatim}


    \begin{Verbatim}[commandchars=\\\{\}]
{\color{incolor}In [{\color{incolor}12}]:} \PY{n}{background\PYZus{}cube} \PY{o}{=} \PY{n}{np}\PY{o}{.}\PY{n}{array}\PY{p}{(}\PY{n}{background\PYZus{}cube}\PY{p}{)}
         
         \PY{n}{background\PYZus{}mean} \PY{o}{=} \PY{n}{np}\PY{o}{.}\PY{n}{mean}\PY{p}{(}\PY{n}{background\PYZus{}cube}\PY{p}{,} \PY{n}{axis}\PY{o}{=}\PY{l+m+mi}{0}\PY{p}{)}
         
         \PY{n}{background\PYZus{}std} \PY{o}{=} \PY{n}{np}\PY{o}{.}\PY{n}{std}\PY{p}{(}\PY{n}{background\PYZus{}cube}\PY{p}{,} \PY{n}{axis}\PY{o}{=}\PY{l+m+mi}{0}\PY{p}{)} \PY{o}{+} \PY{l+m+mf}{0.1}
         
         \PY{n}{diff\PYZus{}mahalanobis} \PY{o}{=} \PY{n}{np}\PY{o}{.}\PY{n}{divide}\PY{p}{(}\PY{n}{np}\PY{o}{.}\PY{n}{square}\PY{p}{(}\PY{n}{np}\PY{o}{.}\PY{n}{subtract}\PY{p}{(}\PY{n}{woman}\PY{p}{,} \PY{n}{background\PYZus{}mean}\PY{p}{)}\PY{p}{)}\PY{p}{,} \PY{n}{np}\PY{o}{.}\PY{n}{square}\PY{p}{(}\PY{n}{background\PYZus{}std}\PY{p}{)}\PY{p}{)}
         
         \PY{n}{threshLevels} \PY{o}{=} \PY{p}{[}\PY{l+m+mi}{2}\PY{p}{,} \PY{l+m+mi}{4}\PY{p}{,} \PY{l+m+mi}{8}\PY{p}{,} \PY{l+m+mi}{16}\PY{p}{,} \PY{l+m+mi}{24}\PY{p}{,} \PY{l+m+mi}{32}\PY{p}{]}
         \PY{n}{f}\PY{p}{,} \PY{n}{axarr} \PY{o}{=} \PY{n}{plt}\PY{o}{.}\PY{n}{subplots}\PY{p}{(}\PY{l+m+mi}{2}\PY{p}{,} \PY{l+m+mi}{3}\PY{p}{,} \PY{n}{sharex}\PY{o}{=}\PY{l+s+s1}{\PYZsq{}}\PY{l+s+s1}{col}\PY{l+s+s1}{\PYZsq{}}\PY{p}{,} \PY{n}{sharey}\PY{o}{=}\PY{l+s+s1}{\PYZsq{}}\PY{l+s+s1}{row}\PY{l+s+s1}{\PYZsq{}}\PY{p}{,} \PY{n}{dpi}\PY{o}{=}\PY{l+m+mi}{200}\PY{p}{)}
         \PY{k}{for} \PY{n}{thresh} \PY{o+ow}{in} \PY{n}{threshLevels}\PY{p}{:}
             \PY{n}{tImg} \PY{o}{=} \PY{n}{diff\PYZus{}mahalanobis} \PY{o}{\PYZgt{}} \PY{n}{thresh}
             \PY{n}{idx} \PY{o}{=} \PY{n}{threshLevels}\PY{o}{.}\PY{n}{index}\PY{p}{(}\PY{n}{thresh}\PY{p}{)}
             \PY{c+c1}{\PYZsh{}print((int(idx/3), idx\PYZpc{}3))}
             \PY{n}{axarr}\PY{p}{[}\PY{n}{idx} \PY{o}{/}\PY{o}{/} \PY{l+m+mi}{3}\PY{p}{,} \PY{n}{idx}\PY{o}{\PYZpc{}}\PY{k}{3}].axis(\PYZsq{}off\PYZsq{})
             \PY{n}{axarr}\PY{p}{[}\PY{n}{idx} \PY{o}{/}\PY{o}{/} \PY{l+m+mi}{3}\PY{p}{,} \PY{n}{idx}\PY{o}{\PYZpc{}}\PY{k}{3}].set\PYZus{}title(f\PYZsq{}Value = \PYZob{}thresh\PYZcb{}\PYZsq{})
             \PY{n}{axarr}\PY{p}{[}\PY{n}{idx} \PY{o}{/}\PY{o}{/} \PY{l+m+mi}{3}\PY{p}{,} \PY{n}{idx}\PY{o}{\PYZpc{}}\PY{k}{3}].imshow(tImg, cmap = \PYZsq{}gray\PYZsq{}, aspect=\PYZsq{}auto\PYZsq{})
             
         \PY{n}{thresh} \PY{o}{=} \PY{n}{diff\PYZus{}mahalanobis} \PY{o}{\PYZgt{}} \PY{l+m+mi}{4}
\end{Verbatim}


    \begin{center}
    \adjustimage{max size={0.9\linewidth}{0.9\paperheight}}{output_20_0.png}
    \end{center}
    { \hspace*{\fill} \\}
    
    With our statistical model built upon a better understanding of what it
looks like to be part of background, we notice that the person is
detected pretty accurately. Also notice that this model is more tolerant
to increased thresholds. Even at \texttt{T\ =\ 32} which covers more
than 5 times the standard deviations, a level of significance so low
that its barely used in most statistical analysis, we see that our
thresholding holds up the upper body of the person relatively the same.
This robustness I believe is completely worth the (minor) computation
overhead that we incur.

    \hypertarget{dilate-image}{%
\section{Dilate image}\label{dilate-image}}

    \begin{Verbatim}[commandchars=\\\{\}]
{\color{incolor}In [{\color{incolor}15}]:} \PY{k+kn}{from} \PY{n+nn}{skimage}\PY{n+nn}{.}\PY{n+nn}{morphology} \PY{k}{import} \PY{n}{closing}
         \PY{k+kn}{from} \PY{n+nn}{skimage}\PY{n+nn}{.}\PY{n+nn}{morphology} \PY{k}{import} \PY{n}{square}
         
         \PY{n}{plt}\PY{o}{.}\PY{n}{imshow}\PY{p}{(}\PY{n}{closing}\PY{p}{(}\PY{n}{thresh}\PY{p}{,} \PY{n}{square}\PY{p}{(}\PY{l+m+mi}{3}\PY{p}{)}\PY{p}{)}\PY{p}{,} \PY{n}{cmap}\PY{o}{=}\PY{l+s+s1}{\PYZsq{}}\PY{l+s+s1}{gray}\PY{l+s+s1}{\PYZsq{}}\PY{p}{)}
\end{Verbatim}


\begin{Verbatim}[commandchars=\\\{\}]
{\color{outcolor}Out[{\color{outcolor}15}]:} <matplotlib.image.AxesImage at 0x237a9788048>
\end{Verbatim}
            
    \begin{center}
    \adjustimage{max size={0.9\linewidth}{0.9\paperheight}}{output_23_1.png}
    \end{center}
    { \hspace*{\fill} \\}
    
    As expected that with a reasonable threshold, our dialated image
generates a pretty neat ``blob'' of the person. Notice that the hip
region is still broken, but this seems to be a quirk of the image, and
not of the model that we have built

    \hypertarget{connected-components-algorithm}{%
\section{Connected Components
Algorithm}\label{connected-components-algorithm}}

    \begin{Verbatim}[commandchars=\\\{\}]
{\color{incolor}In [{\color{incolor}16}]:} \PY{k+kn}{from} \PY{n+nn}{skimage}\PY{n+nn}{.}\PY{n+nn}{segmentation} \PY{k}{import} \PY{n}{clear\PYZus{}border}
         \PY{k+kn}{from} \PY{n+nn}{skimage}\PY{n+nn}{.}\PY{n+nn}{measure} \PY{k}{import} \PY{n}{label}\PY{p}{,} \PY{n}{regionprops}
         \PY{k+kn}{from} \PY{n+nn}{skimage}\PY{n+nn}{.}\PY{n+nn}{morphology} \PY{k}{import} \PY{n}{closing}\PY{p}{,} \PY{n}{square}
         \PY{k+kn}{from} \PY{n+nn}{skimage}\PY{n+nn}{.}\PY{n+nn}{color} \PY{k}{import} \PY{n}{label2rgb}
         
         \PY{n}{bw} \PY{o}{=} \PY{n}{dilation}\PY{p}{(}\PY{n}{thresh}\PY{p}{,} \PY{n}{square}\PY{p}{(}\PY{l+m+mi}{3}\PY{p}{)}\PY{p}{)}
         
         \PY{n}{cleared} \PY{o}{=} \PY{n}{clear\PYZus{}border}\PY{p}{(}\PY{n}{bw}\PY{p}{)}
         
         \PY{n}{label\PYZus{}image} \PY{o}{=} \PY{n}{label}\PY{p}{(}\PY{n}{cleared}\PY{p}{)}
         \PY{n}{image\PYZus{}label\PYZus{}overlay} \PY{o}{=} \PY{n}{label2rgb}\PY{p}{(}\PY{n}{label\PYZus{}image}\PY{p}{,} \PY{n}{image}\PY{o}{=}\PY{n}{woman}\PY{p}{)}
         
         \PY{n}{plt}\PY{o}{.}\PY{n}{imshow}\PY{p}{(}\PY{n}{image\PYZus{}label\PYZus{}overlay}\PY{p}{)}
\end{Verbatim}


\begin{Verbatim}[commandchars=\\\{\}]
{\color{outcolor}Out[{\color{outcolor}16}]:} <matplotlib.image.AxesImage at 0x237a97e6278>
\end{Verbatim}
            
    \begin{center}
    \adjustimage{max size={0.9\linewidth}{0.9\paperheight}}{output_26_1.png}
    \end{center}
    { \hspace*{\fill} \\}
    
    We can observe that our connected components algorithm does a pretty
good job of identifying the person as a single component, (highlighted
here in blue)


    % Add a bibliography block to the postdoc
    
    
    
    \end{document}
