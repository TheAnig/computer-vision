
% Default to the notebook output style

    


% Inherit from the specified cell style.




    
\documentclass[11pt]{article}

    
    
    \usepackage[T1]{fontenc}
    % Nicer default font (+ math font) than Computer Modern for most use cases
    \usepackage{mathpazo}

    % Basic figure setup, for now with no caption control since it's done
    % automatically by Pandoc (which extracts ![](path) syntax from Markdown).
    \usepackage{graphicx}
    % We will generate all images so they have a width \maxwidth. This means
    % that they will get their normal width if they fit onto the page, but
    % are scaled down if they would overflow the margins.
    \makeatletter
    \def\maxwidth{\ifdim\Gin@nat@width>\linewidth\linewidth
    \else\Gin@nat@width\fi}
    \makeatother
    \let\Oldincludegraphics\includegraphics
    % Set max figure width to be 80% of text width, for now hardcoded.
    \renewcommand{\includegraphics}[1]{\Oldincludegraphics[width=.8\maxwidth]{#1}}
    % Ensure that by default, figures have no caption (until we provide a
    % proper Figure object with a Caption API and a way to capture that
    % in the conversion process - todo).
    \usepackage{caption}
    \DeclareCaptionLabelFormat{nolabel}{}
    \captionsetup{labelformat=nolabel}

    \usepackage{adjustbox} % Used to constrain images to a maximum size 
    \usepackage{xcolor} % Allow colors to be defined
    \usepackage{enumerate} % Needed for markdown enumerations to work
    \usepackage{geometry} % Used to adjust the document margins
    \usepackage{amsmath} % Equations
    \usepackage{amssymb} % Equations
    \usepackage{textcomp} % defines textquotesingle
    % Hack from http://tex.stackexchange.com/a/47451/13684:
    \AtBeginDocument{%
        \def\PYZsq{\textquotesingle}% Upright quotes in Pygmentized code
    }
    \usepackage{upquote} % Upright quotes for verbatim code
    \usepackage{eurosym} % defines \euro
    \usepackage[mathletters]{ucs} % Extended unicode (utf-8) support
    \usepackage[utf8x]{inputenc} % Allow utf-8 characters in the tex document
    \usepackage{fancyvrb} % verbatim replacement that allows latex
    \usepackage{grffile} % extends the file name processing of package graphics 
                         % to support a larger range 
    % The hyperref package gives us a pdf with properly built
    % internal navigation ('pdf bookmarks' for the table of contents,
    % internal cross-reference links, web links for URLs, etc.)
    \usepackage{hyperref}
    \usepackage{longtable} % longtable support required by pandoc >1.10
    \usepackage{booktabs}  % table support for pandoc > 1.12.2
    \usepackage[inline]{enumitem} % IRkernel/repr support (it uses the enumerate* environment)
    \usepackage[normalem]{ulem} % ulem is needed to support strikethroughs (\sout)
                                % normalem makes italics be italics, not underlines
    

    
    
    % Colors for the hyperref package
    \definecolor{urlcolor}{rgb}{0,.145,.698}
    \definecolor{linkcolor}{rgb}{.71,0.21,0.01}
    \definecolor{citecolor}{rgb}{.12,.54,.11}

    % ANSI colors
    \definecolor{ansi-black}{HTML}{3E424D}
    \definecolor{ansi-black-intense}{HTML}{282C36}
    \definecolor{ansi-red}{HTML}{E75C58}
    \definecolor{ansi-red-intense}{HTML}{B22B31}
    \definecolor{ansi-green}{HTML}{00A250}
    \definecolor{ansi-green-intense}{HTML}{007427}
    \definecolor{ansi-yellow}{HTML}{DDB62B}
    \definecolor{ansi-yellow-intense}{HTML}{B27D12}
    \definecolor{ansi-blue}{HTML}{208FFB}
    \definecolor{ansi-blue-intense}{HTML}{0065CA}
    \definecolor{ansi-magenta}{HTML}{D160C4}
    \definecolor{ansi-magenta-intense}{HTML}{A03196}
    \definecolor{ansi-cyan}{HTML}{60C6C8}
    \definecolor{ansi-cyan-intense}{HTML}{258F8F}
    \definecolor{ansi-white}{HTML}{C5C1B4}
    \definecolor{ansi-white-intense}{HTML}{A1A6B2}

    % commands and environments needed by pandoc snippets
    % extracted from the output of `pandoc -s`
    \providecommand{\tightlist}{%
      \setlength{\itemsep}{0pt}\setlength{\parskip}{0pt}}
    \DefineVerbatimEnvironment{Highlighting}{Verbatim}{commandchars=\\\{\}}
    % Add ',fontsize=\small' for more characters per line
    \newenvironment{Shaded}{}{}
    \newcommand{\KeywordTok}[1]{\textcolor[rgb]{0.00,0.44,0.13}{\textbf{{#1}}}}
    \newcommand{\DataTypeTok}[1]{\textcolor[rgb]{0.56,0.13,0.00}{{#1}}}
    \newcommand{\DecValTok}[1]{\textcolor[rgb]{0.25,0.63,0.44}{{#1}}}
    \newcommand{\BaseNTok}[1]{\textcolor[rgb]{0.25,0.63,0.44}{{#1}}}
    \newcommand{\FloatTok}[1]{\textcolor[rgb]{0.25,0.63,0.44}{{#1}}}
    \newcommand{\CharTok}[1]{\textcolor[rgb]{0.25,0.44,0.63}{{#1}}}
    \newcommand{\StringTok}[1]{\textcolor[rgb]{0.25,0.44,0.63}{{#1}}}
    \newcommand{\CommentTok}[1]{\textcolor[rgb]{0.38,0.63,0.69}{\textit{{#1}}}}
    \newcommand{\OtherTok}[1]{\textcolor[rgb]{0.00,0.44,0.13}{{#1}}}
    \newcommand{\AlertTok}[1]{\textcolor[rgb]{1.00,0.00,0.00}{\textbf{{#1}}}}
    \newcommand{\FunctionTok}[1]{\textcolor[rgb]{0.02,0.16,0.49}{{#1}}}
    \newcommand{\RegionMarkerTok}[1]{{#1}}
    \newcommand{\ErrorTok}[1]{\textcolor[rgb]{1.00,0.00,0.00}{\textbf{{#1}}}}
    \newcommand{\NormalTok}[1]{{#1}}
    
    % Additional commands for more recent versions of Pandoc
    \newcommand{\ConstantTok}[1]{\textcolor[rgb]{0.53,0.00,0.00}{{#1}}}
    \newcommand{\SpecialCharTok}[1]{\textcolor[rgb]{0.25,0.44,0.63}{{#1}}}
    \newcommand{\VerbatimStringTok}[1]{\textcolor[rgb]{0.25,0.44,0.63}{{#1}}}
    \newcommand{\SpecialStringTok}[1]{\textcolor[rgb]{0.73,0.40,0.53}{{#1}}}
    \newcommand{\ImportTok}[1]{{#1}}
    \newcommand{\DocumentationTok}[1]{\textcolor[rgb]{0.73,0.13,0.13}{\textit{{#1}}}}
    \newcommand{\AnnotationTok}[1]{\textcolor[rgb]{0.38,0.63,0.69}{\textbf{\textit{{#1}}}}}
    \newcommand{\CommentVarTok}[1]{\textcolor[rgb]{0.38,0.63,0.69}{\textbf{\textit{{#1}}}}}
    \newcommand{\VariableTok}[1]{\textcolor[rgb]{0.10,0.09,0.49}{{#1}}}
    \newcommand{\ControlFlowTok}[1]{\textcolor[rgb]{0.00,0.44,0.13}{\textbf{{#1}}}}
    \newcommand{\OperatorTok}[1]{\textcolor[rgb]{0.40,0.40,0.40}{{#1}}}
    \newcommand{\BuiltInTok}[1]{{#1}}
    \newcommand{\ExtensionTok}[1]{{#1}}
    \newcommand{\PreprocessorTok}[1]{\textcolor[rgb]{0.74,0.48,0.00}{{#1}}}
    \newcommand{\AttributeTok}[1]{\textcolor[rgb]{0.49,0.56,0.16}{{#1}}}
    \newcommand{\InformationTok}[1]{\textcolor[rgb]{0.38,0.63,0.69}{\textbf{\textit{{#1}}}}}
    \newcommand{\WarningTok}[1]{\textcolor[rgb]{0.38,0.63,0.69}{\textbf{\textit{{#1}}}}}
    
    
    % Define a nice break command that doesn't care if a line doesn't already
    % exist.
    \def\br{\hspace*{\fill} \\* }
    % Math Jax compatability definitions
    \def\gt{>}
    \def\lt{<}
    % Document parameters
    \title{hw4-soln}
    
    
    

    % Pygments definitions
    
\makeatletter
\def\PY@reset{\let\PY@it=\relax \let\PY@bf=\relax%
    \let\PY@ul=\relax \let\PY@tc=\relax%
    \let\PY@bc=\relax \let\PY@ff=\relax}
\def\PY@tok#1{\csname PY@tok@#1\endcsname}
\def\PY@toks#1+{\ifx\relax#1\empty\else%
    \PY@tok{#1}\expandafter\PY@toks\fi}
\def\PY@do#1{\PY@bc{\PY@tc{\PY@ul{%
    \PY@it{\PY@bf{\PY@ff{#1}}}}}}}
\def\PY#1#2{\PY@reset\PY@toks#1+\relax+\PY@do{#2}}

\expandafter\def\csname PY@tok@w\endcsname{\def\PY@tc##1{\textcolor[rgb]{0.73,0.73,0.73}{##1}}}
\expandafter\def\csname PY@tok@c\endcsname{\let\PY@it=\textit\def\PY@tc##1{\textcolor[rgb]{0.25,0.50,0.50}{##1}}}
\expandafter\def\csname PY@tok@cp\endcsname{\def\PY@tc##1{\textcolor[rgb]{0.74,0.48,0.00}{##1}}}
\expandafter\def\csname PY@tok@k\endcsname{\let\PY@bf=\textbf\def\PY@tc##1{\textcolor[rgb]{0.00,0.50,0.00}{##1}}}
\expandafter\def\csname PY@tok@kp\endcsname{\def\PY@tc##1{\textcolor[rgb]{0.00,0.50,0.00}{##1}}}
\expandafter\def\csname PY@tok@kt\endcsname{\def\PY@tc##1{\textcolor[rgb]{0.69,0.00,0.25}{##1}}}
\expandafter\def\csname PY@tok@o\endcsname{\def\PY@tc##1{\textcolor[rgb]{0.40,0.40,0.40}{##1}}}
\expandafter\def\csname PY@tok@ow\endcsname{\let\PY@bf=\textbf\def\PY@tc##1{\textcolor[rgb]{0.67,0.13,1.00}{##1}}}
\expandafter\def\csname PY@tok@nb\endcsname{\def\PY@tc##1{\textcolor[rgb]{0.00,0.50,0.00}{##1}}}
\expandafter\def\csname PY@tok@nf\endcsname{\def\PY@tc##1{\textcolor[rgb]{0.00,0.00,1.00}{##1}}}
\expandafter\def\csname PY@tok@nc\endcsname{\let\PY@bf=\textbf\def\PY@tc##1{\textcolor[rgb]{0.00,0.00,1.00}{##1}}}
\expandafter\def\csname PY@tok@nn\endcsname{\let\PY@bf=\textbf\def\PY@tc##1{\textcolor[rgb]{0.00,0.00,1.00}{##1}}}
\expandafter\def\csname PY@tok@ne\endcsname{\let\PY@bf=\textbf\def\PY@tc##1{\textcolor[rgb]{0.82,0.25,0.23}{##1}}}
\expandafter\def\csname PY@tok@nv\endcsname{\def\PY@tc##1{\textcolor[rgb]{0.10,0.09,0.49}{##1}}}
\expandafter\def\csname PY@tok@no\endcsname{\def\PY@tc##1{\textcolor[rgb]{0.53,0.00,0.00}{##1}}}
\expandafter\def\csname PY@tok@nl\endcsname{\def\PY@tc##1{\textcolor[rgb]{0.63,0.63,0.00}{##1}}}
\expandafter\def\csname PY@tok@ni\endcsname{\let\PY@bf=\textbf\def\PY@tc##1{\textcolor[rgb]{0.60,0.60,0.60}{##1}}}
\expandafter\def\csname PY@tok@na\endcsname{\def\PY@tc##1{\textcolor[rgb]{0.49,0.56,0.16}{##1}}}
\expandafter\def\csname PY@tok@nt\endcsname{\let\PY@bf=\textbf\def\PY@tc##1{\textcolor[rgb]{0.00,0.50,0.00}{##1}}}
\expandafter\def\csname PY@tok@nd\endcsname{\def\PY@tc##1{\textcolor[rgb]{0.67,0.13,1.00}{##1}}}
\expandafter\def\csname PY@tok@s\endcsname{\def\PY@tc##1{\textcolor[rgb]{0.73,0.13,0.13}{##1}}}
\expandafter\def\csname PY@tok@sd\endcsname{\let\PY@it=\textit\def\PY@tc##1{\textcolor[rgb]{0.73,0.13,0.13}{##1}}}
\expandafter\def\csname PY@tok@si\endcsname{\let\PY@bf=\textbf\def\PY@tc##1{\textcolor[rgb]{0.73,0.40,0.53}{##1}}}
\expandafter\def\csname PY@tok@se\endcsname{\let\PY@bf=\textbf\def\PY@tc##1{\textcolor[rgb]{0.73,0.40,0.13}{##1}}}
\expandafter\def\csname PY@tok@sr\endcsname{\def\PY@tc##1{\textcolor[rgb]{0.73,0.40,0.53}{##1}}}
\expandafter\def\csname PY@tok@ss\endcsname{\def\PY@tc##1{\textcolor[rgb]{0.10,0.09,0.49}{##1}}}
\expandafter\def\csname PY@tok@sx\endcsname{\def\PY@tc##1{\textcolor[rgb]{0.00,0.50,0.00}{##1}}}
\expandafter\def\csname PY@tok@m\endcsname{\def\PY@tc##1{\textcolor[rgb]{0.40,0.40,0.40}{##1}}}
\expandafter\def\csname PY@tok@gh\endcsname{\let\PY@bf=\textbf\def\PY@tc##1{\textcolor[rgb]{0.00,0.00,0.50}{##1}}}
\expandafter\def\csname PY@tok@gu\endcsname{\let\PY@bf=\textbf\def\PY@tc##1{\textcolor[rgb]{0.50,0.00,0.50}{##1}}}
\expandafter\def\csname PY@tok@gd\endcsname{\def\PY@tc##1{\textcolor[rgb]{0.63,0.00,0.00}{##1}}}
\expandafter\def\csname PY@tok@gi\endcsname{\def\PY@tc##1{\textcolor[rgb]{0.00,0.63,0.00}{##1}}}
\expandafter\def\csname PY@tok@gr\endcsname{\def\PY@tc##1{\textcolor[rgb]{1.00,0.00,0.00}{##1}}}
\expandafter\def\csname PY@tok@ge\endcsname{\let\PY@it=\textit}
\expandafter\def\csname PY@tok@gs\endcsname{\let\PY@bf=\textbf}
\expandafter\def\csname PY@tok@gp\endcsname{\let\PY@bf=\textbf\def\PY@tc##1{\textcolor[rgb]{0.00,0.00,0.50}{##1}}}
\expandafter\def\csname PY@tok@go\endcsname{\def\PY@tc##1{\textcolor[rgb]{0.53,0.53,0.53}{##1}}}
\expandafter\def\csname PY@tok@gt\endcsname{\def\PY@tc##1{\textcolor[rgb]{0.00,0.27,0.87}{##1}}}
\expandafter\def\csname PY@tok@err\endcsname{\def\PY@bc##1{\setlength{\fboxsep}{0pt}\fcolorbox[rgb]{1.00,0.00,0.00}{1,1,1}{\strut ##1}}}
\expandafter\def\csname PY@tok@kc\endcsname{\let\PY@bf=\textbf\def\PY@tc##1{\textcolor[rgb]{0.00,0.50,0.00}{##1}}}
\expandafter\def\csname PY@tok@kd\endcsname{\let\PY@bf=\textbf\def\PY@tc##1{\textcolor[rgb]{0.00,0.50,0.00}{##1}}}
\expandafter\def\csname PY@tok@kn\endcsname{\let\PY@bf=\textbf\def\PY@tc##1{\textcolor[rgb]{0.00,0.50,0.00}{##1}}}
\expandafter\def\csname PY@tok@kr\endcsname{\let\PY@bf=\textbf\def\PY@tc##1{\textcolor[rgb]{0.00,0.50,0.00}{##1}}}
\expandafter\def\csname PY@tok@bp\endcsname{\def\PY@tc##1{\textcolor[rgb]{0.00,0.50,0.00}{##1}}}
\expandafter\def\csname PY@tok@fm\endcsname{\def\PY@tc##1{\textcolor[rgb]{0.00,0.00,1.00}{##1}}}
\expandafter\def\csname PY@tok@vc\endcsname{\def\PY@tc##1{\textcolor[rgb]{0.10,0.09,0.49}{##1}}}
\expandafter\def\csname PY@tok@vg\endcsname{\def\PY@tc##1{\textcolor[rgb]{0.10,0.09,0.49}{##1}}}
\expandafter\def\csname PY@tok@vi\endcsname{\def\PY@tc##1{\textcolor[rgb]{0.10,0.09,0.49}{##1}}}
\expandafter\def\csname PY@tok@vm\endcsname{\def\PY@tc##1{\textcolor[rgb]{0.10,0.09,0.49}{##1}}}
\expandafter\def\csname PY@tok@sa\endcsname{\def\PY@tc##1{\textcolor[rgb]{0.73,0.13,0.13}{##1}}}
\expandafter\def\csname PY@tok@sb\endcsname{\def\PY@tc##1{\textcolor[rgb]{0.73,0.13,0.13}{##1}}}
\expandafter\def\csname PY@tok@sc\endcsname{\def\PY@tc##1{\textcolor[rgb]{0.73,0.13,0.13}{##1}}}
\expandafter\def\csname PY@tok@dl\endcsname{\def\PY@tc##1{\textcolor[rgb]{0.73,0.13,0.13}{##1}}}
\expandafter\def\csname PY@tok@s2\endcsname{\def\PY@tc##1{\textcolor[rgb]{0.73,0.13,0.13}{##1}}}
\expandafter\def\csname PY@tok@sh\endcsname{\def\PY@tc##1{\textcolor[rgb]{0.73,0.13,0.13}{##1}}}
\expandafter\def\csname PY@tok@s1\endcsname{\def\PY@tc##1{\textcolor[rgb]{0.73,0.13,0.13}{##1}}}
\expandafter\def\csname PY@tok@mb\endcsname{\def\PY@tc##1{\textcolor[rgb]{0.40,0.40,0.40}{##1}}}
\expandafter\def\csname PY@tok@mf\endcsname{\def\PY@tc##1{\textcolor[rgb]{0.40,0.40,0.40}{##1}}}
\expandafter\def\csname PY@tok@mh\endcsname{\def\PY@tc##1{\textcolor[rgb]{0.40,0.40,0.40}{##1}}}
\expandafter\def\csname PY@tok@mi\endcsname{\def\PY@tc##1{\textcolor[rgb]{0.40,0.40,0.40}{##1}}}
\expandafter\def\csname PY@tok@il\endcsname{\def\PY@tc##1{\textcolor[rgb]{0.40,0.40,0.40}{##1}}}
\expandafter\def\csname PY@tok@mo\endcsname{\def\PY@tc##1{\textcolor[rgb]{0.40,0.40,0.40}{##1}}}
\expandafter\def\csname PY@tok@ch\endcsname{\let\PY@it=\textit\def\PY@tc##1{\textcolor[rgb]{0.25,0.50,0.50}{##1}}}
\expandafter\def\csname PY@tok@cm\endcsname{\let\PY@it=\textit\def\PY@tc##1{\textcolor[rgb]{0.25,0.50,0.50}{##1}}}
\expandafter\def\csname PY@tok@cpf\endcsname{\let\PY@it=\textit\def\PY@tc##1{\textcolor[rgb]{0.25,0.50,0.50}{##1}}}
\expandafter\def\csname PY@tok@c1\endcsname{\let\PY@it=\textit\def\PY@tc##1{\textcolor[rgb]{0.25,0.50,0.50}{##1}}}
\expandafter\def\csname PY@tok@cs\endcsname{\let\PY@it=\textit\def\PY@tc##1{\textcolor[rgb]{0.25,0.50,0.50}{##1}}}

\def\PYZbs{\char`\\}
\def\PYZus{\char`\_}
\def\PYZob{\char`\{}
\def\PYZcb{\char`\}}
\def\PYZca{\char`\^}
\def\PYZam{\char`\&}
\def\PYZlt{\char`\<}
\def\PYZgt{\char`\>}
\def\PYZsh{\char`\#}
\def\PYZpc{\char`\%}
\def\PYZdl{\char`\$}
\def\PYZhy{\char`\-}
\def\PYZsq{\char`\'}
\def\PYZdq{\char`\"}
\def\PYZti{\char`\~}
% for compatibility with earlier versions
\def\PYZat{@}
\def\PYZlb{[}
\def\PYZrb{]}
\makeatother


    % Exact colors from NB
    \definecolor{incolor}{rgb}{0.0, 0.0, 0.5}
    \definecolor{outcolor}{rgb}{0.545, 0.0, 0.0}



    
    % Prevent overflowing lines due to hard-to-break entities
    \sloppy 
    % Setup hyperref package
    \hypersetup{
      breaklinks=true,  % so long urls are correctly broken across lines
      colorlinks=true,
      urlcolor=urlcolor,
      linkcolor=linkcolor,
      citecolor=citecolor,
      }
    % Slightly bigger margins than the latex defaults
    
    \geometry{verbose,tmargin=1in,bmargin=1in,lmargin=1in,rmargin=1in}
    
    

    \begin{document}
    
    
    \noindent
\large\textbf{Homework Assignment 4} \hfill \textbf{Anirudh Ganesh} \\
\normalsize Computer Vision for HCI \hfill CSE5524 (Au `18) \\
Prof. Jim Davis \hfill Score: \_\_\_/10 \\
TA: Sayan Mandal \hfill Due Date: 09/18/18
    
    

    
    \hypertarget{imports}{%
\subsection{Imports}\label{imports}}

    \begin{Verbatim}[commandchars=\\\{\}]
{\color{incolor}In [{\color{incolor}1}]:} \PY{k+kn}{from} \PY{n+nn}{skimage}\PY{n+nn}{.}\PY{n+nn}{io} \PY{k}{import} \PY{n}{imread}
        \PY{k+kn}{from} \PY{n+nn}{skimage}\PY{n+nn}{.}\PY{n+nn}{filters} \PY{k}{import} \PY{n}{gaussian}
        \PY{k+kn}{import} \PY{n+nn}{numpy} \PY{k}{as} \PY{n+nn}{np}
        \PY{k+kn}{from} \PY{n+nn}{matplotlib} \PY{k}{import} \PY{n}{pyplot} \PY{k}{as} \PY{n}{plt}
        \PY{k+kn}{from} \PY{n+nn}{skimage} \PY{k}{import} \PY{n}{img\PYZus{}as\PYZus{}float}
        \PY{k+kn}{import} \PY{n+nn}{math}
\end{Verbatim}


    \hypertarget{function-to-compute-the-moments-of-the-image}{%
\section{Function to compute the moments of the
image}\label{function-to-compute-the-moments-of-the-image}}

    \begin{Verbatim}[commandchars=\\\{\}]
{\color{incolor}In [{\color{incolor}2}]:} \PY{k+kn}{from} \PY{n+nn}{numpy} \PY{k}{import} \PY{n}{mgrid}\PY{p}{,} \PY{n+nb}{sum}
        
        
        \PY{k}{def} \PY{n+nf}{similitudeMoments}\PY{p}{(}\PY{n}{image}\PY{p}{)}\PY{p}{:}
            
            \PY{c+c1}{\PYZsh{}\PYZsh{}\PYZsh{} Make sure the image is a grayscale image}
            \PY{k}{assert} \PY{n+nb}{len}\PY{p}{(}\PY{n}{image}\PY{o}{.}\PY{n}{shape}\PY{p}{)} \PY{o}{==} \PY{l+m+mi}{2}
            
            \PY{c+c1}{\PYZsh{}\PYZsh{} Temp grid for storing intermediate operations}
            \PY{n}{x}\PY{p}{,} \PY{n}{y} \PY{o}{=} \PY{n}{mgrid}\PY{p}{[}\PY{p}{:}\PY{n}{image}\PY{o}{.}\PY{n}{shape}\PY{p}{[}\PY{l+m+mi}{0}\PY{p}{]}\PY{p}{,}\PY{p}{:}\PY{n}{image}\PY{o}{.}\PY{n}{shape}\PY{p}{[}\PY{l+m+mi}{1}\PY{p}{]}\PY{p}{]}
            
            \PY{c+c1}{\PYZsh{}\PYZsh{} Our final dictionary that contains the }
            \PY{n}{moments} \PY{o}{=} \PY{p}{\PYZob{}}\PY{p}{\PYZcb{}}
            
            
            \PY{n}{moments}\PY{p}{[}\PY{l+s+s1}{\PYZsq{}}\PY{l+s+s1}{mean\PYZus{}x}\PY{l+s+s1}{\PYZsq{}}\PY{p}{]} \PY{o}{=} \PY{n+nb}{sum}\PY{p}{(}\PY{n}{x}\PY{o}{*}\PY{n}{image}\PY{p}{)}\PY{o}{/}\PY{n+nb}{sum}\PY{p}{(}\PY{n}{image}\PY{p}{)}
            \PY{n}{moments}\PY{p}{[}\PY{l+s+s1}{\PYZsq{}}\PY{l+s+s1}{mean\PYZus{}y}\PY{l+s+s1}{\PYZsq{}}\PY{p}{]} \PY{o}{=} \PY{n+nb}{sum}\PY{p}{(}\PY{n}{y}\PY{o}{*}\PY{n}{image}\PY{p}{)}\PY{o}{/}\PY{n+nb}{sum}\PY{p}{(}\PY{n}{image}\PY{p}{)}
            
            \PY{c+c1}{\PYZsh{}\PYZsh{} Spatial moments: Spatial moments  often used to describe region shape}
            
            \PY{c+c1}{\PYZsh{} Zeroth Order}
            \PY{n}{moments}\PY{p}{[}\PY{l+s+s1}{\PYZsq{}}\PY{l+s+s1}{m00}\PY{l+s+s1}{\PYZsq{}}\PY{p}{]} \PY{o}{=} \PY{n+nb}{sum}\PY{p}{(}\PY{n}{image}\PY{p}{)}
            
            \PY{c+c1}{\PYZsh{} First Order}
            \PY{n}{moments}\PY{p}{[}\PY{l+s+s1}{\PYZsq{}}\PY{l+s+s1}{m01}\PY{l+s+s1}{\PYZsq{}}\PY{p}{]} \PY{o}{=} \PY{n+nb}{sum}\PY{p}{(}\PY{n}{x}\PY{o}{*}\PY{n}{image}\PY{p}{)}
            \PY{n}{moments}\PY{p}{[}\PY{l+s+s1}{\PYZsq{}}\PY{l+s+s1}{m10}\PY{l+s+s1}{\PYZsq{}}\PY{p}{]} \PY{o}{=} \PY{n+nb}{sum}\PY{p}{(}\PY{n}{y}\PY{o}{*}\PY{n}{image}\PY{p}{)}
            
            \PY{c+c1}{\PYZsh{} Second Order}
            \PY{n}{moments}\PY{p}{[}\PY{l+s+s1}{\PYZsq{}}\PY{l+s+s1}{m11}\PY{l+s+s1}{\PYZsq{}}\PY{p}{]} \PY{o}{=} \PY{n+nb}{sum}\PY{p}{(}\PY{n}{y}\PY{o}{*}\PY{n}{x}\PY{o}{*}\PY{n}{image}\PY{p}{)}
            \PY{n}{moments}\PY{p}{[}\PY{l+s+s1}{\PYZsq{}}\PY{l+s+s1}{m02}\PY{l+s+s1}{\PYZsq{}}\PY{p}{]} \PY{o}{=} \PY{n+nb}{sum}\PY{p}{(}\PY{n}{x}\PY{o}{*}\PY{o}{*}\PY{l+m+mi}{2}\PY{o}{*}\PY{n}{image}\PY{p}{)}
            \PY{n}{moments}\PY{p}{[}\PY{l+s+s1}{\PYZsq{}}\PY{l+s+s1}{m20}\PY{l+s+s1}{\PYZsq{}}\PY{p}{]} \PY{o}{=} \PY{n+nb}{sum}\PY{p}{(}\PY{n}{y}\PY{o}{*}\PY{o}{*}\PY{l+m+mi}{2}\PY{o}{*}\PY{n}{image}\PY{p}{)}
            
            \PY{c+c1}{\PYZsh{} Third Order}
            \PY{n}{moments}\PY{p}{[}\PY{l+s+s1}{\PYZsq{}}\PY{l+s+s1}{m12}\PY{l+s+s1}{\PYZsq{}}\PY{p}{]} \PY{o}{=} \PY{n+nb}{sum}\PY{p}{(}\PY{n}{x}\PY{o}{*}\PY{n}{y}\PY{o}{*}\PY{o}{*}\PY{l+m+mi}{2}\PY{o}{*}\PY{n}{image}\PY{p}{)}
            \PY{n}{moments}\PY{p}{[}\PY{l+s+s1}{\PYZsq{}}\PY{l+s+s1}{m21}\PY{l+s+s1}{\PYZsq{}}\PY{p}{]} \PY{o}{=} \PY{n+nb}{sum}\PY{p}{(}\PY{n}{x}\PY{o}{*}\PY{o}{*}\PY{l+m+mi}{2}\PY{o}{*}\PY{n}{y}\PY{o}{*}\PY{n}{image}\PY{p}{)}
            \PY{n}{moments}\PY{p}{[}\PY{l+s+s1}{\PYZsq{}}\PY{l+s+s1}{m03}\PY{l+s+s1}{\PYZsq{}}\PY{p}{]} \PY{o}{=} \PY{n+nb}{sum}\PY{p}{(}\PY{n}{x}\PY{o}{*}\PY{o}{*}\PY{l+m+mi}{3}\PY{o}{*}\PY{n}{image}\PY{p}{)}
            \PY{n}{moments}\PY{p}{[}\PY{l+s+s1}{\PYZsq{}}\PY{l+s+s1}{m30}\PY{l+s+s1}{\PYZsq{}}\PY{p}{]} \PY{o}{=} \PY{n+nb}{sum}\PY{p}{(}\PY{n}{y}\PY{o}{*}\PY{o}{*}\PY{l+m+mi}{3}\PY{o}{*}\PY{n}{image}\PY{p}{)}
            
            \PY{c+c1}{\PYZsh{}\PYZsh{} Central moments: Translation Invariant}
            
            \PY{c+c1}{\PYZsh{} First Order (Seem useless, in terms of calculating the final nu moments. But still kept)}
            \PY{n}{moments}\PY{p}{[}\PY{l+s+s1}{\PYZsq{}}\PY{l+s+s1}{mu01}\PY{l+s+s1}{\PYZsq{}}\PY{p}{]}\PY{o}{=} \PY{n+nb}{sum}\PY{p}{(}\PY{p}{(}\PY{n}{y}\PY{o}{\PYZhy{}}\PY{n}{moments}\PY{p}{[}\PY{l+s+s1}{\PYZsq{}}\PY{l+s+s1}{mean\PYZus{}y}\PY{l+s+s1}{\PYZsq{}}\PY{p}{]}\PY{p}{)}\PY{o}{*}\PY{n}{image}\PY{p}{)}
            \PY{n}{moments}\PY{p}{[}\PY{l+s+s1}{\PYZsq{}}\PY{l+s+s1}{mu10}\PY{l+s+s1}{\PYZsq{}}\PY{p}{]}\PY{o}{=} \PY{n+nb}{sum}\PY{p}{(}\PY{p}{(}\PY{n}{x}\PY{o}{\PYZhy{}}\PY{n}{moments}\PY{p}{[}\PY{l+s+s1}{\PYZsq{}}\PY{l+s+s1}{mean\PYZus{}x}\PY{l+s+s1}{\PYZsq{}}\PY{p}{]}\PY{p}{)}\PY{o}{*}\PY{n}{image}\PY{p}{)}
            
            \PY{c+c1}{\PYZsh{} Second Order (Moment Ellipse Orientation)}
            \PY{n}{moments}\PY{p}{[}\PY{l+s+s1}{\PYZsq{}}\PY{l+s+s1}{mu11}\PY{l+s+s1}{\PYZsq{}}\PY{p}{]} \PY{o}{=} \PY{n+nb}{sum}\PY{p}{(}\PY{p}{(}\PY{n}{x}\PY{o}{\PYZhy{}}\PY{n}{moments}\PY{p}{[}\PY{l+s+s1}{\PYZsq{}}\PY{l+s+s1}{mean\PYZus{}x}\PY{l+s+s1}{\PYZsq{}}\PY{p}{]}\PY{p}{)}\PY{o}{*}\PY{p}{(}\PY{n}{y}\PY{o}{\PYZhy{}}\PY{n}{moments}\PY{p}{[}\PY{l+s+s1}{\PYZsq{}}\PY{l+s+s1}{mean\PYZus{}y}\PY{l+s+s1}{\PYZsq{}}\PY{p}{]}\PY{p}{)}\PY{o}{*}\PY{n}{image}\PY{p}{)}
            \PY{n}{moments}\PY{p}{[}\PY{l+s+s1}{\PYZsq{}}\PY{l+s+s1}{mu02}\PY{l+s+s1}{\PYZsq{}}\PY{p}{]} \PY{o}{=} \PY{n+nb}{sum}\PY{p}{(}\PY{p}{(}\PY{n}{y}\PY{o}{\PYZhy{}}\PY{n}{moments}\PY{p}{[}\PY{l+s+s1}{\PYZsq{}}\PY{l+s+s1}{mean\PYZus{}y}\PY{l+s+s1}{\PYZsq{}}\PY{p}{]}\PY{p}{)}\PY{o}{*}\PY{o}{*}\PY{l+m+mi}{2}\PY{o}{*}\PY{n}{image}\PY{p}{)}
            \PY{n}{moments}\PY{p}{[}\PY{l+s+s1}{\PYZsq{}}\PY{l+s+s1}{mu20}\PY{l+s+s1}{\PYZsq{}}\PY{p}{]} \PY{o}{=} \PY{n+nb}{sum}\PY{p}{(}\PY{p}{(}\PY{n}{x}\PY{o}{\PYZhy{}}\PY{n}{moments}\PY{p}{[}\PY{l+s+s1}{\PYZsq{}}\PY{l+s+s1}{mean\PYZus{}x}\PY{l+s+s1}{\PYZsq{}}\PY{p}{]}\PY{p}{)}\PY{o}{*}\PY{o}{*}\PY{l+m+mi}{2}\PY{o}{*}\PY{n}{image}\PY{p}{)}
            
            \PY{c+c1}{\PYZsh{} Third Order (Skewness of the Image)}
            \PY{n}{moments}\PY{p}{[}\PY{l+s+s1}{\PYZsq{}}\PY{l+s+s1}{mu12}\PY{l+s+s1}{\PYZsq{}}\PY{p}{]} \PY{o}{=} \PY{n+nb}{sum}\PY{p}{(}\PY{p}{(}\PY{n}{x}\PY{o}{\PYZhy{}}\PY{n}{moments}\PY{p}{[}\PY{l+s+s1}{\PYZsq{}}\PY{l+s+s1}{mean\PYZus{}x}\PY{l+s+s1}{\PYZsq{}}\PY{p}{]}\PY{p}{)}\PY{o}{*}\PY{p}{(}\PY{n}{y}\PY{o}{\PYZhy{}}\PY{n}{moments}\PY{p}{[}\PY{l+s+s1}{\PYZsq{}}\PY{l+s+s1}{mean\PYZus{}y}\PY{l+s+s1}{\PYZsq{}}\PY{p}{]}\PY{p}{)}\PY{o}{*}\PY{o}{*}\PY{l+m+mi}{2}\PY{o}{*}\PY{n}{image}\PY{p}{)}
            \PY{n}{moments}\PY{p}{[}\PY{l+s+s1}{\PYZsq{}}\PY{l+s+s1}{mu21}\PY{l+s+s1}{\PYZsq{}}\PY{p}{]} \PY{o}{=} \PY{n+nb}{sum}\PY{p}{(}\PY{p}{(}\PY{n}{x}\PY{o}{\PYZhy{}}\PY{n}{moments}\PY{p}{[}\PY{l+s+s1}{\PYZsq{}}\PY{l+s+s1}{mean\PYZus{}x}\PY{l+s+s1}{\PYZsq{}}\PY{p}{]}\PY{p}{)}\PY{o}{*}\PY{o}{*}\PY{l+m+mi}{2}\PY{o}{*}\PY{p}{(}\PY{n}{y}\PY{o}{\PYZhy{}}\PY{n}{moments}\PY{p}{[}\PY{l+s+s1}{\PYZsq{}}\PY{l+s+s1}{mean\PYZus{}y}\PY{l+s+s1}{\PYZsq{}}\PY{p}{]}\PY{p}{)}\PY{o}{*}\PY{n}{image}\PY{p}{)}
            \PY{n}{moments}\PY{p}{[}\PY{l+s+s1}{\PYZsq{}}\PY{l+s+s1}{mu03}\PY{l+s+s1}{\PYZsq{}}\PY{p}{]} \PY{o}{=} \PY{n+nb}{sum}\PY{p}{(}\PY{p}{(}\PY{n}{y}\PY{o}{\PYZhy{}}\PY{n}{moments}\PY{p}{[}\PY{l+s+s1}{\PYZsq{}}\PY{l+s+s1}{mean\PYZus{}y}\PY{l+s+s1}{\PYZsq{}}\PY{p}{]}\PY{p}{)}\PY{o}{*}\PY{o}{*}\PY{l+m+mi}{3}\PY{o}{*}\PY{n}{image}\PY{p}{)}
            \PY{n}{moments}\PY{p}{[}\PY{l+s+s1}{\PYZsq{}}\PY{l+s+s1}{mu30}\PY{l+s+s1}{\PYZsq{}}\PY{p}{]} \PY{o}{=} \PY{n+nb}{sum}\PY{p}{(}\PY{p}{(}\PY{n}{x}\PY{o}{\PYZhy{}}\PY{n}{moments}\PY{p}{[}\PY{l+s+s1}{\PYZsq{}}\PY{l+s+s1}{mean\PYZus{}x}\PY{l+s+s1}{\PYZsq{}}\PY{p}{]}\PY{p}{)}\PY{o}{*}\PY{o}{*}\PY{l+m+mi}{3}\PY{o}{*}\PY{n}{image}\PY{p}{)}
            
            \PY{c+c1}{\PYZsh{} Similitude Moments: Invariant to translation and scale}
            
            \PY{n}{moments}\PY{p}{[}\PY{l+s+s1}{\PYZsq{}}\PY{l+s+s1}{eta11}\PY{l+s+s1}{\PYZsq{}}\PY{p}{]} \PY{o}{=} \PY{n}{moments}\PY{p}{[}\PY{l+s+s1}{\PYZsq{}}\PY{l+s+s1}{mu11}\PY{l+s+s1}{\PYZsq{}}\PY{p}{]} \PY{o}{/} \PY{n+nb}{sum}\PY{p}{(}\PY{n}{image}\PY{p}{)}\PY{o}{*}\PY{o}{*}\PY{p}{(}\PY{l+m+mi}{2}\PY{o}{/}\PY{l+m+mi}{2}\PY{o}{+}\PY{l+m+mi}{1}\PY{p}{)}
            \PY{n}{moments}\PY{p}{[}\PY{l+s+s1}{\PYZsq{}}\PY{l+s+s1}{eta12}\PY{l+s+s1}{\PYZsq{}}\PY{p}{]} \PY{o}{=} \PY{n}{moments}\PY{p}{[}\PY{l+s+s1}{\PYZsq{}}\PY{l+s+s1}{mu12}\PY{l+s+s1}{\PYZsq{}}\PY{p}{]} \PY{o}{/} \PY{n+nb}{sum}\PY{p}{(}\PY{n}{image}\PY{p}{)}\PY{o}{*}\PY{o}{*}\PY{p}{(}\PY{l+m+mi}{3}\PY{o}{/}\PY{l+m+mi}{2}\PY{o}{+}\PY{l+m+mi}{1}\PY{p}{)}
            \PY{n}{moments}\PY{p}{[}\PY{l+s+s1}{\PYZsq{}}\PY{l+s+s1}{eta21}\PY{l+s+s1}{\PYZsq{}}\PY{p}{]} \PY{o}{=} \PY{n}{moments}\PY{p}{[}\PY{l+s+s1}{\PYZsq{}}\PY{l+s+s1}{mu21}\PY{l+s+s1}{\PYZsq{}}\PY{p}{]} \PY{o}{/} \PY{n+nb}{sum}\PY{p}{(}\PY{n}{image}\PY{p}{)}\PY{o}{*}\PY{o}{*}\PY{p}{(}\PY{l+m+mi}{3}\PY{o}{/}\PY{l+m+mi}{2}\PY{o}{+}\PY{l+m+mi}{1}\PY{p}{)}
            \PY{n}{moments}\PY{p}{[}\PY{l+s+s1}{\PYZsq{}}\PY{l+s+s1}{eta02}\PY{l+s+s1}{\PYZsq{}}\PY{p}{]} \PY{o}{=} \PY{n}{moments}\PY{p}{[}\PY{l+s+s1}{\PYZsq{}}\PY{l+s+s1}{mu02}\PY{l+s+s1}{\PYZsq{}}\PY{p}{]} \PY{o}{/} \PY{n+nb}{sum}\PY{p}{(}\PY{n}{image}\PY{p}{)}\PY{o}{*}\PY{o}{*}\PY{p}{(}\PY{l+m+mi}{2}\PY{o}{/}\PY{l+m+mi}{2}\PY{o}{+}\PY{l+m+mi}{1}\PY{p}{)}
            \PY{n}{moments}\PY{p}{[}\PY{l+s+s1}{\PYZsq{}}\PY{l+s+s1}{eta20}\PY{l+s+s1}{\PYZsq{}}\PY{p}{]} \PY{o}{=} \PY{n}{moments}\PY{p}{[}\PY{l+s+s1}{\PYZsq{}}\PY{l+s+s1}{mu20}\PY{l+s+s1}{\PYZsq{}}\PY{p}{]} \PY{o}{/} \PY{n+nb}{sum}\PY{p}{(}\PY{n}{image}\PY{p}{)}\PY{o}{*}\PY{o}{*}\PY{p}{(}\PY{l+m+mi}{2}\PY{o}{/}\PY{l+m+mi}{2}\PY{o}{+}\PY{l+m+mi}{1}\PY{p}{)}
            \PY{n}{moments}\PY{p}{[}\PY{l+s+s1}{\PYZsq{}}\PY{l+s+s1}{eta03}\PY{l+s+s1}{\PYZsq{}}\PY{p}{]} \PY{o}{=} \PY{n}{moments}\PY{p}{[}\PY{l+s+s1}{\PYZsq{}}\PY{l+s+s1}{mu03}\PY{l+s+s1}{\PYZsq{}}\PY{p}{]} \PY{o}{/} \PY{n+nb}{sum}\PY{p}{(}\PY{n}{image}\PY{p}{)}\PY{o}{*}\PY{o}{*}\PY{p}{(}\PY{l+m+mi}{3}\PY{o}{/}\PY{l+m+mi}{2}\PY{o}{+}\PY{l+m+mi}{1}\PY{p}{)}
            \PY{n}{moments}\PY{p}{[}\PY{l+s+s1}{\PYZsq{}}\PY{l+s+s1}{eta30}\PY{l+s+s1}{\PYZsq{}}\PY{p}{]} \PY{o}{=} \PY{n}{moments}\PY{p}{[}\PY{l+s+s1}{\PYZsq{}}\PY{l+s+s1}{mu30}\PY{l+s+s1}{\PYZsq{}}\PY{p}{]} \PY{o}{/} \PY{n+nb}{sum}\PY{p}{(}\PY{n}{image}\PY{p}{)}\PY{o}{*}\PY{o}{*}\PY{p}{(}\PY{l+m+mi}{3}\PY{o}{/}\PY{l+m+mi}{2}\PY{o}{+}\PY{l+m+mi}{1}\PY{p}{)}
            
            \PY{k}{return} \PY{n}{moments}
\end{Verbatim}


    \begin{Verbatim}[commandchars=\\\{\}]
{\color{incolor}In [{\color{incolor}3}]:} \PY{n}{image\PYZus{}cube} \PY{o}{=} \PY{p}{[}\PY{p}{]}
        
        \PY{k}{for} \PY{n}{i} \PY{o+ow}{in} \PY{n+nb}{range}\PY{p}{(}\PY{l+m+mi}{1}\PY{p}{,}\PY{l+m+mi}{5}\PY{p}{)}\PY{p}{:}
            \PY{n}{tmp} \PY{o}{=} \PY{n}{imread}\PY{p}{(}\PY{n}{f}\PY{l+s+s1}{\PYZsq{}}\PY{l+s+s1}{./data/boxIm}\PY{l+s+si}{\PYZob{}i\PYZcb{}}\PY{l+s+s1}{.bmp}\PY{l+s+s1}{\PYZsq{}}\PY{p}{)}
            \PY{n}{tmp} \PY{o}{=} \PY{n}{img\PYZus{}as\PYZus{}float}\PY{p}{(}\PY{n}{tmp}\PY{p}{)}
            \PY{n}{image\PYZus{}cube}\PY{o}{.}\PY{n}{append}\PY{p}{(}\PY{n}{tmp}\PY{p}{)}
\end{Verbatim}


    \begin{Verbatim}[commandchars=\\\{\}]
{\color{incolor}In [{\color{incolor}4}]:} \PY{n}{tmom} \PY{o}{=} \PY{n}{similitudeMoments}\PY{p}{(}\PY{n}{image\PYZus{}cube}\PY{p}{[}\PY{l+m+mi}{1}\PY{p}{]}\PY{p}{)}
        \PY{n}{tmom}
\end{Verbatim}


\begin{Verbatim}[commandchars=\\\{\}]
{\color{outcolor}Out[{\color{outcolor}4}]:} \{'eta02': 0.1646090534979424,
         'eta03': 0.0,
         'eta11': 0.0,
         'eta12': 0.0,
         'eta20': 0.04215597711532671,
         'eta21': 0.0,
         'eta30': 0.0,
         'm00': 3321.0,
         'm01': 129519.0,
         'm02': 5516181.0,
         'm03': 251396379.0,
         'm10': 195939.0,
         'm11': 7641621.0,
         'm12': 521659359.0,
         'm20': 13375881.0,
         'm21': 325454679.0,
         'm30': 1003403619.0,
         'mean\_x': 39.0,
         'mean\_y': 59.0,
         'mu01': 0.0,
         'mu02': 1815480.0,
         'mu03': 0.0,
         'mu10': 0.0,
         'mu11': 0.0,
         'mu12': 0.0,
         'mu20': 464940.0,
         'mu21': 0.0,
         'mu30': 0.0\}
\end{Verbatim}
            
    \begin{Verbatim}[commandchars=\\\{\}]
{\color{incolor}In [{\color{incolor}5}]:} \PY{n}{f}\PY{p}{,} \PY{n}{axarr} \PY{o}{=} \PY{n}{plt}\PY{o}{.}\PY{n}{subplots}\PY{p}{(}\PY{l+m+mi}{4}\PY{p}{,}\PY{l+m+mi}{1}\PY{p}{,} \PY{n}{sharey}\PY{o}{=}\PY{l+s+s1}{\PYZsq{}}\PY{l+s+s1}{row}\PY{l+s+s1}{\PYZsq{}}\PY{p}{,} \PY{n}{dpi}\PY{o}{=}\PY{l+m+mi}{200}\PY{p}{)}
        \PY{n}{s1}\PY{o}{=} \PY{l+s+s2}{\PYZdq{}}\PY{l+s+s2}{\PYZdl{}}\PY{l+s+s2}{\PYZbs{}}\PY{l+s+s2}{eta\PYZus{}}\PY{l+s+si}{\PYZob{}02\PYZcb{}}\PY{l+s+s2}{\PYZdl{}}\PY{l+s+s2}{\PYZdq{}}
        \PY{n}{s2} \PY{o}{=} \PY{l+s+s2}{\PYZdq{}}\PY{l+s+s2}{\PYZdl{}}\PY{l+s+s2}{\PYZbs{}}\PY{l+s+s2}{eta\PYZus{}}\PY{l+s+si}{\PYZob{}20\PYZcb{}}\PY{l+s+s2}{\PYZdl{}}\PY{l+s+s2}{\PYZdq{}}
        \PY{k}{for} \PY{n}{i} \PY{o+ow}{in} \PY{n+nb}{range}\PY{p}{(}\PY{l+m+mi}{4}\PY{p}{)}\PY{p}{:}
            \PY{n}{tmom} \PY{o}{=} \PY{n}{similitudeMoments}\PY{p}{(}\PY{n}{image\PYZus{}cube}\PY{p}{[}\PY{n}{i}\PY{p}{]}\PY{p}{)}
            \PY{n}{axarr}\PY{p}{[}\PY{n}{i}\PY{p}{]}\PY{o}{.}\PY{n}{imshow}\PY{p}{(}\PY{n}{image\PYZus{}cube}\PY{p}{[}\PY{n}{i}\PY{p}{]}\PY{p}{,} \PY{n}{cmap}\PY{o}{=}\PY{l+s+s1}{\PYZsq{}}\PY{l+s+s1}{gray}\PY{l+s+s1}{\PYZsq{}}\PY{p}{)}
            \PY{n}{axarr}\PY{p}{[}\PY{n}{i}\PY{p}{]}\PY{o}{.}\PY{n}{axis}\PY{p}{(}\PY{l+s+s1}{\PYZsq{}}\PY{l+s+s1}{off}\PY{l+s+s1}{\PYZsq{}}\PY{p}{)}
            \PY{n}{axarr}\PY{p}{[}\PY{n}{i}\PY{p}{]}\PY{o}{.}\PY{n}{set\PYZus{}title}\PY{p}{(}\PY{n}{f}\PY{l+s+s2}{\PYZdq{}}\PY{l+s+si}{\PYZob{}s1\PYZcb{}}\PY{l+s+s2}{:}\PY{l+s+s2}{\PYZob{}}\PY{l+s+s2}{round(tmom[}\PY{l+s+s2}{\PYZsq{}}\PY{l+s+s2}{eta02}\PY{l+s+s2}{\PYZsq{}}\PY{l+s+s2}{],3)\PYZcb{}, }\PY{l+s+si}{\PYZob{}s2\PYZcb{}}\PY{l+s+s2}{:}\PY{l+s+s2}{\PYZob{}}\PY{l+s+s2}{round(tmom[}\PY{l+s+s2}{\PYZsq{}}\PY{l+s+s2}{eta20}\PY{l+s+s2}{\PYZsq{}}\PY{l+s+s2}{],3)\PYZcb{}}\PY{l+s+s2}{\PYZdq{}}\PY{p}{,} \PY{n}{fontsize}\PY{o}{=}\PY{l+m+mi}{8}\PY{p}{)}
        \PY{n}{plt}\PY{o}{.}\PY{n}{tight\PYZus{}layout}\PY{p}{(}\PY{p}{)}
\end{Verbatim}


    \begin{center}
    \adjustimage{max size={0.9\linewidth}{0.9\paperheight}}{output_6_0.png}
    \end{center}
    { \hspace*{\fill} \\}
    
    (The other moments were also calculated, but were ignored since they
amount to 0).

From the above plot it is clear that the rectangles only have
\(\eta_{02}\) and \(\eta_{20}\) values. This is because of their perfect
symmetry, and ``squarish'' nature that they do not exhibit most other
properties other than the orientation, which is given by the
aforementioned second order moments.

Note that for the 1 rectangle thats oriented transposed of the others,
the values swap, thus telling us that the orientation has itself
transposed.

    \hypertarget{load-up-eigdata.txt-and-perform-background-subtraction}{%
\section{Load up eigdata.txt and perform
background-subtraction}\label{load-up-eigdata.txt-and-perform-background-subtraction}}

    \begin{Verbatim}[commandchars=\\\{\}]
{\color{incolor}In [{\color{incolor}6}]:} \PY{n}{X} \PY{o}{=} \PY{n}{np}\PY{o}{.}\PY{n}{loadtxt}\PY{p}{(}\PY{l+s+s1}{\PYZsq{}}\PY{l+s+s1}{./data/eigdata.txt}\PY{l+s+s1}{\PYZsq{}}\PY{p}{)}
\end{Verbatim}


    \begin{Verbatim}[commandchars=\\\{\}]
{\color{incolor}In [{\color{incolor}7}]:} \PY{n}{plt}\PY{o}{.}\PY{n}{scatter}\PY{p}{(}\PY{n}{X}\PY{p}{[}\PY{p}{:}\PY{p}{,}\PY{l+m+mi}{0}\PY{p}{]}\PY{p}{,}\PY{n}{X}\PY{p}{[}\PY{p}{:}\PY{p}{,}\PY{l+m+mi}{1}\PY{p}{]}\PY{p}{,} \PY{n}{s}\PY{o}{=}\PY{l+m+mf}{0.5}\PY{p}{,} \PY{n}{c}\PY{o}{=}\PY{l+s+s1}{\PYZsq{}}\PY{l+s+s1}{blue}\PY{l+s+s1}{\PYZsq{}}\PY{p}{)}
\end{Verbatim}


\begin{Verbatim}[commandchars=\\\{\}]
{\color{outcolor}Out[{\color{outcolor}7}]:} <matplotlib.collections.PathCollection at 0x1ec3d565b00>
\end{Verbatim}
            
    \begin{center}
    \adjustimage{max size={0.9\linewidth}{0.9\paperheight}}{output_10_1.png}
    \end{center}
    { \hspace*{\fill} \\}
    
    \begin{Verbatim}[commandchars=\\\{\}]
{\color{incolor}In [{\color{incolor}8}]:} \PY{n}{m} \PY{o}{=} \PY{n}{np}\PY{o}{.}\PY{n}{mean}\PY{p}{(}\PY{n}{X}\PY{p}{,} \PY{n}{axis}\PY{o}{=}\PY{l+m+mi}{0}\PY{p}{)}
        
        \PY{n}{Y} \PY{o}{=} \PY{n}{X} \PY{o}{\PYZhy{}} \PY{n}{m}
\end{Verbatim}


    \begin{Verbatim}[commandchars=\\\{\}]
{\color{incolor}In [{\color{incolor}9}]:} \PY{n}{plt}\PY{o}{.}\PY{n}{scatter}\PY{p}{(}\PY{n}{Y}\PY{p}{[}\PY{p}{:}\PY{p}{,}\PY{l+m+mi}{0}\PY{p}{]}\PY{p}{,}\PY{n}{Y}\PY{p}{[}\PY{p}{:}\PY{p}{,}\PY{l+m+mi}{1}\PY{p}{]}\PY{p}{,} \PY{n}{s}\PY{o}{=}\PY{l+m+mf}{0.5}\PY{p}{,} \PY{n}{c}\PY{o}{=}\PY{l+s+s1}{\PYZsq{}}\PY{l+s+s1}{red}\PY{l+s+s1}{\PYZsq{}}\PY{p}{)}
\end{Verbatim}


\begin{Verbatim}[commandchars=\\\{\}]
{\color{outcolor}Out[{\color{outcolor}9}]:} <matplotlib.collections.PathCollection at 0x1ec3d5ea588>
\end{Verbatim}
            
    \begin{center}
    \adjustimage{max size={0.9\linewidth}{0.9\paperheight}}{output_12_1.png}
    \end{center}
    { \hspace*{\fill} \\}
    
    \hypertarget{compute-eigenvalues-and-eigenvectors-of-the-data}{%
\section{Compute Eigenvalues and Eigenvectors of the
data}\label{compute-eigenvalues-and-eigenvectors-of-the-data}}

    First we calculate the covariance matrix

    \begin{Verbatim}[commandchars=\\\{\}]
{\color{incolor}In [{\color{incolor}10}]:} \PY{n}{covariance} \PY{o}{=} \PY{n}{np}\PY{o}{.}\PY{n}{cov}\PY{p}{(}\PY{n}{Y}\PY{o}{.}\PY{n}{T}\PY{p}{)}
         
         \PY{n}{covariance}\PY{o}{.}\PY{n}{shape}
\end{Verbatim}


\begin{Verbatim}[commandchars=\\\{\}]
{\color{outcolor}Out[{\color{outcolor}10}]:} (2, 2)
\end{Verbatim}
            
    Now we calculate the eigenvalues and eigenvectors from the covariance
matrix (\texttt{Note:\ We\ are\ not\ using\ np.linalg.inv(covariance)})
This would mean that the upcoming steps will have a slight difference.

    \begin{Verbatim}[commandchars=\\\{\}]
{\color{incolor}In [{\color{incolor}11}]:} \PY{p}{[}\PY{n}{w}\PY{p}{,} \PY{n}{v}\PY{p}{]} \PY{o}{=} \PY{n}{np}\PY{o}{.}\PY{n}{linalg}\PY{o}{.}\PY{n}{eig}\PY{p}{(}\PY{n}{covariance}\PY{p}{)}
         \PY{n}{largest\PYZus{}eigvec} \PY{o}{=} \PY{n}{v}\PY{p}{[}\PY{l+m+mi}{0}\PY{p}{]}
         \PY{n}{smallest\PYZus{}eigvec} \PY{o}{=} \PY{n}{v}\PY{p}{[}\PY{l+m+mi}{1}\PY{p}{]}
\end{Verbatim}


    Calculating the angle between x-axis and the largest eigenvector, since
we will aim to project our data onto that axis. We also need to ensure
that this angle is between \([0,2\pi]\), though our \texttt{arctan2}
gives us the angle in \([-\pi,\pi]\). Hence we shift it by \(2\pi\) if
negative.

    \begin{Verbatim}[commandchars=\\\{\}]
{\color{incolor}In [{\color{incolor}12}]:} \PY{n}{angle} \PY{o}{=} \PY{n}{np}\PY{o}{.}\PY{n}{arctan2}\PY{p}{(}\PY{n}{largest\PYZus{}eigvec}\PY{p}{[}\PY{l+m+mi}{1}\PY{p}{]}\PY{p}{,} \PY{n}{largest\PYZus{}eigvec}\PY{p}{[}\PY{l+m+mi}{0}\PY{p}{]}\PY{p}{)}
         \PY{k}{if} \PY{n}{angle} \PY{o}{\PYZlt{}} \PY{l+m+mi}{0}\PY{p}{:}
             \PY{n}{angle} \PY{o}{=} \PY{n}{angle} \PY{o}{+} \PY{l+m+mi}{2}\PY{o}{*}\PY{n}{np}\PY{o}{.}\PY{n}{pi}
\end{Verbatim}


    Co-ordinates of our data mean, basically so that we can center our
plots, no real function in terms of our projection and de-correlation.

    \begin{Verbatim}[commandchars=\\\{\}]
{\color{incolor}In [{\color{incolor}13}]:} \PY{n}{avg} \PY{o}{=} \PY{n}{np}\PY{o}{.}\PY{n}{mean}\PY{p}{(}\PY{n}{Y}\PY{p}{,} \PY{n}{axis}\PY{o}{=}\PY{l+m+mi}{0}\PY{p}{)}
\end{Verbatim}


    The \(3\sigma\) std-deviation corresponds to \texttt{0.9973} in terms of
confidence interval, and our corresponding \(\chi^2\) value then becomes
\texttt{7.378}. If we have to center our ellipse around this, we take
sq, root since each \(\sigma_x\) and \(\sigma_y\) requires this
parameter.

    \begin{Verbatim}[commandchars=\\\{\}]
{\color{incolor}In [{\color{incolor}14}]:} \PY{n}{chisquare\PYZus{}val} \PY{o}{=} \PY{n}{np}\PY{o}{.}\PY{n}{sqrt}\PY{p}{(}\PY{l+m+mf}{7.378}\PY{p}{)}
         \PY{n}{theta\PYZus{}grid} \PY{o}{=} \PY{n}{np}\PY{o}{.}\PY{n}{linspace}\PY{p}{(}\PY{l+m+mi}{0}\PY{p}{,} \PY{l+m+mi}{2}\PY{o}{*}\PY{n}{np}\PY{o}{.}\PY{n}{pi}\PY{p}{)}
         \PY{n}{phi} \PY{o}{=} \PY{o}{\PYZhy{}}\PY{n}{angle}
         \PY{n}{X0} \PY{o}{=} \PY{n}{avg}\PY{p}{[}\PY{l+m+mi}{0}\PY{p}{]}
         \PY{n}{Y0} \PY{o}{=} \PY{n}{avg}\PY{p}{[}\PY{l+m+mi}{1}\PY{p}{]}
\end{Verbatim}


    Generating a gridspace for our ellipse

    \begin{Verbatim}[commandchars=\\\{\}]
{\color{incolor}In [{\color{incolor}15}]:} \PY{n}{a}\PY{o}{=}\PY{n}{chisquare\PYZus{}val}\PY{o}{*}\PY{n}{np}\PY{o}{.}\PY{n}{sqrt}\PY{p}{(}\PY{n}{w}\PY{p}{[}\PY{l+m+mi}{0}\PY{p}{]}\PY{p}{)}
         \PY{n}{b}\PY{o}{=}\PY{n}{chisquare\PYZus{}val}\PY{o}{*}\PY{n}{np}\PY{o}{.}\PY{n}{sqrt}\PY{p}{(}\PY{n}{w}\PY{p}{[}\PY{l+m+mi}{1}\PY{p}{]}\PY{p}{)}
         
         \PY{n}{ellipse\PYZus{}x\PYZus{}r}  \PY{o}{=} \PY{n}{a}\PY{o}{*}\PY{n}{np}\PY{o}{.}\PY{n}{cos}\PY{p}{(} \PY{n}{theta\PYZus{}grid} \PY{p}{)}
         \PY{n}{ellipse\PYZus{}y\PYZus{}r}  \PY{o}{=} \PY{n}{b}\PY{o}{*}\PY{n}{np}\PY{o}{.}\PY{n}{sin}\PY{p}{(} \PY{n}{theta\PYZus{}grid} \PY{p}{)}
\end{Verbatim}


    Rotation matrix for our ellipse, so that it corresponds to our data

    \begin{Verbatim}[commandchars=\\\{\}]
{\color{incolor}In [{\color{incolor}16}]:} \PY{n}{R} \PY{o}{=} \PY{n}{np}\PY{o}{.}\PY{n}{array}\PY{p}{(}\PY{p}{[}\PY{p}{[} \PY{n}{np}\PY{o}{.}\PY{n}{cos}\PY{p}{(}\PY{n}{phi}\PY{p}{)}\PY{p}{,} \PY{n}{np}\PY{o}{.}\PY{n}{sin}\PY{p}{(}\PY{n}{phi}\PY{p}{)}\PY{p}{]}\PY{p}{,} \PY{p}{[}\PY{o}{\PYZhy{}}\PY{n}{np}\PY{o}{.}\PY{n}{sin}\PY{p}{(}\PY{n}{phi}\PY{p}{)}\PY{p}{,} \PY{n}{np}\PY{o}{.}\PY{n}{cos}\PY{p}{(}\PY{n}{phi}\PY{p}{)}\PY{p}{]}\PY{p}{]}\PY{p}{)}\PY{o}{.}\PY{n}{reshape}\PY{p}{(}\PY{p}{(}\PY{l+m+mi}{2}\PY{p}{,}\PY{l+m+mi}{2}\PY{p}{)}\PY{p}{)}
\end{Verbatim}


    \begin{Verbatim}[commandchars=\\\{\}]
{\color{incolor}In [{\color{incolor}17}]:} \PY{n}{r\PYZus{}ellipse} \PY{o}{=} \PY{n}{np}\PY{o}{.}\PY{n}{vstack}\PY{p}{(}\PY{p}{(}\PY{n}{ellipse\PYZus{}x\PYZus{}r}\PY{p}{,} \PY{n}{ellipse\PYZus{}y\PYZus{}r}\PY{p}{)}\PY{p}{)}\PY{o}{.}\PY{n}{T} \PY{o}{@} \PY{n}{R}
         
         \PY{n}{plt}\PY{o}{.}\PY{n}{plot}\PY{p}{(}\PY{n}{r\PYZus{}ellipse}\PY{p}{[}\PY{p}{:}\PY{p}{,}\PY{l+m+mi}{0}\PY{p}{]} \PY{o}{+} \PY{n}{X0}\PY{p}{,}\PY{n}{r\PYZus{}ellipse}\PY{p}{[}\PY{p}{:}\PY{p}{,}\PY{l+m+mi}{1}\PY{p}{]} \PY{o}{+} \PY{n}{Y0}\PY{p}{,} \PY{n}{c}\PY{o}{=}\PY{l+s+s1}{\PYZsq{}}\PY{l+s+s1}{\PYZsh{}db2f27}\PY{l+s+s1}{\PYZsq{}}\PY{p}{)}
         \PY{n}{plt}\PY{o}{.}\PY{n}{scatter}\PY{p}{(}\PY{n}{Y}\PY{p}{[}\PY{p}{:}\PY{p}{,}\PY{l+m+mi}{0}\PY{p}{]}\PY{p}{,}\PY{n}{Y}\PY{p}{[}\PY{p}{:}\PY{p}{,}\PY{l+m+mi}{1}\PY{p}{]}\PY{p}{,} \PY{n}{s}\PY{o}{=}\PY{l+m+mf}{0.5}\PY{p}{,} \PY{n}{c}\PY{o}{=}\PY{l+s+s1}{\PYZsq{}}\PY{l+s+s1}{\PYZsh{}008cbc}\PY{l+s+s1}{\PYZsq{}}\PY{p}{)}
         \PY{n}{plt}\PY{o}{.}\PY{n}{quiver}\PY{p}{(}\PY{l+m+mi}{0}\PY{p}{,} \PY{l+m+mi}{0}\PY{p}{,} \PY{o}{\PYZhy{}}\PY{l+m+mf}{1.5}\PY{o}{*}\PY{n}{np}\PY{o}{.}\PY{n}{sqrt}\PY{p}{(}\PY{n}{largest\PYZus{}eigvec}\PY{p}{)}\PY{p}{,} \PY{l+m+mi}{1}\PY{o}{*}\PY{n}{np}\PY{o}{.}\PY{n}{sqrt}\PY{p}{(}\PY{n}{largest\PYZus{}eigvec}\PY{p}{)}\PY{p}{,} \PY{n}{color}\PY{o}{=}\PY{l+s+s1}{\PYZsq{}}\PY{l+s+s1}{\PYZsh{}007500}\PY{l+s+s1}{\PYZsq{}}\PY{p}{,} \PY{n}{scale}\PY{o}{=}\PY{l+m+mf}{3.4}\PY{p}{)}
         \PY{n}{plt}\PY{o}{.}\PY{n}{quiver}\PY{p}{(}\PY{l+m+mi}{0}\PY{p}{,} \PY{l+m+mi}{0}\PY{p}{,} \PY{l+m+mi}{1}\PY{o}{*}\PY{n}{np}\PY{o}{.}\PY{n}{sqrt}\PY{p}{(}\PY{n}{smallest\PYZus{}eigvec}\PY{p}{)}\PY{p}{,} \PY{l+m+mf}{1.5}\PY{o}{*}\PY{n}{np}\PY{o}{.}\PY{n}{sqrt}\PY{p}{(}\PY{n}{smallest\PYZus{}eigvec}\PY{p}{)}\PY{p}{,} \PY{n}{color}\PY{o}{=}\PY{l+s+s1}{\PYZsq{}}\PY{l+s+s1}{\PYZsh{}0b3954}\PY{l+s+s1}{\PYZsq{}}\PY{p}{,} \PY{n}{scale}\PY{o}{=}\PY{l+m+mi}{15}\PY{p}{)}
\end{Verbatim}


    \begin{Verbatim}[commandchars=\\\{\}]
c:\textbackslash{}users\textbackslash{}theanig\textbackslash{}appdata\textbackslash{}local\textbackslash{}programs\textbackslash{}python\textbackslash{}python36\textbackslash{}lib\textbackslash{}site-packages\textbackslash{}ipykernel\_launcher.py:6: RuntimeWarning: invalid value encountered in sqrt
  

    \end{Verbatim}

\begin{Verbatim}[commandchars=\\\{\}]
{\color{outcolor}Out[{\color{outcolor}17}]:} <matplotlib.quiver.Quiver at 0x1ec3d8a31d0>
\end{Verbatim}
            
    \begin{center}
    \adjustimage{max size={0.9\linewidth}{0.9\paperheight}}{output_28_2.png}
    \end{center}
    { \hspace*{\fill} \\}
    
    In the above plot we can see ellipse at \(3\sigma\) around the data,
along with the two principal axes, given by the eigenvectors.

    \hypertarget{make-the-data-un-correlated}{%
\section{Make the data
un-correlated}\label{make-the-data-un-correlated}}

    In order to de-correlate the data, we rotate the given data about its
axes such that it aligns with our largest eigenvector.

    \begin{Verbatim}[commandchars=\\\{\}]
{\color{incolor}In [{\color{incolor}18}]:} \PY{n}{stddev} \PY{o}{=} \PY{n}{np}\PY{o}{.}\PY{n}{sqrt}\PY{p}{(}\PY{n}{np}\PY{o}{.}\PY{n}{diag}\PY{p}{(}\PY{n}{covariance}\PY{p}{)}\PY{p}{)}
\end{Verbatim}


    \begin{Verbatim}[commandchars=\\\{\}]
{\color{incolor}In [{\color{incolor}19}]:} \PY{n}{decorrelated\PYZus{}y} \PY{o}{=} \PY{n}{np}\PY{o}{.}\PY{n}{divide}\PY{p}{(}\PY{n}{Y}\PY{p}{,} \PY{n}{stddev}\PY{p}{)}
         \PY{n}{decorrelated\PYZus{}y} \PY{o}{=} \PY{n}{decorrelated\PYZus{}y} \PY{o}{@} \PY{n}{v}
         
         \PY{n}{plt}\PY{o}{.}\PY{n}{plot}\PY{p}{(}\PY{n}{ellipse\PYZus{}x\PYZus{}r}\PY{o}{/}\PY{l+m+mi}{4}\PY{p}{,} \PY{n}{ellipse\PYZus{}y\PYZus{}r}\PY{o}{/}\PY{l+m+mi}{4}\PY{p}{,} \PY{n}{c}\PY{o}{=}\PY{l+s+s1}{\PYZsq{}}\PY{l+s+s1}{\PYZsh{}db2f27}\PY{l+s+s1}{\PYZsq{}}\PY{p}{)}
         \PY{n}{plt}\PY{o}{.}\PY{n}{scatter}\PY{p}{(}\PY{n}{decorrelated\PYZus{}y}\PY{p}{[}\PY{p}{:}\PY{p}{,}\PY{l+m+mi}{0}\PY{p}{]}\PY{p}{,}\PY{n}{decorrelated\PYZus{}y}\PY{p}{[}\PY{p}{:}\PY{p}{,}\PY{l+m+mi}{1}\PY{p}{]}\PY{p}{,} \PY{n}{s}\PY{o}{=}\PY{l+m+mf}{0.5}\PY{p}{,} \PY{n}{c}\PY{o}{=}\PY{l+s+s1}{\PYZsq{}}\PY{l+s+s1}{\PYZsh{}008cbc}\PY{l+s+s1}{\PYZsq{}}\PY{p}{)}
         \PY{n}{plt}\PY{o}{.}\PY{n}{quiver}\PY{p}{(}\PY{l+m+mi}{0}\PY{p}{,} \PY{l+m+mi}{0}\PY{p}{,} \PY{l+m+mi}{1}\PY{o}{*}\PY{n}{np}\PY{o}{.}\PY{n}{sqrt}\PY{p}{(}\PY{n}{largest\PYZus{}eigvec}\PY{p}{)}\PY{p}{,} \PY{l+m+mi}{0}\PY{o}{*}\PY{n}{np}\PY{o}{.}\PY{n}{sqrt}\PY{p}{(}\PY{n}{largest\PYZus{}eigvec}\PY{p}{)}\PY{p}{,} \PY{n}{color}\PY{o}{=}\PY{l+s+s1}{\PYZsq{}}\PY{l+s+s1}{\PYZsh{}007500}\PY{l+s+s1}{\PYZsq{}}\PY{p}{,} \PY{n}{scale}\PY{o}{=}\PY{l+m+mf}{2.3}\PY{p}{)}
         \PY{n}{plt}\PY{o}{.}\PY{n}{quiver}\PY{p}{(}\PY{l+m+mi}{0}\PY{p}{,} \PY{l+m+mi}{0}\PY{p}{,} \PY{l+m+mi}{0}\PY{o}{*}\PY{n}{np}\PY{o}{.}\PY{n}{sqrt}\PY{p}{(}\PY{n}{smallest\PYZus{}eigvec}\PY{p}{)}\PY{p}{,} \PY{l+m+mi}{1}\PY{o}{*}\PY{n}{np}\PY{o}{.}\PY{n}{sqrt}\PY{p}{(}\PY{n}{smallest\PYZus{}eigvec}\PY{p}{)}\PY{p}{,} \PY{n}{color}\PY{o}{=}\PY{l+s+s1}{\PYZsq{}}\PY{l+s+s1}{\PYZsh{}0b3954}\PY{l+s+s1}{\PYZsq{}}\PY{p}{,} \PY{n}{scale}\PY{o}{=}\PY{l+m+mf}{4.5}\PY{p}{)}
\end{Verbatim}


    \begin{Verbatim}[commandchars=\\\{\}]
c:\textbackslash{}users\textbackslash{}theanig\textbackslash{}appdata\textbackslash{}local\textbackslash{}programs\textbackslash{}python\textbackslash{}python36\textbackslash{}lib\textbackslash{}site-packages\textbackslash{}ipykernel\_launcher.py:7: RuntimeWarning: invalid value encountered in sqrt
  import sys

    \end{Verbatim}

\begin{Verbatim}[commandchars=\\\{\}]
{\color{outcolor}Out[{\color{outcolor}19}]:} <matplotlib.quiver.Quiver at 0x1ec3d9003c8>
\end{Verbatim}
            
    \begin{center}
    \adjustimage{max size={0.9\linewidth}{0.9\paperheight}}{output_33_2.png}
    \end{center}
    { \hspace*{\fill} \\}
    
    Note that in the above figure, our squished ellipse from previous
question appears to have fattened a little bit, this is merely an
illusion generated due to use of different scaling along the other axis
(done explicitly by me to make the results look better), if we were to
measure the thickness empirically, we would find the two results to be
the same.

    \begin{Verbatim}[commandchars=\\\{\}]
{\color{incolor}In [{\color{incolor}20}]:} \PY{n}{projected\PYZus{}y} \PY{o}{=} \PY{n}{Y} \PY{o}{@} \PY{n}{largest\PYZus{}eigvec}\PY{o}{.}\PY{n}{T}
         \PY{n}{plt}\PY{o}{.}\PY{n}{scatter}\PY{p}{(}\PY{n}{projected\PYZus{}y}\PY{p}{,} \PY{n}{np}\PY{o}{.}\PY{n}{zeros\PYZus{}like}\PY{p}{(}\PY{n}{projected\PYZus{}y}\PY{p}{)}\PY{p}{,} \PY{n}{c}\PY{o}{=}\PY{l+s+s1}{\PYZsq{}}\PY{l+s+s1}{blue}\PY{l+s+s1}{\PYZsq{}}\PY{p}{)}
\end{Verbatim}


\begin{Verbatim}[commandchars=\\\{\}]
{\color{outcolor}Out[{\color{outcolor}20}]:} <matplotlib.collections.PathCollection at 0x1ec3d94edd8>
\end{Verbatim}
            
    \begin{center}
    \adjustimage{max size={0.9\linewidth}{0.9\paperheight}}{output_35_1.png}
    \end{center}
    { \hspace*{\fill} \\}
    
    Now we project the data down to 1D along the largest eigenvector. Note
that in this particular case, there was some information spread about
the other axis, which was lost, but not significant enough that we are
unable to utilize the results.

    At such a low dimension, PCA performs just as good as any, but if the
given data were of higher dimensionality, we could've had some
interesting comparision with more sophisticated dimensionality reduction
algorithms like t-SNE and compared the two.


    % Add a bibliography block to the postdoc
    
    
    
    \end{document}
