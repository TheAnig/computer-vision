
% Default to the notebook output style

    


% Inherit from the specified cell style.




    
\documentclass[11pt]{article}

    
    
    \usepackage[T1]{fontenc}
    % Nicer default font (+ math font) than Computer Modern for most use cases
    \usepackage{mathpazo}

    % Basic figure setup, for now with no caption control since it's done
    % automatically by Pandoc (which extracts ![](path) syntax from Markdown).
    \usepackage{graphicx}
    % We will generate all images so they have a width \maxwidth. This means
    % that they will get their normal width if they fit onto the page, but
    % are scaled down if they would overflow the margins.
    \makeatletter
    \def\maxwidth{\ifdim\Gin@nat@width>\linewidth\linewidth
    \else\Gin@nat@width\fi}
    \makeatother
    \let\Oldincludegraphics\includegraphics
    % Set max figure width to be 80% of text width, for now hardcoded.
    \renewcommand{\includegraphics}[1]{\Oldincludegraphics[width=.8\maxwidth]{#1}}
    % Ensure that by default, figures have no caption (until we provide a
    % proper Figure object with a Caption API and a way to capture that
    % in the conversion process - todo).
    \usepackage{caption}
    \DeclareCaptionLabelFormat{nolabel}{}
    \captionsetup{labelformat=nolabel}

    \usepackage{adjustbox} % Used to constrain images to a maximum size 
    \usepackage{xcolor} % Allow colors to be defined
    \usepackage{enumerate} % Needed for markdown enumerations to work
    \usepackage{geometry} % Used to adjust the document margins
    \usepackage{amsmath} % Equations
    \usepackage{amssymb} % Equations
    \usepackage{textcomp} % defines textquotesingle
    % Hack from http://tex.stackexchange.com/a/47451/13684:
    \AtBeginDocument{%
        \def\PYZsq{\textquotesingle}% Upright quotes in Pygmentized code
    }
    \usepackage{upquote} % Upright quotes for verbatim code
    \usepackage{eurosym} % defines \euro
    \usepackage[mathletters]{ucs} % Extended unicode (utf-8) support
    \usepackage[utf8x]{inputenc} % Allow utf-8 characters in the tex document
    \usepackage{fancyvrb} % verbatim replacement that allows latex
    \usepackage{grffile} % extends the file name processing of package graphics 
                         % to support a larger range 
    % The hyperref package gives us a pdf with properly built
    % internal navigation ('pdf bookmarks' for the table of contents,
    % internal cross-reference links, web links for URLs, etc.)
    \usepackage{hyperref}
    \usepackage{longtable} % longtable support required by pandoc >1.10
    \usepackage{booktabs}  % table support for pandoc > 1.12.2
    \usepackage[inline]{enumitem} % IRkernel/repr support (it uses the enumerate* environment)
    \usepackage[normalem]{ulem} % ulem is needed to support strikethroughs (\sout)
                                % normalem makes italics be italics, not underlines
    

    
    
    % Colors for the hyperref package
    \definecolor{urlcolor}{rgb}{0,.145,.698}
    \definecolor{linkcolor}{rgb}{.71,0.21,0.01}
    \definecolor{citecolor}{rgb}{.12,.54,.11}

    % ANSI colors
    \definecolor{ansi-black}{HTML}{3E424D}
    \definecolor{ansi-black-intense}{HTML}{282C36}
    \definecolor{ansi-red}{HTML}{E75C58}
    \definecolor{ansi-red-intense}{HTML}{B22B31}
    \definecolor{ansi-green}{HTML}{00A250}
    \definecolor{ansi-green-intense}{HTML}{007427}
    \definecolor{ansi-yellow}{HTML}{DDB62B}
    \definecolor{ansi-yellow-intense}{HTML}{B27D12}
    \definecolor{ansi-blue}{HTML}{208FFB}
    \definecolor{ansi-blue-intense}{HTML}{0065CA}
    \definecolor{ansi-magenta}{HTML}{D160C4}
    \definecolor{ansi-magenta-intense}{HTML}{A03196}
    \definecolor{ansi-cyan}{HTML}{60C6C8}
    \definecolor{ansi-cyan-intense}{HTML}{258F8F}
    \definecolor{ansi-white}{HTML}{C5C1B4}
    \definecolor{ansi-white-intense}{HTML}{A1A6B2}

    % commands and environments needed by pandoc snippets
    % extracted from the output of `pandoc -s`
    \providecommand{\tightlist}{%
      \setlength{\itemsep}{0pt}\setlength{\parskip}{0pt}}
    \DefineVerbatimEnvironment{Highlighting}{Verbatim}{commandchars=\\\{\}}
    % Add ',fontsize=\small' for more characters per line
    \newenvironment{Shaded}{}{}
    \newcommand{\KeywordTok}[1]{\textcolor[rgb]{0.00,0.44,0.13}{\textbf{{#1}}}}
    \newcommand{\DataTypeTok}[1]{\textcolor[rgb]{0.56,0.13,0.00}{{#1}}}
    \newcommand{\DecValTok}[1]{\textcolor[rgb]{0.25,0.63,0.44}{{#1}}}
    \newcommand{\BaseNTok}[1]{\textcolor[rgb]{0.25,0.63,0.44}{{#1}}}
    \newcommand{\FloatTok}[1]{\textcolor[rgb]{0.25,0.63,0.44}{{#1}}}
    \newcommand{\CharTok}[1]{\textcolor[rgb]{0.25,0.44,0.63}{{#1}}}
    \newcommand{\StringTok}[1]{\textcolor[rgb]{0.25,0.44,0.63}{{#1}}}
    \newcommand{\CommentTok}[1]{\textcolor[rgb]{0.38,0.63,0.69}{\textit{{#1}}}}
    \newcommand{\OtherTok}[1]{\textcolor[rgb]{0.00,0.44,0.13}{{#1}}}
    \newcommand{\AlertTok}[1]{\textcolor[rgb]{1.00,0.00,0.00}{\textbf{{#1}}}}
    \newcommand{\FunctionTok}[1]{\textcolor[rgb]{0.02,0.16,0.49}{{#1}}}
    \newcommand{\RegionMarkerTok}[1]{{#1}}
    \newcommand{\ErrorTok}[1]{\textcolor[rgb]{1.00,0.00,0.00}{\textbf{{#1}}}}
    \newcommand{\NormalTok}[1]{{#1}}
    
    % Additional commands for more recent versions of Pandoc
    \newcommand{\ConstantTok}[1]{\textcolor[rgb]{0.53,0.00,0.00}{{#1}}}
    \newcommand{\SpecialCharTok}[1]{\textcolor[rgb]{0.25,0.44,0.63}{{#1}}}
    \newcommand{\VerbatimStringTok}[1]{\textcolor[rgb]{0.25,0.44,0.63}{{#1}}}
    \newcommand{\SpecialStringTok}[1]{\textcolor[rgb]{0.73,0.40,0.53}{{#1}}}
    \newcommand{\ImportTok}[1]{{#1}}
    \newcommand{\DocumentationTok}[1]{\textcolor[rgb]{0.73,0.13,0.13}{\textit{{#1}}}}
    \newcommand{\AnnotationTok}[1]{\textcolor[rgb]{0.38,0.63,0.69}{\textbf{\textit{{#1}}}}}
    \newcommand{\CommentVarTok}[1]{\textcolor[rgb]{0.38,0.63,0.69}{\textbf{\textit{{#1}}}}}
    \newcommand{\VariableTok}[1]{\textcolor[rgb]{0.10,0.09,0.49}{{#1}}}
    \newcommand{\ControlFlowTok}[1]{\textcolor[rgb]{0.00,0.44,0.13}{\textbf{{#1}}}}
    \newcommand{\OperatorTok}[1]{\textcolor[rgb]{0.40,0.40,0.40}{{#1}}}
    \newcommand{\BuiltInTok}[1]{{#1}}
    \newcommand{\ExtensionTok}[1]{{#1}}
    \newcommand{\PreprocessorTok}[1]{\textcolor[rgb]{0.74,0.48,0.00}{{#1}}}
    \newcommand{\AttributeTok}[1]{\textcolor[rgb]{0.49,0.56,0.16}{{#1}}}
    \newcommand{\InformationTok}[1]{\textcolor[rgb]{0.38,0.63,0.69}{\textbf{\textit{{#1}}}}}
    \newcommand{\WarningTok}[1]{\textcolor[rgb]{0.38,0.63,0.69}{\textbf{\textit{{#1}}}}}
    
    
    % Define a nice break command that doesn't care if a line doesn't already
    % exist.
    \def\br{\hspace*{\fill} \\* }
    % Math Jax compatability definitions
    \def\gt{>}
    \def\lt{<}
    % Document parameters
    \title{hw2-soln}
    
    
    

    % Pygments definitions
    
\makeatletter
\def\PY@reset{\let\PY@it=\relax \let\PY@bf=\relax%
    \let\PY@ul=\relax \let\PY@tc=\relax%
    \let\PY@bc=\relax \let\PY@ff=\relax}
\def\PY@tok#1{\csname PY@tok@#1\endcsname}
\def\PY@toks#1+{\ifx\relax#1\empty\else%
    \PY@tok{#1}\expandafter\PY@toks\fi}
\def\PY@do#1{\PY@bc{\PY@tc{\PY@ul{%
    \PY@it{\PY@bf{\PY@ff{#1}}}}}}}
\def\PY#1#2{\PY@reset\PY@toks#1+\relax+\PY@do{#2}}

\expandafter\def\csname PY@tok@w\endcsname{\def\PY@tc##1{\textcolor[rgb]{0.73,0.73,0.73}{##1}}}
\expandafter\def\csname PY@tok@c\endcsname{\let\PY@it=\textit\def\PY@tc##1{\textcolor[rgb]{0.25,0.50,0.50}{##1}}}
\expandafter\def\csname PY@tok@cp\endcsname{\def\PY@tc##1{\textcolor[rgb]{0.74,0.48,0.00}{##1}}}
\expandafter\def\csname PY@tok@k\endcsname{\let\PY@bf=\textbf\def\PY@tc##1{\textcolor[rgb]{0.00,0.50,0.00}{##1}}}
\expandafter\def\csname PY@tok@kp\endcsname{\def\PY@tc##1{\textcolor[rgb]{0.00,0.50,0.00}{##1}}}
\expandafter\def\csname PY@tok@kt\endcsname{\def\PY@tc##1{\textcolor[rgb]{0.69,0.00,0.25}{##1}}}
\expandafter\def\csname PY@tok@o\endcsname{\def\PY@tc##1{\textcolor[rgb]{0.40,0.40,0.40}{##1}}}
\expandafter\def\csname PY@tok@ow\endcsname{\let\PY@bf=\textbf\def\PY@tc##1{\textcolor[rgb]{0.67,0.13,1.00}{##1}}}
\expandafter\def\csname PY@tok@nb\endcsname{\def\PY@tc##1{\textcolor[rgb]{0.00,0.50,0.00}{##1}}}
\expandafter\def\csname PY@tok@nf\endcsname{\def\PY@tc##1{\textcolor[rgb]{0.00,0.00,1.00}{##1}}}
\expandafter\def\csname PY@tok@nc\endcsname{\let\PY@bf=\textbf\def\PY@tc##1{\textcolor[rgb]{0.00,0.00,1.00}{##1}}}
\expandafter\def\csname PY@tok@nn\endcsname{\let\PY@bf=\textbf\def\PY@tc##1{\textcolor[rgb]{0.00,0.00,1.00}{##1}}}
\expandafter\def\csname PY@tok@ne\endcsname{\let\PY@bf=\textbf\def\PY@tc##1{\textcolor[rgb]{0.82,0.25,0.23}{##1}}}
\expandafter\def\csname PY@tok@nv\endcsname{\def\PY@tc##1{\textcolor[rgb]{0.10,0.09,0.49}{##1}}}
\expandafter\def\csname PY@tok@no\endcsname{\def\PY@tc##1{\textcolor[rgb]{0.53,0.00,0.00}{##1}}}
\expandafter\def\csname PY@tok@nl\endcsname{\def\PY@tc##1{\textcolor[rgb]{0.63,0.63,0.00}{##1}}}
\expandafter\def\csname PY@tok@ni\endcsname{\let\PY@bf=\textbf\def\PY@tc##1{\textcolor[rgb]{0.60,0.60,0.60}{##1}}}
\expandafter\def\csname PY@tok@na\endcsname{\def\PY@tc##1{\textcolor[rgb]{0.49,0.56,0.16}{##1}}}
\expandafter\def\csname PY@tok@nt\endcsname{\let\PY@bf=\textbf\def\PY@tc##1{\textcolor[rgb]{0.00,0.50,0.00}{##1}}}
\expandafter\def\csname PY@tok@nd\endcsname{\def\PY@tc##1{\textcolor[rgb]{0.67,0.13,1.00}{##1}}}
\expandafter\def\csname PY@tok@s\endcsname{\def\PY@tc##1{\textcolor[rgb]{0.73,0.13,0.13}{##1}}}
\expandafter\def\csname PY@tok@sd\endcsname{\let\PY@it=\textit\def\PY@tc##1{\textcolor[rgb]{0.73,0.13,0.13}{##1}}}
\expandafter\def\csname PY@tok@si\endcsname{\let\PY@bf=\textbf\def\PY@tc##1{\textcolor[rgb]{0.73,0.40,0.53}{##1}}}
\expandafter\def\csname PY@tok@se\endcsname{\let\PY@bf=\textbf\def\PY@tc##1{\textcolor[rgb]{0.73,0.40,0.13}{##1}}}
\expandafter\def\csname PY@tok@sr\endcsname{\def\PY@tc##1{\textcolor[rgb]{0.73,0.40,0.53}{##1}}}
\expandafter\def\csname PY@tok@ss\endcsname{\def\PY@tc##1{\textcolor[rgb]{0.10,0.09,0.49}{##1}}}
\expandafter\def\csname PY@tok@sx\endcsname{\def\PY@tc##1{\textcolor[rgb]{0.00,0.50,0.00}{##1}}}
\expandafter\def\csname PY@tok@m\endcsname{\def\PY@tc##1{\textcolor[rgb]{0.40,0.40,0.40}{##1}}}
\expandafter\def\csname PY@tok@gh\endcsname{\let\PY@bf=\textbf\def\PY@tc##1{\textcolor[rgb]{0.00,0.00,0.50}{##1}}}
\expandafter\def\csname PY@tok@gu\endcsname{\let\PY@bf=\textbf\def\PY@tc##1{\textcolor[rgb]{0.50,0.00,0.50}{##1}}}
\expandafter\def\csname PY@tok@gd\endcsname{\def\PY@tc##1{\textcolor[rgb]{0.63,0.00,0.00}{##1}}}
\expandafter\def\csname PY@tok@gi\endcsname{\def\PY@tc##1{\textcolor[rgb]{0.00,0.63,0.00}{##1}}}
\expandafter\def\csname PY@tok@gr\endcsname{\def\PY@tc##1{\textcolor[rgb]{1.00,0.00,0.00}{##1}}}
\expandafter\def\csname PY@tok@ge\endcsname{\let\PY@it=\textit}
\expandafter\def\csname PY@tok@gs\endcsname{\let\PY@bf=\textbf}
\expandafter\def\csname PY@tok@gp\endcsname{\let\PY@bf=\textbf\def\PY@tc##1{\textcolor[rgb]{0.00,0.00,0.50}{##1}}}
\expandafter\def\csname PY@tok@go\endcsname{\def\PY@tc##1{\textcolor[rgb]{0.53,0.53,0.53}{##1}}}
\expandafter\def\csname PY@tok@gt\endcsname{\def\PY@tc##1{\textcolor[rgb]{0.00,0.27,0.87}{##1}}}
\expandafter\def\csname PY@tok@err\endcsname{\def\PY@bc##1{\setlength{\fboxsep}{0pt}\fcolorbox[rgb]{1.00,0.00,0.00}{1,1,1}{\strut ##1}}}
\expandafter\def\csname PY@tok@kc\endcsname{\let\PY@bf=\textbf\def\PY@tc##1{\textcolor[rgb]{0.00,0.50,0.00}{##1}}}
\expandafter\def\csname PY@tok@kd\endcsname{\let\PY@bf=\textbf\def\PY@tc##1{\textcolor[rgb]{0.00,0.50,0.00}{##1}}}
\expandafter\def\csname PY@tok@kn\endcsname{\let\PY@bf=\textbf\def\PY@tc##1{\textcolor[rgb]{0.00,0.50,0.00}{##1}}}
\expandafter\def\csname PY@tok@kr\endcsname{\let\PY@bf=\textbf\def\PY@tc##1{\textcolor[rgb]{0.00,0.50,0.00}{##1}}}
\expandafter\def\csname PY@tok@bp\endcsname{\def\PY@tc##1{\textcolor[rgb]{0.00,0.50,0.00}{##1}}}
\expandafter\def\csname PY@tok@fm\endcsname{\def\PY@tc##1{\textcolor[rgb]{0.00,0.00,1.00}{##1}}}
\expandafter\def\csname PY@tok@vc\endcsname{\def\PY@tc##1{\textcolor[rgb]{0.10,0.09,0.49}{##1}}}
\expandafter\def\csname PY@tok@vg\endcsname{\def\PY@tc##1{\textcolor[rgb]{0.10,0.09,0.49}{##1}}}
\expandafter\def\csname PY@tok@vi\endcsname{\def\PY@tc##1{\textcolor[rgb]{0.10,0.09,0.49}{##1}}}
\expandafter\def\csname PY@tok@vm\endcsname{\def\PY@tc##1{\textcolor[rgb]{0.10,0.09,0.49}{##1}}}
\expandafter\def\csname PY@tok@sa\endcsname{\def\PY@tc##1{\textcolor[rgb]{0.73,0.13,0.13}{##1}}}
\expandafter\def\csname PY@tok@sb\endcsname{\def\PY@tc##1{\textcolor[rgb]{0.73,0.13,0.13}{##1}}}
\expandafter\def\csname PY@tok@sc\endcsname{\def\PY@tc##1{\textcolor[rgb]{0.73,0.13,0.13}{##1}}}
\expandafter\def\csname PY@tok@dl\endcsname{\def\PY@tc##1{\textcolor[rgb]{0.73,0.13,0.13}{##1}}}
\expandafter\def\csname PY@tok@s2\endcsname{\def\PY@tc##1{\textcolor[rgb]{0.73,0.13,0.13}{##1}}}
\expandafter\def\csname PY@tok@sh\endcsname{\def\PY@tc##1{\textcolor[rgb]{0.73,0.13,0.13}{##1}}}
\expandafter\def\csname PY@tok@s1\endcsname{\def\PY@tc##1{\textcolor[rgb]{0.73,0.13,0.13}{##1}}}
\expandafter\def\csname PY@tok@mb\endcsname{\def\PY@tc##1{\textcolor[rgb]{0.40,0.40,0.40}{##1}}}
\expandafter\def\csname PY@tok@mf\endcsname{\def\PY@tc##1{\textcolor[rgb]{0.40,0.40,0.40}{##1}}}
\expandafter\def\csname PY@tok@mh\endcsname{\def\PY@tc##1{\textcolor[rgb]{0.40,0.40,0.40}{##1}}}
\expandafter\def\csname PY@tok@mi\endcsname{\def\PY@tc##1{\textcolor[rgb]{0.40,0.40,0.40}{##1}}}
\expandafter\def\csname PY@tok@il\endcsname{\def\PY@tc##1{\textcolor[rgb]{0.40,0.40,0.40}{##1}}}
\expandafter\def\csname PY@tok@mo\endcsname{\def\PY@tc##1{\textcolor[rgb]{0.40,0.40,0.40}{##1}}}
\expandafter\def\csname PY@tok@ch\endcsname{\let\PY@it=\textit\def\PY@tc##1{\textcolor[rgb]{0.25,0.50,0.50}{##1}}}
\expandafter\def\csname PY@tok@cm\endcsname{\let\PY@it=\textit\def\PY@tc##1{\textcolor[rgb]{0.25,0.50,0.50}{##1}}}
\expandafter\def\csname PY@tok@cpf\endcsname{\let\PY@it=\textit\def\PY@tc##1{\textcolor[rgb]{0.25,0.50,0.50}{##1}}}
\expandafter\def\csname PY@tok@c1\endcsname{\let\PY@it=\textit\def\PY@tc##1{\textcolor[rgb]{0.25,0.50,0.50}{##1}}}
\expandafter\def\csname PY@tok@cs\endcsname{\let\PY@it=\textit\def\PY@tc##1{\textcolor[rgb]{0.25,0.50,0.50}{##1}}}

\def\PYZbs{\char`\\}
\def\PYZus{\char`\_}
\def\PYZob{\char`\{}
\def\PYZcb{\char`\}}
\def\PYZca{\char`\^}
\def\PYZam{\char`\&}
\def\PYZlt{\char`\<}
\def\PYZgt{\char`\>}
\def\PYZsh{\char`\#}
\def\PYZpc{\char`\%}
\def\PYZdl{\char`\$}
\def\PYZhy{\char`\-}
\def\PYZsq{\char`\'}
\def\PYZdq{\char`\"}
\def\PYZti{\char`\~}
% for compatibility with earlier versions
\def\PYZat{@}
\def\PYZlb{[}
\def\PYZrb{]}
\makeatother


    % Exact colors from NB
    \definecolor{incolor}{rgb}{0.0, 0.0, 0.5}
    \definecolor{outcolor}{rgb}{0.545, 0.0, 0.0}



    
    % Prevent overflowing lines due to hard-to-break entities
    \sloppy 
    % Setup hyperref package
    \hypersetup{
      breaklinks=true,  % so long urls are correctly broken across lines
      colorlinks=true,
      urlcolor=urlcolor,
      linkcolor=linkcolor,
      citecolor=citecolor,
      }
    % Slightly bigger margins than the latex defaults
    
    \geometry{verbose,tmargin=1in,bmargin=1in,lmargin=1in,rmargin=1in}
    
    

    \begin{document}
    
    
    \noindent
\large\textbf{Homework Assignment 2} \hfill \textbf{Anirudh Ganesh} \\
\normalsize Computer Vision for HCI \hfill CSE5524 (Au `18) \\
Prof. Jim Davis \hfill Score: \_\_\_/15 \\
TA: Sayan Mandal \hfill Due Date: 09/04/18
    
    

    
    \hypertarget{importing-libraries}{%
\paragraph{Importing Libraries}\label{importing-libraries}}

    \begin{Verbatim}[commandchars=\\\{\}]
{\color{incolor}In [{\color{incolor}1}]:} \PY{k+kn}{from} \PY{n+nn}{skimage}\PY{n+nn}{.}\PY{n+nn}{io} \PY{k}{import} \PY{n}{imread}
        \PY{k+kn}{from} \PY{n+nn}{skimage}\PY{n+nn}{.}\PY{n+nn}{filters} \PY{k}{import} \PY{n}{gaussian}
        \PY{k+kn}{import} \PY{n+nn}{numpy} \PY{k}{as} \PY{n+nn}{np}
        \PY{k+kn}{from} \PY{n+nn}{matplotlib} \PY{k}{import} \PY{n}{pyplot} \PY{k}{as} \PY{n}{plt}
        \PY{k+kn}{import} \PY{n+nn}{math}
\end{Verbatim}


    \hypertarget{load-teds-face}{%
\paragraph{Load Ted's face}\label{load-teds-face}}

    \begin{Verbatim}[commandchars=\\\{\}]
{\color{incolor}In [{\color{incolor}2}]:} \PY{n}{keanu} \PY{o}{=} \PY{n}{imread}\PY{p}{(}\PY{l+s+s1}{\PYZsq{}}\PY{l+s+s1}{keanu.jpg}\PY{l+s+s1}{\PYZsq{}}\PY{p}{)}
\end{Verbatim}


    \hypertarget{create-a-list-of-sigmas-to-iterate-over}{%
\paragraph{Create a list of sigma's to iterate
over}\label{create-a-list-of-sigmas-to-iterate-over}}

    \begin{Verbatim}[commandchars=\\\{\}]
{\color{incolor}In [{\color{incolor}3}]:} \PY{n}{sigmas} \PY{o}{=} \PY{p}{[}\PY{l+m+mi}{20}\PY{p}{,} \PY{l+m+mi}{15}\PY{p}{,} \PY{l+m+mi}{12}\PY{p}{,} \PY{l+m+mi}{10}\PY{p}{,} \PY{l+m+mi}{1}\PY{p}{,} \PY{l+m+mf}{0.5}\PY{p}{]}
\end{Verbatim}


    Note: In the included document for the homework assignment,
\texttt{fspecial()} for MATLAB took size of gaussian kernel as an
argument (hsize), hence it was necessary to use the formula given in the
slides \(radius = 2*ceil(sigma)+1\). As of current versions of MATLAB
\texttt{fspecial()} has been deprecated for \texttt{imgaussfilt()} which
automatically implements filter size as given
\href{https://www.mathworks.com/help/images/ref/imgaussfilt.html}{here}.
Similarly, the equivalent function in Python that I'm using
\texttt{skimage.filters.gaussian()} is a wrapper around
\texttt{scipy.ndi.gaussian()} which implements the Gaussian filter
similarly. Hence, there is no need to pass the second argument.
\href{https://github.com/scipy/scipy/blob/v0.14.0/scipy/ndimage/filters.py\#L250}{Source
for Python function}

    \hypertarget{perform-gaussian-smoothing-over-keanus-face}{%
\section{Perform Gaussian smoothing over Keanu's
face}\label{perform-gaussian-smoothing-over-keanus-face}}

    \begin{Verbatim}[commandchars=\\\{\}]
{\color{incolor}In [{\color{incolor}4}]:} \PY{n}{plt}\PY{o}{.}\PY{n}{close}\PY{p}{(}\PY{l+s+s1}{\PYZsq{}}\PY{l+s+s1}{all}\PY{l+s+s1}{\PYZsq{}}\PY{p}{)}
        \PY{n}{f}\PY{p}{,} \PY{n}{axarr} \PY{o}{=} \PY{n}{plt}\PY{o}{.}\PY{n}{subplots}\PY{p}{(}\PY{l+m+mi}{2}\PY{p}{,} \PY{l+m+mi}{3}\PY{p}{,} \PY{n}{sharex}\PY{o}{=}\PY{l+s+s1}{\PYZsq{}}\PY{l+s+s1}{col}\PY{l+s+s1}{\PYZsq{}}\PY{p}{,} \PY{n}{sharey}\PY{o}{=}\PY{l+s+s1}{\PYZsq{}}\PY{l+s+s1}{row}\PY{l+s+s1}{\PYZsq{}}\PY{p}{,} \PY{n}{dpi}\PY{o}{=}\PY{l+m+mi}{200}\PY{p}{)}
        \PY{k}{for} \PY{n}{sigma} \PY{o+ow}{in} \PY{n}{sigmas}\PY{p}{:}
            \PY{n}{filteredIm} \PY{o}{=} \PY{n}{gaussian}\PY{p}{(}\PY{n}{keanu}\PY{p}{,} \PY{n}{sigma}\PY{p}{)}
            \PY{n}{idx} \PY{o}{=} \PY{n}{sigmas}\PY{o}{.}\PY{n}{index}\PY{p}{(}\PY{n}{sigma}\PY{p}{)}
            \PY{c+c1}{\PYZsh{}print((int(idx/3), idx\PYZpc{}3))}
            \PY{n}{axarr}\PY{p}{[}\PY{n+nb}{int}\PY{p}{(}\PY{n}{idx}\PY{o}{/}\PY{l+m+mi}{3}\PY{p}{)}\PY{p}{,} \PY{n}{idx}\PY{o}{\PYZpc{}}\PY{k}{3}].axis(\PYZsq{}off\PYZsq{})
            \PY{n}{axarr}\PY{p}{[}\PY{n+nb}{int}\PY{p}{(}\PY{n}{idx}\PY{o}{/}\PY{l+m+mi}{3}\PY{p}{)}\PY{p}{,} \PY{n}{idx}\PY{o}{\PYZpc{}}\PY{k}{3}].set\PYZus{}title(f\PYZsq{}\PYZdl{}\PYZbs{}sigma\PYZdl{} = \PYZob{}sigma\PYZcb{}\PYZsq{})
            \PY{n}{axarr}\PY{p}{[}\PY{n+nb}{int}\PY{p}{(}\PY{n}{idx}\PY{o}{/}\PY{l+m+mi}{3}\PY{p}{)}\PY{p}{,} \PY{n}{idx}\PY{o}{\PYZpc{}}\PY{k}{3}].imshow(filteredIm, cmap = \PYZsq{}gray\PYZsq{}, aspect=\PYZsq{}auto\PYZsq{})
\end{Verbatim}


    \begin{center}
    \adjustimage{max size={0.9\linewidth}{0.9\paperheight}}{output_8_0.png}
    \end{center}
    { \hspace*{\fill} \\}
    
    Notice that the list of \(\sigma\)s used in the above example doesn't
follow any linear or even log-linear scale. This is because I have
chosen these values after some trial and error with my friends in order
of cognitive difficulty in terms of identifying Keanu Reeves.
Interesting observation I found was \(\sigma=10\) was the point of most
controversy, as this seems like the value where some of my friends were
able to identify but others were clueless. Also interesting to note is
that \(\sigma\) values less than 5 seem to have no effect on anyone.

    \hypertarget{function-to-compute-and-display-2d-gaussian-derivative-masks}{%
\section{Function to compute and display 2D Gaussian Derivative
masks}\label{function-to-compute-and-display-2d-gaussian-derivative-masks}}

    From the class notes, we know that,

\[G_x = \frac{-x}{2\pi\sigma^4}e^{-\frac{x^2 + y^2}{2\sigma}}\]
\[G_y = \frac{-y}{2\pi\sigma^4}e^{-\frac{x^2 + y^2}{2\sigma}}\]

    From the above set of equations and comparing with instructions
presented in our homework document, we can implement a naive derivative
filter to be used with \texttt{sklearn.ndi.generic\_filter}.

\begin{verbatim}
def gaussX(sigma):
    l = 2 * sigma + 1
    ax = np.arange(-l, l)
    xx, yy = np.meshgrid(ax, ax)

    kernel = -(xx * np.exp(-(xx**2 + yy**2) / (2. * sigma**2)))/ (2*math.pi* (sigma**4))

    kernel = kernel / np.sum(kernel)

    return kernel

def gaussY(sigma):
    l = 2 * sigma + 1
    ax = np.arange(-l, l)
    xx, yy = np.meshgrid(ax, ax)

    kernel = -(yy * np.exp(-(xx**2 + yy**2) / (2. * sigma**2)))/ (2*math.pi* (sigma**4))

    kernel = kernel / np.sum(kernel)

    return kernel
\end{verbatim}

The reason why I haven't done this is because, while its relatively easy
to implement the filter itself this way, due to the way skimage
processes, applying the filter becomes a pain point in terms of applying
it programmatically.

    Instead I choose to apply the same derivative using a more efficient
method (drawn from our class discussion of being able to split a
gaussian filter into two steps of 1D filters).

This is calculated by creating a base gaussian kernel given by
\(k = \frac{1}{2\pi\sigma}\sum e^{\frac{x^2}{2\sigma}}\) then
multiplying, \(t = \frac{x}{\sigma^2}\), thus we get our \(G_x\) and
\(G_y\) respectively.

Implementing it this way, also allows us to transpose our \(G_x\) to get
\(G_y\) and thus provides for a succint implementation of the required
function

    \begin{Verbatim}[commandchars=\\\{\}]
{\color{incolor}In [{\color{incolor}5}]:} \PY{k+kn}{from} \PY{n+nn}{scipy}\PY{n+nn}{.}\PY{n+nn}{ndimage}\PY{n+nn}{.}\PY{n+nn}{filters} \PY{k}{import} \PY{n}{correlate1d}
        \PY{k+kn}{from} \PY{n+nn}{scipy}\PY{n+nn}{.}\PY{n+nn}{ndimage}\PY{n+nn}{.}\PY{n+nn}{filters} \PY{k}{import} \PY{n}{\PYZus{}ni\PYZus{}support}
        \PY{k+kn}{from} \PY{n+nn}{skimage}\PY{n+nn}{.}\PY{n+nn}{color} \PY{k}{import} \PY{n}{guess\PYZus{}spatial\PYZus{}dimensions}
        \PY{k+kn}{from} \PY{n+nn}{skimage}\PY{n+nn}{.}\PY{n+nn}{\PYZus{}shared}\PY{n+nn}{.}\PY{n+nn}{utils} \PY{k}{import} \PY{n}{convert\PYZus{}to\PYZus{}float}
        \PY{k+kn}{from} \PY{n+nn}{scipy} \PY{k}{import} \PY{n}{ndimage} \PY{k}{as} \PY{n}{ndi}
        
        \PY{k}{def} \PY{n+nf}{gaussian\PYZus{}derivative\PYZus{}filter1d}\PY{p}{(}\PY{n+nb}{input}\PY{p}{,} \PY{n}{sigma}\PY{p}{,} \PY{n}{axis}\PY{o}{=}\PY{o}{\PYZhy{}}\PY{l+m+mi}{1}\PY{p}{,} \PY{n}{output}\PY{o}{=}\PY{k+kc}{None}\PY{p}{,}\PY{n}{mode}\PY{o}{=}\PY{l+s+s2}{\PYZdq{}}\PY{l+s+s2}{reflect}\PY{l+s+s2}{\PYZdq{}}\PY{p}{,} \PY{n}{cval}\PY{o}{=}\PY{l+m+mf}{0.0}\PY{p}{,} \PY{n}{truncate}\PY{o}{=}\PY{l+m+mf}{4.0}\PY{p}{)}\PY{p}{:}
            
            \PY{n}{sd} \PY{o}{=} \PY{n+nb}{float}\PY{p}{(}\PY{n}{sigma}\PY{p}{)}
            \PY{c+c1}{\PYZsh{} make the radius of the filter equal to truncate standard deviations}
            \PY{n}{lw} \PY{o}{=} \PY{n+nb}{int}\PY{p}{(}\PY{n}{truncate} \PY{o}{*} \PY{n}{sd} \PY{o}{+} \PY{l+m+mf}{0.5}\PY{p}{)}
            \PY{n}{weights} \PY{o}{=} \PY{p}{[}\PY{l+m+mf}{0.0}\PY{p}{]} \PY{o}{*} \PY{p}{(}\PY{l+m+mi}{2} \PY{o}{*} \PY{n}{lw} \PY{o}{+} \PY{l+m+mi}{1}\PY{p}{)}
            \PY{n}{weights}\PY{p}{[}\PY{n}{lw}\PY{p}{]} \PY{o}{=} \PY{l+m+mf}{1.0}
            \PY{n+nb}{sum} \PY{o}{=} \PY{l+m+mf}{1.0}
            \PY{n}{sd} \PY{o}{=} \PY{n}{sd} \PY{o}{*} \PY{n}{sd}
            \PY{c+c1}{\PYZsh{} calculate the kernel:}
            \PY{k}{for} \PY{n}{ii} \PY{o+ow}{in} \PY{n+nb}{range}\PY{p}{(}\PY{l+m+mi}{1}\PY{p}{,} \PY{n}{lw} \PY{o}{+} \PY{l+m+mi}{1}\PY{p}{)}\PY{p}{:}
                \PY{n}{tmp} \PY{o}{=} \PY{n}{math}\PY{o}{.}\PY{n}{exp}\PY{p}{(}\PY{o}{\PYZhy{}}\PY{l+m+mf}{0.5} \PY{o}{*} \PY{n+nb}{float}\PY{p}{(}\PY{n}{ii} \PY{o}{*} \PY{n}{ii}\PY{p}{)} \PY{o}{/} \PY{n}{sd}\PY{p}{)}
                \PY{n}{weights}\PY{p}{[}\PY{n}{lw} \PY{o}{+} \PY{n}{ii}\PY{p}{]} \PY{o}{=} \PY{n}{tmp}
                \PY{n}{weights}\PY{p}{[}\PY{n}{lw} \PY{o}{\PYZhy{}} \PY{n}{ii}\PY{p}{]} \PY{o}{=} \PY{n}{tmp}
                \PY{n+nb}{sum} \PY{o}{+}\PY{o}{=} \PY{l+m+mf}{2.0} \PY{o}{*} \PY{n}{tmp}
            \PY{k}{for} \PY{n}{ii} \PY{o+ow}{in} \PY{n+nb}{range}\PY{p}{(}\PY{l+m+mi}{2} \PY{o}{*} \PY{n}{lw} \PY{o}{+} \PY{l+m+mi}{1}\PY{p}{)}\PY{p}{:}
                \PY{n}{weights}\PY{p}{[}\PY{n}{ii}\PY{p}{]} \PY{o}{/}\PY{o}{=} \PY{n+nb}{sum}
            \PY{c+c1}{\PYZsh{} implement first order derivative:}
            \PY{n}{weights}\PY{p}{[}\PY{n}{lw}\PY{p}{]} \PY{o}{=} \PY{l+m+mf}{0.0}
            \PY{k}{for} \PY{n}{ii} \PY{o+ow}{in} \PY{n+nb}{range}\PY{p}{(}\PY{l+m+mi}{1}\PY{p}{,} \PY{n}{lw} \PY{o}{+} \PY{l+m+mi}{1}\PY{p}{)}\PY{p}{:}
                \PY{n}{x} \PY{o}{=} \PY{n+nb}{float}\PY{p}{(}\PY{n}{ii}\PY{p}{)}
                \PY{n}{tmp} \PY{o}{=} \PY{o}{\PYZhy{}}\PY{n}{x} \PY{o}{/} \PY{n}{sd} \PY{o}{*} \PY{n}{weights}\PY{p}{[}\PY{n}{lw} \PY{o}{+} \PY{n}{ii}\PY{p}{]}
                \PY{n}{weights}\PY{p}{[}\PY{n}{lw} \PY{o}{+} \PY{n}{ii}\PY{p}{]} \PY{o}{=} \PY{o}{\PYZhy{}}\PY{n}{tmp}
                \PY{n}{weights}\PY{p}{[}\PY{n}{lw} \PY{o}{\PYZhy{}} \PY{n}{ii}\PY{p}{]} \PY{o}{=} \PY{n}{tmp}
            \PY{k}{return} \PY{n}{correlate1d}\PY{p}{(}\PY{n+nb}{input}\PY{p}{,} \PY{n}{weights}\PY{p}{,} \PY{n}{axis}\PY{p}{,} \PY{n}{output}\PY{p}{,} \PY{n}{mode}\PY{p}{,} \PY{n}{cval}\PY{p}{,} \PY{l+m+mi}{0}\PY{p}{)}
        
        \PY{k}{def} \PY{n+nf}{gaussDerive2D}\PY{p}{(}\PY{n+nb}{input}\PY{p}{,} \PY{n}{sigma}\PY{p}{,} \PY{n}{output}\PY{o}{=}\PY{k+kc}{None}\PY{p}{)}\PY{p}{:}
            \PY{n+nb}{input} \PY{o}{=} \PY{n}{np}\PY{o}{.}\PY{n}{asarray}\PY{p}{(}\PY{n+nb}{input}\PY{p}{)}
            \PY{n}{output} \PY{o}{=} \PY{n}{\PYZus{}ni\PYZus{}support}\PY{o}{.}\PY{n}{\PYZus{}get\PYZus{}output}\PY{p}{(}\PY{k+kc}{None}\PY{p}{,} \PY{n+nb}{input}\PY{p}{)}
            \PY{n}{sigmas} \PY{o}{=} \PY{n}{\PYZus{}ni\PYZus{}support}\PY{o}{.}\PY{n}{\PYZus{}normalize\PYZus{}sequence}\PY{p}{(}\PY{n}{sigma}\PY{p}{,} \PY{n+nb}{input}\PY{o}{.}\PY{n}{ndim}\PY{p}{)}
            \PY{n}{axes} \PY{o}{=} \PY{n+nb}{list}\PY{p}{(}\PY{n+nb}{range}\PY{p}{(}\PY{n+nb}{input}\PY{o}{.}\PY{n}{ndim}\PY{p}{)}\PY{p}{)}
            \PY{n}{axes} \PY{o}{=} \PY{p}{[}\PY{p}{(}\PY{n}{axes}\PY{p}{[}\PY{n}{ii}\PY{p}{]}\PY{p}{,} \PY{n}{sigmas}\PY{p}{[}\PY{n}{ii}\PY{p}{]}\PY{p}{)}
                                \PY{k}{for} \PY{n}{ii} \PY{o+ow}{in} \PY{n+nb}{range}\PY{p}{(}\PY{n+nb}{len}\PY{p}{(}\PY{n}{axes}\PY{p}{)}\PY{p}{)} \PY{k}{if} \PY{n}{sigmas}\PY{p}{[}\PY{n}{ii}\PY{p}{]} \PY{o}{\PYZgt{}} \PY{l+m+mf}{1e\PYZhy{}15}\PY{p}{]}
            \PY{k}{if} \PY{n+nb}{len}\PY{p}{(}\PY{n}{axes}\PY{p}{)} \PY{o}{\PYZgt{}} \PY{l+m+mi}{0}\PY{p}{:}
                \PY{k}{for} \PY{n}{axis}\PY{p}{,} \PY{n}{sigma} \PY{o+ow}{in} \PY{n}{axes}\PY{p}{:}
                    \PY{n}{gaussian\PYZus{}derivative\PYZus{}filter1d}\PY{p}{(}\PY{n+nb}{input}\PY{p}{,} \PY{n}{sigma}\PY{p}{,} \PY{n}{axis}\PY{p}{,} \PY{n}{output}\PY{p}{)}
                    \PY{n+nb}{input} \PY{o}{=} \PY{n}{output}
            \PY{k}{else}\PY{p}{:}
                \PY{n}{output}\PY{p}{[}\PY{o}{.}\PY{o}{.}\PY{o}{.}\PY{p}{]} \PY{o}{=} \PY{n+nb}{input}\PY{p}{[}\PY{o}{.}\PY{o}{.}\PY{o}{.}\PY{p}{]}
            \PY{k}{return} \PY{n}{output}
        
        \PY{k}{def} \PY{n+nf}{gaussDeriveApplyFilter}\PY{p}{(}\PY{n}{image}\PY{p}{,} \PY{n}{sigma}\PY{o}{=}\PY{l+m+mi}{1}\PY{p}{,} \PY{n}{output}\PY{o}{=}\PY{k+kc}{None}\PY{p}{,} \PY{n}{mode}\PY{o}{=}\PY{l+s+s1}{\PYZsq{}}\PY{l+s+s1}{nearest}\PY{l+s+s1}{\PYZsq{}}\PY{p}{,} \PY{n}{cval}\PY{o}{=}\PY{l+m+mi}{0}\PY{p}{,}\PY{n}{multichannel}\PY{o}{=}\PY{k+kc}{None}\PY{p}{,} \PY{n}{preserve\PYZus{}range}\PY{o}{=}\PY{k+kc}{False}\PY{p}{,} \PY{n}{truncate}\PY{o}{=}\PY{l+m+mf}{4.0}\PY{p}{)}\PY{p}{:}
            \PY{n}{spatial\PYZus{}dims} \PY{o}{=} \PY{n}{guess\PYZus{}spatial\PYZus{}dimensions}\PY{p}{(}\PY{n}{image}\PY{p}{)}
            \PY{k}{if} \PY{n}{np}\PY{o}{.}\PY{n}{any}\PY{p}{(}\PY{n}{np}\PY{o}{.}\PY{n}{asarray}\PY{p}{(}\PY{n}{sigma}\PY{p}{)} \PY{o}{\PYZlt{}} \PY{l+m+mf}{0.0}\PY{p}{)}\PY{p}{:}
                \PY{k}{raise} \PY{n+ne}{ValueError}\PY{p}{(}\PY{l+s+s2}{\PYZdq{}}\PY{l+s+s2}{Sigma values less than zero are not valid}\PY{l+s+s2}{\PYZdq{}}\PY{p}{)}
            \PY{k}{if} \PY{n}{multichannel}\PY{p}{:}
                \PY{c+c1}{\PYZsh{} do not filter across channels}
                \PY{k}{if} \PY{o+ow}{not} \PY{n+nb}{isinstance}\PY{p}{(}\PY{n}{sigma}\PY{p}{,} \PY{n}{coll}\PY{o}{.}\PY{n}{Iterable}\PY{p}{)}\PY{p}{:}
                    \PY{n}{sigma} \PY{o}{=} \PY{p}{[}\PY{n}{sigma}\PY{p}{]} \PY{o}{*} \PY{p}{(}\PY{n}{image}\PY{o}{.}\PY{n}{ndim} \PY{o}{\PYZhy{}} \PY{l+m+mi}{1}\PY{p}{)}
                \PY{k}{if} \PY{n+nb}{len}\PY{p}{(}\PY{n}{sigma}\PY{p}{)} \PY{o}{!=} \PY{n}{image}\PY{o}{.}\PY{n}{ndim}\PY{p}{:}
                    \PY{n}{sigma} \PY{o}{=} \PY{n}{np}\PY{o}{.}\PY{n}{concatenate}\PY{p}{(}\PY{p}{(}\PY{n}{np}\PY{o}{.}\PY{n}{asarray}\PY{p}{(}\PY{n}{sigma}\PY{p}{)}\PY{p}{,} \PY{p}{[}\PY{l+m+mi}{0}\PY{p}{]}\PY{p}{)}\PY{p}{)}
            \PY{n}{image} \PY{o}{=} \PY{n}{convert\PYZus{}to\PYZus{}float}\PY{p}{(}\PY{n}{image}\PY{p}{,} \PY{n}{preserve\PYZus{}range}\PY{p}{)}
            \PY{k}{return} \PY{n}{gaussDerive2D}\PY{p}{(}\PY{n}{image}\PY{p}{,} \PY{n}{sigma}\PY{p}{)}
\end{Verbatim}


    \hypertarget{compute-and-display-the-effect-of-our-filter}{%
\section{Compute and Display the effect of our
filter}\label{compute-and-display-the-effect-of-our-filter}}

    \begin{Verbatim}[commandchars=\\\{\}]
{\color{incolor}In [{\color{incolor}6}]:} \PY{k+kn}{from} \PY{n+nn}{sklearn}\PY{n+nn}{.}\PY{n+nn}{preprocessing} \PY{k}{import} \PY{n}{normalize}
        
        \PY{n}{sigmas} \PY{o}{=} \PY{p}{[}\PY{l+m+mi}{20}\PY{p}{,} \PY{l+m+mi}{10}\PY{p}{,} \PY{l+m+mi}{5}\PY{p}{,} \PY{l+m+mi}{3}\PY{p}{,} \PY{l+m+mi}{1}\PY{p}{,} \PY{l+m+mf}{0.5}\PY{p}{]}
        
        \PY{n}{f}\PY{p}{,} \PY{n}{axarr} \PY{o}{=} \PY{n}{plt}\PY{o}{.}\PY{n}{subplots}\PY{p}{(}\PY{l+m+mi}{2}\PY{p}{,} \PY{l+m+mi}{3}\PY{p}{,} \PY{n}{sharex}\PY{o}{=}\PY{l+s+s1}{\PYZsq{}}\PY{l+s+s1}{col}\PY{l+s+s1}{\PYZsq{}}\PY{p}{,} \PY{n}{sharey}\PY{o}{=}\PY{l+s+s1}{\PYZsq{}}\PY{l+s+s1}{row}\PY{l+s+s1}{\PYZsq{}}\PY{p}{,} \PY{n}{dpi}\PY{o}{=}\PY{l+m+mi}{200}\PY{p}{)}
        \PY{k}{for} \PY{n}{sigma} \PY{o+ow}{in} \PY{n}{sigmas}\PY{p}{:}
            \PY{n}{magIm} \PY{o}{=} \PY{n}{gaussDeriveApplyFilter}\PY{p}{(}\PY{n}{keanu}\PY{p}{,} \PY{n}{sigma}\PY{p}{)}
            \PY{n}{magIm} \PY{o}{=} \PY{n}{normalize}\PY{p}{(}\PY{n}{magIm}\PY{p}{,} \PY{n}{axis}\PY{o}{=}\PY{l+m+mi}{0}\PY{p}{,} \PY{n}{norm}\PY{o}{=}\PY{l+s+s1}{\PYZsq{}}\PY{l+s+s1}{max}\PY{l+s+s1}{\PYZsq{}}\PY{p}{)} \PY{o}{*} \PY{l+m+mf}{255.0}
            \PY{n}{idx} \PY{o}{=} \PY{n}{sigmas}\PY{o}{.}\PY{n}{index}\PY{p}{(}\PY{n}{sigma}\PY{p}{)}
            \PY{c+c1}{\PYZsh{}print((int(idx/3), idx\PYZpc{}3))}
            \PY{n}{axarr}\PY{p}{[}\PY{n+nb}{int}\PY{p}{(}\PY{n}{idx}\PY{o}{/}\PY{l+m+mi}{3}\PY{p}{)}\PY{p}{,} \PY{n}{idx}\PY{o}{\PYZpc{}}\PY{k}{3}].axis(\PYZsq{}off\PYZsq{})
            \PY{n}{axarr}\PY{p}{[}\PY{n+nb}{int}\PY{p}{(}\PY{n}{idx}\PY{o}{/}\PY{l+m+mi}{3}\PY{p}{)}\PY{p}{,} \PY{n}{idx}\PY{o}{\PYZpc{}}\PY{k}{3}].set\PYZus{}title(f\PYZsq{}\PYZdl{}\PYZbs{}sigma\PYZdl{} = \PYZob{}sigma\PYZcb{}\PYZsq{})
            \PY{n}{axarr}\PY{p}{[}\PY{n+nb}{int}\PY{p}{(}\PY{n}{idx}\PY{o}{/}\PY{l+m+mi}{3}\PY{p}{)}\PY{p}{,} \PY{n}{idx}\PY{o}{\PYZpc{}}\PY{k}{3}].imshow(magIm, cmap = \PYZsq{}gray\PYZsq{}, aspect=\PYZsq{}auto\PYZsq{})
\end{Verbatim}


    \begin{center}
    \adjustimage{max size={0.9\linewidth}{0.9\paperheight}}{output_16_0.png}
    \end{center}
    { \hspace*{\fill} \\}
    
    Note: Due to conversion of float to int while displaying, the
\texttt{cmap=\textquotesingle{}gray\textquotesingle{}} acts a little
wonky, but functionally there is little difference

    \hypertarget{threshold-with-different-t-levels}{%
\section{Threshold with different T
levels}\label{threshold-with-different-t-levels}}

    \begin{Verbatim}[commandchars=\\\{\}]
{\color{incolor}In [{\color{incolor}7}]:} \PY{n}{magIm} \PY{o}{=} \PY{n}{gaussDeriveApplyFilter}\PY{p}{(}\PY{n}{keanu}\PY{p}{,} \PY{n}{sigma} \PY{o}{=} \PY{l+m+mi}{3}\PY{p}{)}
        
        \PY{c+c1}{\PYZsh{}magIm = (magIm \PYZhy{} magIm.min(0)) *255.0 / magIm.ptp(0)}
        
        \PY{n}{magIm} \PY{o}{=} \PY{n}{normalize}\PY{p}{(}\PY{n}{magIm}\PY{p}{,} \PY{n}{axis}\PY{o}{=}\PY{l+m+mi}{0}\PY{p}{,} \PY{n}{norm}\PY{o}{=}\PY{l+s+s1}{\PYZsq{}}\PY{l+s+s1}{max}\PY{l+s+s1}{\PYZsq{}}\PY{p}{)} \PY{o}{*} \PY{l+m+mf}{255.0}
\end{Verbatim}


    \begin{Verbatim}[commandchars=\\\{\}]
{\color{incolor}In [{\color{incolor}8}]:} \PY{n}{threshLevels} \PY{o}{=} \PY{p}{[}\PY{l+m+mi}{25}\PY{p}{,} \PY{l+m+mi}{50}\PY{p}{,} \PY{l+m+mi}{75}\PY{p}{,} \PY{l+m+mi}{125}\PY{p}{,} \PY{l+m+mi}{175}\PY{p}{,} \PY{l+m+mi}{225}\PY{p}{]}
        \PY{n}{f}\PY{p}{,} \PY{n}{axarr} \PY{o}{=} \PY{n}{plt}\PY{o}{.}\PY{n}{subplots}\PY{p}{(}\PY{l+m+mi}{2}\PY{p}{,} \PY{l+m+mi}{3}\PY{p}{,} \PY{n}{sharex}\PY{o}{=}\PY{l+s+s1}{\PYZsq{}}\PY{l+s+s1}{col}\PY{l+s+s1}{\PYZsq{}}\PY{p}{,} \PY{n}{sharey}\PY{o}{=}\PY{l+s+s1}{\PYZsq{}}\PY{l+s+s1}{row}\PY{l+s+s1}{\PYZsq{}}\PY{p}{,} \PY{n}{dpi}\PY{o}{=}\PY{l+m+mi}{200}\PY{p}{)}
        \PY{k}{for} \PY{n}{thresh} \PY{o+ow}{in} \PY{n}{threshLevels}\PY{p}{:}
            \PY{n}{tImg} \PY{o}{=} \PY{n}{magIm} \PY{o}{\PYZgt{}} \PY{n}{thresh}
            \PY{n}{idx} \PY{o}{=} \PY{n}{threshLevels}\PY{o}{.}\PY{n}{index}\PY{p}{(}\PY{n}{thresh}\PY{p}{)}
            \PY{c+c1}{\PYZsh{}print((int(idx/3), idx\PYZpc{}3))}
            \PY{n}{axarr}\PY{p}{[}\PY{n+nb}{int}\PY{p}{(}\PY{n}{idx}\PY{o}{/}\PY{l+m+mi}{3}\PY{p}{)}\PY{p}{,} \PY{n}{idx}\PY{o}{\PYZpc{}}\PY{k}{3}].axis(\PYZsq{}off\PYZsq{})
            \PY{n}{axarr}\PY{p}{[}\PY{n+nb}{int}\PY{p}{(}\PY{n}{idx}\PY{o}{/}\PY{l+m+mi}{3}\PY{p}{)}\PY{p}{,} \PY{n}{idx}\PY{o}{\PYZpc{}}\PY{k}{3}].set\PYZus{}title(f\PYZsq{}Value = \PYZob{}thresh\PYZcb{}\PYZsq{})
            \PY{n}{axarr}\PY{p}{[}\PY{n+nb}{int}\PY{p}{(}\PY{n}{idx}\PY{o}{/}\PY{l+m+mi}{3}\PY{p}{)}\PY{p}{,} \PY{n}{idx}\PY{o}{\PYZpc{}}\PY{k}{3}].imshow(tImg, cmap = \PYZsq{}gray\PYZsq{}, aspect=\PYZsq{}auto\PYZsq{})
\end{Verbatim}


    \begin{center}
    \adjustimage{max size={0.9\linewidth}{0.9\paperheight}}{output_20_0.png}
    \end{center}
    { \hspace*{\fill} \\}
    
    We notice that our native implementation, when thresholded properly (
value of 50) generates a semi-decent edge detector. But due to the
calculation of gaussian manually and what seems to be precision round
off errors, we can notice some grainy-ness in the final result. (There
are some specks of activation on the forehead for example). Later we
would observe that Sobel operator gives a similar result without the
grainyness because of (what I assume is) the kernel using whole numbers.

    \hypertarget{compare-with-sobel-masks}{%
\section{Compare with Sobel masks}\label{compare-with-sobel-masks}}

    \begin{Verbatim}[commandchars=\\\{\}]
{\color{incolor}In [{\color{incolor}9}]:} \PY{k+kn}{from} \PY{n+nn}{skimage}\PY{n+nn}{.}\PY{n+nn}{filters} \PY{k}{import} \PY{n}{sobel}
        
        \PY{n}{edges1} \PY{o}{=} \PY{n}{sobel}\PY{p}{(}\PY{n}{keanu}\PY{p}{)}
        \PY{n}{edges2} \PY{o}{=} \PY{n}{sobel}\PY{p}{(}\PY{n}{gaussian}\PY{p}{(}\PY{n}{keanu}\PY{p}{,} \PY{n}{sigma}\PY{o}{=}\PY{l+m+mi}{3}\PY{p}{)}\PY{p}{)}
        
        \PY{c+c1}{\PYZsh{} display results}
        \PY{n}{fig}\PY{p}{,} \PY{p}{(}\PY{n}{ax1}\PY{p}{,} \PY{n}{ax2}\PY{p}{,} \PY{n}{ax3}\PY{p}{)} \PY{o}{=} \PY{n}{plt}\PY{o}{.}\PY{n}{subplots}\PY{p}{(}\PY{n}{nrows}\PY{o}{=}\PY{l+m+mi}{1}\PY{p}{,} \PY{n}{ncols}\PY{o}{=}\PY{l+m+mi}{3}\PY{p}{,} \PY{n}{dpi}\PY{o}{=}\PY{l+m+mi}{200}\PY{p}{,}
                                            \PY{n}{sharex}\PY{o}{=}\PY{k+kc}{True}\PY{p}{,} \PY{n}{sharey}\PY{o}{=}\PY{k+kc}{True}\PY{p}{)}
        
        \PY{n}{ax1}\PY{o}{.}\PY{n}{imshow}\PY{p}{(}\PY{n}{keanu}\PY{p}{,} \PY{n}{cmap}\PY{o}{=}\PY{n}{plt}\PY{o}{.}\PY{n}{cm}\PY{o}{.}\PY{n}{gray}\PY{p}{)}
        \PY{n}{ax1}\PY{o}{.}\PY{n}{axis}\PY{p}{(}\PY{l+s+s1}{\PYZsq{}}\PY{l+s+s1}{off}\PY{l+s+s1}{\PYZsq{}}\PY{p}{)}
        \PY{n}{ax1}\PY{o}{.}\PY{n}{set\PYZus{}title}\PY{p}{(}\PY{l+s+s1}{\PYZsq{}}\PY{l+s+s1}{Original Image}\PY{l+s+s1}{\PYZsq{}}\PY{p}{)}
        
        \PY{n}{ax2}\PY{o}{.}\PY{n}{imshow}\PY{p}{(}\PY{n}{edges1}\PY{p}{,} \PY{n}{cmap}\PY{o}{=}\PY{n}{plt}\PY{o}{.}\PY{n}{cm}\PY{o}{.}\PY{n}{gray}\PY{p}{)}
        \PY{n}{ax2}\PY{o}{.}\PY{n}{axis}\PY{p}{(}\PY{l+s+s1}{\PYZsq{}}\PY{l+s+s1}{off}\PY{l+s+s1}{\PYZsq{}}\PY{p}{)}
        \PY{n}{ax2}\PY{o}{.}\PY{n}{set\PYZus{}title}\PY{p}{(}\PY{l+s+s1}{\PYZsq{}}\PY{l+s+s1}{Sobel filter, no blur}\PY{l+s+s1}{\PYZsq{}}\PY{p}{)}
        
        \PY{n}{ax3}\PY{o}{.}\PY{n}{imshow}\PY{p}{(}\PY{n}{edges2}\PY{p}{,} \PY{n}{cmap}\PY{o}{=}\PY{n}{plt}\PY{o}{.}\PY{n}{cm}\PY{o}{.}\PY{n}{gray}\PY{p}{)}
        \PY{n}{ax3}\PY{o}{.}\PY{n}{axis}\PY{p}{(}\PY{l+s+s1}{\PYZsq{}}\PY{l+s+s1}{off}\PY{l+s+s1}{\PYZsq{}}\PY{p}{)}
        \PY{n}{ax3}\PY{o}{.}\PY{n}{set\PYZus{}title}\PY{p}{(}\PY{l+s+s1}{\PYZsq{}}\PY{l+s+s1}{Sobel filter, \PYZdl{}}\PY{l+s+s1}{\PYZbs{}}\PY{l+s+s1}{sigma=3\PYZdl{}}\PY{l+s+s1}{\PYZsq{}}\PY{p}{)}
        
        \PY{n}{plt}\PY{o}{.}\PY{n}{show}\PY{p}{(}\PY{p}{)}
\end{Verbatim}


    \begin{center}
    \adjustimage{max size={0.9\linewidth}{0.9\paperheight}}{output_23_0.png}
    \end{center}
    { \hspace*{\fill} \\}
    
    From the above image we can see that the sobel filter generates a more
fine and smooth lines for the edges compared to the gaussian derivative
filter that we created. In order to account for the difference in
smoothing, if we try to blur the image before applying Sobel's filter,
we would intuitively expect the lines to become more continuous, but
upon doing so we see that the blurring only makes the edges more
prominent and doesn't improve the overall continuity of the edges.

    \hypertarget{canny-edge-detector}{%
\section{Canny Edge detector}\label{canny-edge-detector}}

    \begin{Verbatim}[commandchars=\\\{\}]
{\color{incolor}In [{\color{incolor}10}]:} \PY{k+kn}{from} \PY{n+nn}{scipy} \PY{k}{import} \PY{n}{ndimage} \PY{k}{as} \PY{n}{ndi}
         
         \PY{k+kn}{from} \PY{n+nn}{skimage} \PY{k}{import} \PY{n}{feature}
         
         \PY{n}{edges1} \PY{o}{=} \PY{n}{feature}\PY{o}{.}\PY{n}{canny}\PY{p}{(}\PY{n}{keanu}\PY{p}{)}
         \PY{n}{edges2} \PY{o}{=} \PY{n}{feature}\PY{o}{.}\PY{n}{canny}\PY{p}{(}\PY{n}{keanu}\PY{p}{,} \PY{n}{sigma}\PY{o}{=}\PY{l+m+mi}{3}\PY{p}{)}
         
         \PY{c+c1}{\PYZsh{} display results}
         \PY{n}{fig}\PY{p}{,} \PY{p}{(}\PY{n}{ax1}\PY{p}{,} \PY{n}{ax2}\PY{p}{,} \PY{n}{ax3}\PY{p}{)} \PY{o}{=} \PY{n}{plt}\PY{o}{.}\PY{n}{subplots}\PY{p}{(}\PY{n}{nrows}\PY{o}{=}\PY{l+m+mi}{1}\PY{p}{,} \PY{n}{ncols}\PY{o}{=}\PY{l+m+mi}{3}\PY{p}{,} \PY{n}{dpi}\PY{o}{=}\PY{l+m+mi}{200}\PY{p}{,}
                                             \PY{n}{sharex}\PY{o}{=}\PY{k+kc}{True}\PY{p}{,} \PY{n}{sharey}\PY{o}{=}\PY{k+kc}{True}\PY{p}{)}
         
         \PY{n}{ax1}\PY{o}{.}\PY{n}{imshow}\PY{p}{(}\PY{n}{keanu}\PY{p}{,} \PY{n}{cmap}\PY{o}{=}\PY{n}{plt}\PY{o}{.}\PY{n}{cm}\PY{o}{.}\PY{n}{gray}\PY{p}{)}
         \PY{n}{ax1}\PY{o}{.}\PY{n}{axis}\PY{p}{(}\PY{l+s+s1}{\PYZsq{}}\PY{l+s+s1}{off}\PY{l+s+s1}{\PYZsq{}}\PY{p}{)}
         \PY{n}{ax1}\PY{o}{.}\PY{n}{set\PYZus{}title}\PY{p}{(}\PY{l+s+s1}{\PYZsq{}}\PY{l+s+s1}{Original Image}\PY{l+s+s1}{\PYZsq{}}\PY{p}{)}
         
         \PY{n}{ax2}\PY{o}{.}\PY{n}{imshow}\PY{p}{(}\PY{n}{edges1}\PY{p}{,} \PY{n}{cmap}\PY{o}{=}\PY{n}{plt}\PY{o}{.}\PY{n}{cm}\PY{o}{.}\PY{n}{gray}\PY{p}{)}
         \PY{n}{ax2}\PY{o}{.}\PY{n}{axis}\PY{p}{(}\PY{l+s+s1}{\PYZsq{}}\PY{l+s+s1}{off}\PY{l+s+s1}{\PYZsq{}}\PY{p}{)}
         \PY{n}{ax2}\PY{o}{.}\PY{n}{set\PYZus{}title}\PY{p}{(}\PY{l+s+s1}{\PYZsq{}}\PY{l+s+s1}{Canny filter, \PYZdl{}}\PY{l+s+s1}{\PYZbs{}}\PY{l+s+s1}{sigma=1\PYZdl{}}\PY{l+s+s1}{\PYZsq{}}\PY{p}{)}
         
         \PY{n}{ax3}\PY{o}{.}\PY{n}{imshow}\PY{p}{(}\PY{n}{edges2}\PY{p}{,} \PY{n}{cmap}\PY{o}{=}\PY{n}{plt}\PY{o}{.}\PY{n}{cm}\PY{o}{.}\PY{n}{gray}\PY{p}{)}
         \PY{n}{ax3}\PY{o}{.}\PY{n}{axis}\PY{p}{(}\PY{l+s+s1}{\PYZsq{}}\PY{l+s+s1}{off}\PY{l+s+s1}{\PYZsq{}}\PY{p}{)}
         \PY{n}{ax3}\PY{o}{.}\PY{n}{set\PYZus{}title}\PY{p}{(}\PY{l+s+s1}{\PYZsq{}}\PY{l+s+s1}{Canny filter, \PYZdl{}}\PY{l+s+s1}{\PYZbs{}}\PY{l+s+s1}{sigma=3\PYZdl{}}\PY{l+s+s1}{\PYZsq{}}\PY{p}{)}
         
         \PY{n}{plt}\PY{o}{.}\PY{n}{show}\PY{p}{(}\PY{p}{)}
\end{Verbatim}


    \begin{center}
    \adjustimage{max size={0.9\linewidth}{0.9\paperheight}}{output_26_0.png}
    \end{center}
    { \hspace*{\fill} \\}
    
    We can see that unlike Sobel or our Gaussian derivative filter, the
canny edge detector generates a more complete contour line that is
continuous. Even upon increasing \(\sigma\) we can observe that we do
not loose the continuity, but some of the smaller less relevant features
like eyebrows or tufts of hairline are not detected as edges. This
version would work better for something like face recognization feature.

    Though the homework ends here, I have included some optional bonus stuff
that illustrates some fun little uses of the things we've learnt so far.

    \hypertarget{optional-etch-a-sketch-effect}{%
\section{Optional: Etch-a-sketch
effect}\label{optional-etch-a-sketch-effect}}

    This is a fun little exercise where we can apply Sobel's operator with
some pre-processing to generate ``sketch-like'' effect. The
pre-processing entails a simple gaussian blur for making the edges more
prominent, then applying sobel operator and thresholding it to generate
what appears to be pencil strokes. (Also the thickness of the pencil is
determined by the strength of our blur)

    \begin{Verbatim}[commandchars=\\\{\}]
{\color{incolor}In [{\color{incolor}11}]:} \PY{k+kn}{from} \PY{n+nn}{skimage}\PY{n+nn}{.}\PY{n+nn}{color} \PY{k}{import} \PY{n}{rgb2gray}
         \PY{k+kn}{from} \PY{n+nn}{skimage}\PY{n+nn}{.}\PY{n+nn}{util} \PY{k}{import} \PY{n}{invert}
         \PY{k+kn}{from} \PY{n+nn}{skimage}\PY{n+nn}{.}\PY{n+nn}{filters} \PY{k}{import} \PY{n}{sobel}\PY{p}{,} \PY{n}{threshold\PYZus{}otsu}
         
         \PY{n}{keanu\PYZus{}colour} \PY{o}{=} \PY{n}{imread}\PY{p}{(}\PY{l+s+s1}{\PYZsq{}}\PY{l+s+s1}{keanu\PYZus{}colour.jpg}\PY{l+s+s1}{\PYZsq{}}\PY{p}{)}
         
         \PY{n}{blur\PYZus{}keanu} \PY{o}{=} \PY{n}{gaussian}\PY{p}{(}\PY{n}{keanu\PYZus{}colour}\PY{p}{)}
         
         \PY{n}{clean\PYZus{}keanu} \PY{o}{=} \PY{n}{rgb2gray}\PY{p}{(}\PY{n}{blur\PYZus{}keanu}\PY{p}{)}
         
         \PY{n}{edge\PYZus{}keanu} \PY{o}{=} \PY{n}{sobel}\PY{p}{(}\PY{n}{clean\PYZus{}keanu}\PY{p}{)}
         
         \PY{n}{thresh\PYZus{}mesh} \PY{o}{=} \PY{n}{threshold\PYZus{}otsu}\PY{p}{(}\PY{n}{edge\PYZus{}keanu}\PY{p}{)}
         
         \PY{n}{binary\PYZus{}keanu} \PY{o}{=} \PY{n}{edge\PYZus{}keanu} \PY{o}{\PYZgt{}} \PY{n}{thresh\PYZus{}mesh}
         
         \PY{n}{plt}\PY{o}{.}\PY{n}{imshow}\PY{p}{(}\PY{n}{invert}\PY{p}{(}\PY{n}{binary\PYZus{}keanu}\PY{p}{)}\PY{p}{,} \PY{n}{cmap} \PY{o}{=} \PY{l+s+s1}{\PYZsq{}}\PY{l+s+s1}{gray}\PY{l+s+s1}{\PYZsq{}}\PY{p}{)}
\end{Verbatim}


    \begin{Verbatim}[commandchars=\\\{\}]
/usr/lib/python3.7/site-packages/skimage/filters/\_gaussian.py:108: RuntimeWarning: Images with dimensions (M, N, 3) are interpreted as 2D+RGB by default. Use `multichannel=False` to interpret as 3D image with last dimension of length 3.
  warn(RuntimeWarning(msg))

    \end{Verbatim}

\begin{Verbatim}[commandchars=\\\{\}]
{\color{outcolor}Out[{\color{outcolor}11}]:} <matplotlib.image.AxesImage at 0x7f48dad8e4a8>
\end{Verbatim}
            
    \begin{center}
    \adjustimage{max size={0.9\linewidth}{0.9\paperheight}}{output_31_2.png}
    \end{center}
    { \hspace*{\fill} \\}
    
    \hypertarget{optional-a-scanner-darkly-effect-using-image-segmentation}{%
\section{Optional: A Scanner Darkly Effect using Image
Segmentation}\label{optional-a-scanner-darkly-effect-using-image-segmentation}}

    Note: The effect used in the movie is actually a
\href{https://www.wikiwand.com/en/Cel_shading}{cel-shading effect},
which is performed by computing the normals for world surfaces (which
uses 3D models) and blending them with the Sobel operator (used for
edges to give ``Toon''-lines). Since, we work with a 2D image for our
course, I will be using Image Segmentation to achieve a similar effect,
albeit one that's more simplistic looking.

    \begin{Verbatim}[commandchars=\\\{\}]
{\color{incolor}In [{\color{incolor}12}]:} \PY{k+kn}{from} \PY{n+nn}{skimage} \PY{k}{import} \PY{n}{data}\PY{p}{,} \PY{n}{io}\PY{p}{,} \PY{n}{segmentation}\PY{p}{,} \PY{n}{color}
         \PY{k+kn}{from} \PY{n+nn}{skimage}\PY{n+nn}{.}\PY{n+nn}{future} \PY{k}{import} \PY{n}{graph}
         \PY{k+kn}{from} \PY{n+nn}{matplotlib} \PY{k}{import} \PY{n}{pyplot} \PY{k}{as} \PY{n}{plt}
         
         
         \PY{n}{keanu\PYZus{}colour} \PY{o}{=} \PY{n}{imread}\PY{p}{(}\PY{l+s+s1}{\PYZsq{}}\PY{l+s+s1}{keanu\PYZus{}colour.jpg}\PY{l+s+s1}{\PYZsq{}}\PY{p}{)}
         
         \PY{n}{labels1} \PY{o}{=} \PY{n}{segmentation}\PY{o}{.}\PY{n}{quickshift}\PY{p}{(}\PY{n}{keanu\PYZus{}colour}\PY{p}{,} \PY{n}{kernel\PYZus{}size}\PY{o}{=}\PY{l+m+mi}{7}\PY{p}{,} \PY{n}{max\PYZus{}dist}\PY{o}{=}\PY{l+m+mi}{6}\PY{p}{,} \PY{n}{ratio}\PY{o}{=}\PY{l+m+mf}{0.5}\PY{p}{)}
         \PY{n}{out1} \PY{o}{=} \PY{n}{color}\PY{o}{.}\PY{n}{label2rgb}\PY{p}{(}\PY{n}{labels1}\PY{p}{,} \PY{n}{keanu\PYZus{}colour}\PY{p}{,} \PY{n}{kind}\PY{o}{=}\PY{l+s+s1}{\PYZsq{}}\PY{l+s+s1}{avg}\PY{l+s+s1}{\PYZsq{}}\PY{p}{)}
         
         \PY{n}{g} \PY{o}{=} \PY{n}{graph}\PY{o}{.}\PY{n}{rag\PYZus{}mean\PYZus{}color}\PY{p}{(}\PY{n}{keanu\PYZus{}colour}\PY{p}{,} \PY{n}{labels1}\PY{p}{)}
         \PY{n}{labels2} \PY{o}{=} \PY{n}{graph}\PY{o}{.}\PY{n}{cut\PYZus{}threshold}\PY{p}{(}\PY{n}{labels1}\PY{p}{,} \PY{n}{g}\PY{p}{,} \PY{l+m+mi}{20}\PY{p}{)}
         \PY{n}{out2} \PY{o}{=} \PY{n}{color}\PY{o}{.}\PY{n}{label2rgb}\PY{p}{(}\PY{n}{labels2}\PY{p}{,} \PY{n}{keanu\PYZus{}colour}\PY{p}{,} \PY{n}{kind}\PY{o}{=}\PY{l+s+s1}{\PYZsq{}}\PY{l+s+s1}{avg}\PY{l+s+s1}{\PYZsq{}}\PY{p}{)}
         \PY{n}{out2} \PY{o}{=} \PY{n}{segmentation}\PY{o}{.}\PY{n}{mark\PYZus{}boundaries}\PY{p}{(}\PY{n}{out2}\PY{p}{,} \PY{n}{labels2}\PY{p}{,} \PY{p}{(}\PY{l+m+mi}{0}\PY{p}{,} \PY{l+m+mi}{0}\PY{p}{,} \PY{l+m+mi}{0}\PY{p}{)}\PY{p}{)}
         \PY{n}{fig}\PY{p}{,} \PY{n}{ax} \PY{o}{=} \PY{n}{plt}\PY{o}{.}\PY{n}{subplots}\PY{p}{(}\PY{n}{nrows}\PY{o}{=}\PY{l+m+mi}{2}\PY{p}{,} \PY{n}{sharex}\PY{o}{=}\PY{k+kc}{True}\PY{p}{,} \PY{n}{sharey}\PY{o}{=}\PY{k+kc}{True}\PY{p}{,} \PY{n}{dpi}\PY{o}{=}\PY{l+m+mi}{200}\PY{p}{)}
         
         \PY{n}{ax}\PY{p}{[}\PY{l+m+mi}{0}\PY{p}{]}\PY{o}{.}\PY{n}{imshow}\PY{p}{(}\PY{n}{out1}\PY{p}{)}
         \PY{n}{ax}\PY{p}{[}\PY{l+m+mi}{0}\PY{p}{]}\PY{o}{.}\PY{n}{set\PYZus{}title}\PY{p}{(}\PY{l+s+s1}{\PYZsq{}}\PY{l+s+s1}{Initial quickshift segmentation}\PY{l+s+s1}{\PYZsq{}}\PY{p}{)}
         \PY{n}{ax}\PY{p}{[}\PY{l+m+mi}{1}\PY{p}{]}\PY{o}{.}\PY{n}{imshow}\PY{p}{(}\PY{n}{out2}\PY{p}{)}
         \PY{n}{ax}\PY{p}{[}\PY{l+m+mi}{1}\PY{p}{]}\PY{o}{.}\PY{n}{set\PYZus{}title}\PY{p}{(}\PY{l+s+s1}{\PYZsq{}}\PY{l+s+s1}{Applying regional averaging}\PY{l+s+s1}{\PYZsq{}}\PY{p}{)}
         \PY{k}{for} \PY{n}{a} \PY{o+ow}{in} \PY{n}{ax}\PY{p}{:}
             \PY{n}{a}\PY{o}{.}\PY{n}{axis}\PY{p}{(}\PY{l+s+s1}{\PYZsq{}}\PY{l+s+s1}{off}\PY{l+s+s1}{\PYZsq{}}\PY{p}{)}
         
         \PY{c+c1}{\PYZsh{}plt.tight\PYZus{}layout()}
\end{Verbatim}


    \begin{center}
    \adjustimage{max size={0.9\linewidth}{0.9\paperheight}}{output_34_0.png}
    \end{center}
    { \hspace*{\fill} \\}
    

    % Add a bibliography block to the postdoc
    
    
    
    \end{document}
